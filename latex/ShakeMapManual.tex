% Generated by Sphinx.
\def\sphinxdocclass{report}
\documentclass[letterpaper,10pt,english]{sphinxmanual}

\usepackage[utf8]{inputenc}
\ifdefined\DeclareUnicodeCharacter
  \DeclareUnicodeCharacter{00A0}{\nobreakspace}
\else\fi
\usepackage{cmap}
\usepackage[T1]{fontenc}
\usepackage{amsmath,amssymb}
\usepackage{babel}
\usepackage{times}
\usepackage[Bjarne]{fncychap}
\usepackage{longtable}
\usepackage{sphinx}
\usepackage{multirow}
\usepackage{eqparbox}


\addto\captionsenglish{\renewcommand{\figurename}{Fig. }}
\addto\captionsenglish{\renewcommand{\tablename}{Table }}
\SetupFloatingEnvironment{literal-block}{name=Listing }

\addto\extrasenglish{\def\pageautorefname{page}}

\setcounter{tocdepth}{1}


\title{ShakeMap Manual}
\date{Jul 27, 2016}
\release{2.0}
\author{C. Bruce Worden, David Wald}
\newcommand{\sphinxlogo}{}
\renewcommand{\releasename}{Release}
\makeindex

\makeatletter
\def\PYG@reset{\let\PYG@it=\relax \let\PYG@bf=\relax%
    \let\PYG@ul=\relax \let\PYG@tc=\relax%
    \let\PYG@bc=\relax \let\PYG@ff=\relax}
\def\PYG@tok#1{\csname PYG@tok@#1\endcsname}
\def\PYG@toks#1+{\ifx\relax#1\empty\else%
    \PYG@tok{#1}\expandafter\PYG@toks\fi}
\def\PYG@do#1{\PYG@bc{\PYG@tc{\PYG@ul{%
    \PYG@it{\PYG@bf{\PYG@ff{#1}}}}}}}
\def\PYG#1#2{\PYG@reset\PYG@toks#1+\relax+\PYG@do{#2}}

\expandafter\def\csname PYG@tok@sd\endcsname{\let\PYG@it=\textit\def\PYG@tc##1{\textcolor[rgb]{0.25,0.44,0.63}{##1}}}
\expandafter\def\csname PYG@tok@err\endcsname{\def\PYG@bc##1{\setlength{\fboxsep}{0pt}\fcolorbox[rgb]{1.00,0.00,0.00}{1,1,1}{\strut ##1}}}
\expandafter\def\csname PYG@tok@sx\endcsname{\def\PYG@tc##1{\textcolor[rgb]{0.78,0.36,0.04}{##1}}}
\expandafter\def\csname PYG@tok@kr\endcsname{\let\PYG@bf=\textbf\def\PYG@tc##1{\textcolor[rgb]{0.00,0.44,0.13}{##1}}}
\expandafter\def\csname PYG@tok@se\endcsname{\let\PYG@bf=\textbf\def\PYG@tc##1{\textcolor[rgb]{0.25,0.44,0.63}{##1}}}
\expandafter\def\csname PYG@tok@kp\endcsname{\def\PYG@tc##1{\textcolor[rgb]{0.00,0.44,0.13}{##1}}}
\expandafter\def\csname PYG@tok@ge\endcsname{\let\PYG@it=\textit}
\expandafter\def\csname PYG@tok@nl\endcsname{\let\PYG@bf=\textbf\def\PYG@tc##1{\textcolor[rgb]{0.00,0.13,0.44}{##1}}}
\expandafter\def\csname PYG@tok@nt\endcsname{\let\PYG@bf=\textbf\def\PYG@tc##1{\textcolor[rgb]{0.02,0.16,0.45}{##1}}}
\expandafter\def\csname PYG@tok@sc\endcsname{\def\PYG@tc##1{\textcolor[rgb]{0.25,0.44,0.63}{##1}}}
\expandafter\def\csname PYG@tok@cp\endcsname{\def\PYG@tc##1{\textcolor[rgb]{0.00,0.44,0.13}{##1}}}
\expandafter\def\csname PYG@tok@ne\endcsname{\def\PYG@tc##1{\textcolor[rgb]{0.00,0.44,0.13}{##1}}}
\expandafter\def\csname PYG@tok@vg\endcsname{\def\PYG@tc##1{\textcolor[rgb]{0.73,0.38,0.84}{##1}}}
\expandafter\def\csname PYG@tok@ss\endcsname{\def\PYG@tc##1{\textcolor[rgb]{0.32,0.47,0.09}{##1}}}
\expandafter\def\csname PYG@tok@gi\endcsname{\def\PYG@tc##1{\textcolor[rgb]{0.00,0.63,0.00}{##1}}}
\expandafter\def\csname PYG@tok@gd\endcsname{\def\PYG@tc##1{\textcolor[rgb]{0.63,0.00,0.00}{##1}}}
\expandafter\def\csname PYG@tok@sb\endcsname{\def\PYG@tc##1{\textcolor[rgb]{0.25,0.44,0.63}{##1}}}
\expandafter\def\csname PYG@tok@gt\endcsname{\def\PYG@tc##1{\textcolor[rgb]{0.00,0.27,0.87}{##1}}}
\expandafter\def\csname PYG@tok@nn\endcsname{\let\PYG@bf=\textbf\def\PYG@tc##1{\textcolor[rgb]{0.05,0.52,0.71}{##1}}}
\expandafter\def\csname PYG@tok@mh\endcsname{\def\PYG@tc##1{\textcolor[rgb]{0.13,0.50,0.31}{##1}}}
\expandafter\def\csname PYG@tok@sr\endcsname{\def\PYG@tc##1{\textcolor[rgb]{0.14,0.33,0.53}{##1}}}
\expandafter\def\csname PYG@tok@cs\endcsname{\def\PYG@tc##1{\textcolor[rgb]{0.25,0.50,0.56}{##1}}\def\PYG@bc##1{\setlength{\fboxsep}{0pt}\colorbox[rgb]{1.00,0.94,0.94}{\strut ##1}}}
\expandafter\def\csname PYG@tok@k\endcsname{\let\PYG@bf=\textbf\def\PYG@tc##1{\textcolor[rgb]{0.00,0.44,0.13}{##1}}}
\expandafter\def\csname PYG@tok@ow\endcsname{\let\PYG@bf=\textbf\def\PYG@tc##1{\textcolor[rgb]{0.00,0.44,0.13}{##1}}}
\expandafter\def\csname PYG@tok@mb\endcsname{\def\PYG@tc##1{\textcolor[rgb]{0.13,0.50,0.31}{##1}}}
\expandafter\def\csname PYG@tok@mi\endcsname{\def\PYG@tc##1{\textcolor[rgb]{0.13,0.50,0.31}{##1}}}
\expandafter\def\csname PYG@tok@mo\endcsname{\def\PYG@tc##1{\textcolor[rgb]{0.13,0.50,0.31}{##1}}}
\expandafter\def\csname PYG@tok@sh\endcsname{\def\PYG@tc##1{\textcolor[rgb]{0.25,0.44,0.63}{##1}}}
\expandafter\def\csname PYG@tok@no\endcsname{\def\PYG@tc##1{\textcolor[rgb]{0.38,0.68,0.84}{##1}}}
\expandafter\def\csname PYG@tok@m\endcsname{\def\PYG@tc##1{\textcolor[rgb]{0.13,0.50,0.31}{##1}}}
\expandafter\def\csname PYG@tok@vc\endcsname{\def\PYG@tc##1{\textcolor[rgb]{0.73,0.38,0.84}{##1}}}
\expandafter\def\csname PYG@tok@c1\endcsname{\let\PYG@it=\textit\def\PYG@tc##1{\textcolor[rgb]{0.25,0.50,0.56}{##1}}}
\expandafter\def\csname PYG@tok@gu\endcsname{\let\PYG@bf=\textbf\def\PYG@tc##1{\textcolor[rgb]{0.50,0.00,0.50}{##1}}}
\expandafter\def\csname PYG@tok@kd\endcsname{\let\PYG@bf=\textbf\def\PYG@tc##1{\textcolor[rgb]{0.00,0.44,0.13}{##1}}}
\expandafter\def\csname PYG@tok@go\endcsname{\def\PYG@tc##1{\textcolor[rgb]{0.20,0.20,0.20}{##1}}}
\expandafter\def\csname PYG@tok@kc\endcsname{\let\PYG@bf=\textbf\def\PYG@tc##1{\textcolor[rgb]{0.00,0.44,0.13}{##1}}}
\expandafter\def\csname PYG@tok@w\endcsname{\def\PYG@tc##1{\textcolor[rgb]{0.73,0.73,0.73}{##1}}}
\expandafter\def\csname PYG@tok@cpf\endcsname{\let\PYG@it=\textit\def\PYG@tc##1{\textcolor[rgb]{0.25,0.50,0.56}{##1}}}
\expandafter\def\csname PYG@tok@si\endcsname{\let\PYG@it=\textit\def\PYG@tc##1{\textcolor[rgb]{0.44,0.63,0.82}{##1}}}
\expandafter\def\csname PYG@tok@cm\endcsname{\let\PYG@it=\textit\def\PYG@tc##1{\textcolor[rgb]{0.25,0.50,0.56}{##1}}}
\expandafter\def\csname PYG@tok@gp\endcsname{\let\PYG@bf=\textbf\def\PYG@tc##1{\textcolor[rgb]{0.78,0.36,0.04}{##1}}}
\expandafter\def\csname PYG@tok@ch\endcsname{\let\PYG@it=\textit\def\PYG@tc##1{\textcolor[rgb]{0.25,0.50,0.56}{##1}}}
\expandafter\def\csname PYG@tok@s\endcsname{\def\PYG@tc##1{\textcolor[rgb]{0.25,0.44,0.63}{##1}}}
\expandafter\def\csname PYG@tok@bp\endcsname{\def\PYG@tc##1{\textcolor[rgb]{0.00,0.44,0.13}{##1}}}
\expandafter\def\csname PYG@tok@nd\endcsname{\let\PYG@bf=\textbf\def\PYG@tc##1{\textcolor[rgb]{0.33,0.33,0.33}{##1}}}
\expandafter\def\csname PYG@tok@nf\endcsname{\def\PYG@tc##1{\textcolor[rgb]{0.02,0.16,0.49}{##1}}}
\expandafter\def\csname PYG@tok@o\endcsname{\def\PYG@tc##1{\textcolor[rgb]{0.40,0.40,0.40}{##1}}}
\expandafter\def\csname PYG@tok@nv\endcsname{\def\PYG@tc##1{\textcolor[rgb]{0.73,0.38,0.84}{##1}}}
\expandafter\def\csname PYG@tok@c\endcsname{\let\PYG@it=\textit\def\PYG@tc##1{\textcolor[rgb]{0.25,0.50,0.56}{##1}}}
\expandafter\def\csname PYG@tok@il\endcsname{\def\PYG@tc##1{\textcolor[rgb]{0.13,0.50,0.31}{##1}}}
\expandafter\def\csname PYG@tok@nb\endcsname{\def\PYG@tc##1{\textcolor[rgb]{0.00,0.44,0.13}{##1}}}
\expandafter\def\csname PYG@tok@vi\endcsname{\def\PYG@tc##1{\textcolor[rgb]{0.73,0.38,0.84}{##1}}}
\expandafter\def\csname PYG@tok@gr\endcsname{\def\PYG@tc##1{\textcolor[rgb]{1.00,0.00,0.00}{##1}}}
\expandafter\def\csname PYG@tok@mf\endcsname{\def\PYG@tc##1{\textcolor[rgb]{0.13,0.50,0.31}{##1}}}
\expandafter\def\csname PYG@tok@ni\endcsname{\let\PYG@bf=\textbf\def\PYG@tc##1{\textcolor[rgb]{0.84,0.33,0.22}{##1}}}
\expandafter\def\csname PYG@tok@na\endcsname{\def\PYG@tc##1{\textcolor[rgb]{0.25,0.44,0.63}{##1}}}
\expandafter\def\csname PYG@tok@kn\endcsname{\let\PYG@bf=\textbf\def\PYG@tc##1{\textcolor[rgb]{0.00,0.44,0.13}{##1}}}
\expandafter\def\csname PYG@tok@nc\endcsname{\let\PYG@bf=\textbf\def\PYG@tc##1{\textcolor[rgb]{0.05,0.52,0.71}{##1}}}
\expandafter\def\csname PYG@tok@kt\endcsname{\def\PYG@tc##1{\textcolor[rgb]{0.56,0.13,0.00}{##1}}}
\expandafter\def\csname PYG@tok@s1\endcsname{\def\PYG@tc##1{\textcolor[rgb]{0.25,0.44,0.63}{##1}}}
\expandafter\def\csname PYG@tok@s2\endcsname{\def\PYG@tc##1{\textcolor[rgb]{0.25,0.44,0.63}{##1}}}
\expandafter\def\csname PYG@tok@gs\endcsname{\let\PYG@bf=\textbf}
\expandafter\def\csname PYG@tok@gh\endcsname{\let\PYG@bf=\textbf\def\PYG@tc##1{\textcolor[rgb]{0.00,0.00,0.50}{##1}}}

\def\PYGZbs{\char`\\}
\def\PYGZus{\char`\_}
\def\PYGZob{\char`\{}
\def\PYGZcb{\char`\}}
\def\PYGZca{\char`\^}
\def\PYGZam{\char`\&}
\def\PYGZlt{\char`\<}
\def\PYGZgt{\char`\>}
\def\PYGZsh{\char`\#}
\def\PYGZpc{\char`\%}
\def\PYGZdl{\char`\$}
\def\PYGZhy{\char`\-}
\def\PYGZsq{\char`\'}
\def\PYGZdq{\char`\"}
\def\PYGZti{\char`\~}
% for compatibility with earlier versions
\def\PYGZat{@}
\def\PYGZlb{[}
\def\PYGZrb{]}
\makeatother

\renewcommand\PYGZsq{\textquotesingle}

\begin{document}

\maketitle
\tableofcontents
\phantomsection\label{shake_index::doc}



\chapter{Introduction}
\label{introduction::doc}\label{introduction:shakemap-manual}\label{introduction:id1}\label{introduction:introduction}
This ShakeMap Manual ({\hyperref[references:worden2016b]{\crossref{\DUrole{std,std-ref}{Worden and Wald, 2016}}}}), is a
significant update of the
original ({\hyperref[references:wald2005]{\crossref{\DUrole{std,std-ref}{Wald et al., 2005}}}}) ShakeMap Manual.  We employ Python document
generator \href{http://sphinx-doc.org/}{Sphinx}, with the source
under \href{http://github.com/}{GitHub} version control.
The online version of the manual is available at \url{http://usgs.github.io/shakemap}.
\begin{figure}[htbp]
\centering
\capstart
\scalebox{0.700000}{\includegraphics{{Napa_ShakeMap_cover}.png}}\caption{2014 M6.0 Napa, CA, earthquake intensity ShakeMap.}\label{introduction:id5}\end{figure}

\href{http://earthquake.usgs.gov/shakemap/}{ShakeMap®},
developed by the U.S. Geological Survey (USGS), facilitates communication of
earthquake information beyond just magnitude and location. By rapidly mapping out
earthquake ground motions, ShakeMap portrays the distribution and severity of shaking.
This information is critical for gauging the extent of the areas affected, determining which areas
are potentially hardest hit, and allowing for rapid estimation of losses. Key to
ShakeMap's success, algorithms were developed that take advantage of any high-quality
recorded ground motions---and any available macroseismic intensity data---to provide
ground-truth constraints on shaking. Yet ShakeMap also utilizes best practices
for both interpolating recordings and---critically---providing
event-specific estimates of shaking in areas where observations are sparse or nonexistent. Thus, ShakeMap portrays
the best possible description of shaking by employing a combination of recorded and
estimated shaking values.

This Manual provides background on technical aspects of ShakeMap including: 1) information on
the wide range of products and formats ShakeMap produces, 2) the uses of these products,
and 3) guidance for
ShakeMap developers and operators. Readers interested in understanding how
ShakeMaps works can navigate to the {\hyperref[technical_guide:technical\string-guide]{\crossref{\DUrole{std,std-ref}{Technical Guide}}}}. Those who want to use
ShakeMap products and understand their varied forms can jump to the {\hyperref[users_guide:users\string-guide]{\crossref{\DUrole{std,std-ref}{User's Guide}}}}.
If your goal is to install and operate ShakeMap, see the {\hyperref[software_guide:software\string-guide]{\crossref{\DUrole{std,std-ref}{Software \& Implementation Guide}}}}. The
{\hyperref[software_guide:software\string-guide]{\crossref{\DUrole{std,std-ref}{Software \& Implementation Guide}}}} also points users to the ShakeMap software distribution and
\href{http://usgs.github.io/shakemap/\_static/SoftwareGuideV3\_5.pdf}{Software Guide}.


\chapter{Technical Guide}
\label{technical_guide::doc}\label{technical_guide:id1}\label{technical_guide:technical-guide}
This ShakeMap Technical Guide is meant as the definitive source of information
pertaining to the generation of ShakeMaps.  Many of the initial descriptions in {\hyperref[references:wald1999a]{\crossref{\DUrole{std,std-ref}{Wald et al.
(1999a)}}}} are outdated and are superseded by this current report.  Technical users of
ShakeMap should also consult the {\hyperref[users_guide:users\string-guide]{\crossref{\DUrole{std,std-ref}{User's Guide}}}} for additional information
pertaining to the format and range of available ShakeMap products.

Throughout this document, all parameters are italicized; specific parameters that can be configured within the
ShakeMap software are indicated in parentheses.  These configurable
parameters are further described in the ShakeMap {\hyperref[software_guide:sm35\string-software\string-guide]{\crossref{\DUrole{std,std-ref}{Software Guide}}}}.


\section{ShakeMap Software Overview}
\label{tg_software::doc}\label{tg_software:shakemap-software-overview}\label{tg_software:sec-software-overview}
ShakeMap is a collection of modules written in PERL and C.  PERL is a powerful, freely
available scripting language that runs on all computer platforms.  The collection of PERL
modules allows the processing to flow in discrete steps that can be run collectively or
individually.  Within the PERL scripts, other software packages are called, specifically
packages that enable the graphics and much of the heavy grid-based computation.  For
instance, maps are made using the Generic Mapping Tool (GMT; {\hyperref[references:gmt\string-ref]{\crossref{\DUrole{std,std-ref}{Wessel and Smith,
1991)}}}}, and the Postscript output from GMT is converted to JPEG format using
\href{http://www.imagemagick.org/script/index.php}{ImageMagick} or
\href{http://www.graphicsmagick.org/}{GraphicsMagick}.  In the design of ShakeMap,
all components are built
from freely available, open-source packages.

While the PERL language is not the fastest possible way to implement ShakeMap, we
note that much of the heavy computational load is handled by highly optimized programs
(usually written in C) that are called from within the PERL programs. Even for networks
with hundreds of stations over large regions, ShakeMap takes only a minute or so to run
on a modern computer (and much of that time is spent in product generation, e.g.,
converting PostScript images to JPEG---something that would be very difficult to
optimize further).

To enable customization for specific earthquakes or for different regions, each ShakeMap
module has an accompanying collection of configuration files.  For example, in these
files, one assigns the regional geographic boundaries and mapping characteristics to be
used by GMT, which ground motion prediction equation (GMPE) to use, where and how
to transfer the maps, email recipient lists, and so on.  Specific details about the software
and configuration files are described in detail in the {\hyperref[software_guide:sm35\string-software\string-guide]{\crossref{\DUrole{std,std-ref}{Software Guide}}}}.

With standardization in GIS and web application interfaces (API), several aspects of the
ShakeMap system could be accomplished within GIS applications, but the open-source,
freely available nature of GMT combined with PERL scripting tools allows for a flexible
and readily available ShakeMap software package.  Nonetheless, we do generate a
number of GIS product formats for that widespread user group as described in the {\hyperref[users_guide:users\string-guide]{\crossref{\DUrole{std,std-ref}{User's Guide}}}}.


\section{Philosophy of Estimating and Interpolating Ground Motions}
\label{tg_philosophy::doc}\label{tg_philosophy:sec-philosophy}\label{tg_philosophy:philosophy-of-estimating-and-interpolating-ground-motions}
The overall strategy for the deployment of stations under the ANSS implementation plan
relies on dense instrumentation concentrated in urban areas with high seismic hazards
({\hyperref[references:usgs1999]{\crossref{\DUrole{std,std-ref}{USGS, 1999}}}}) and fewer stations in outlying areas.  Based on this philosophy, and when
fully deployed, maps generated in these urban regions are expected to be most accurate
where the population at risk is the greatest, and therefore, where emergency response and
recovery efforts will likely be most urgent and complex.

Even so, significant gaps in the observed shaking distribution will likely remain,
especially in the transition from urban to more rural environments. Likewise, many
critical facilities and lifelines are widely distributed, away from population centers and
their dense seismic sensor networks.  Thus, as a fundamental strategy for ShakeMap, we
have developed algorithms to best describe the shaking in more remote areas by utilizing
a variety of seismological tools.  In addition to the areas without sufficient
instrumentation where we would like to estimate motions to help assess the situation, and
as a fail-safe backup, it is also useful to have these algorithms in place in
the event of potential communication dropout from a portion of the network.  The same
tools are, in fact, beneficial for interpolating between observations (i.e., seismic stations) even
within densely instrumented portions of the networks.

If there were stations at each of the tens of thousands of map grid points needed to
adequately portray shaking, then the creation of shaking maps would be relatively simple.
Of course, stations are not available for the overwhelming majority of these grid points, and in many cases grid
points may be tens of kilometers or more from the nearest reporting station.  The overall mapping
philosophy is then to combine information from individual stations, site amplification
characteristics, and ground-motion prediction equations for the distance to the hypocenter
(or to the causative fault) to create the best composite map.  The procedure should
produce reasonable estimates at grid points located far from available data while
preserving the detailed shaking information available for regions where there are stations
nearby.

It should be mentioned that mathematically, or algorithmically, geospatial interpolation
can take many forms. There are some good reasons to employ geospatial kriging-with-a-trend.
However, the complexity of the trends (GMPE, as well as inter-event bias
corrections per Intensity Measure or IM), the use of multiply-weighted strong-motion and macroseimic
data, and the real-time nature of the processing require other considerations. Effectively,
the approach ShakeMap currently employs for interpolation ({\hyperref[references:worden2010]{\crossref{\DUrole{std,std-ref}{Worden et al., 2010}}}}), which
employs a predetermined spatial correlation function, is broadly analogous to \href{https://en.wikipedia.org/wiki/Kriging}{kriging-with-a-trend}
mathematically. We address this possibility further in {\hyperref[future_directions:future\string-directions]{\crossref{\DUrole{std,std-ref}{Future Directions}}}}.

Estimating motions where there are few stations, and then interpolating the recordings and
estimates to a fine grid for mapping and contouring, requires several steps. In the
following sections, we describe the process from input to final interpolated grid. Where
beneficial, we illustrate the effects of key steps with example ShakeMap figures.


\section{Recorded Ground-motion Parameters}
\label{tg_parameters::doc}\label{tg_parameters:sec-tg-parameters}\label{tg_parameters:recorded-ground-motion-parameters}

\subsection{Data Acquisition}
\label{tg_parameters:data-acquisition}
ShakeMap requires estimates of magnitude, location, and (optionally, but preferably)
shaking IMs at seismic stations. As such, ShakeMap has been interfaced with
several types of seismic processing systems in wide use at numerous networks across the U.S. and
around the world, including \href{http://antelopeusersgroup.org/}{Antelope},
\href{https://www.seiscomp3.org/}{SeisComP3}, and \href{http://www.isti.com/products/aqms/}{AQMS}.
The ShakeMap system,
however, is a stand-alone software package and is really a passive consumer of seismic
data. In other words, the ShakeMap software itself contains no data acquisition component. It is
assumed that ShakeMap earthquake source information and parametric data are packaged
for delivery to ShakeMap and that that delivery will trigger a ShakeMap run. The
required format is an XML format, as fully described in the {\hyperref[software_guide:sm35\string-software\string-guide]{\crossref{\DUrole{std,std-ref}{Software Guide}}}}.
Some programs are provided to convert ASCII text and other formats to the required input
XML.
It is assumed that station data delivered to ShakeMap are free-field sites that have been
vetted by the contributing network. Each station must have stand-alone metadata
describing its station location, contributing network, channel, and location code. While some
additional outlier and data quality checks are performed within ShakeMap (see
{\hyperref[tg_processing:sec\string-shakemap\string-processing]{\crossref{\DUrole{std,std-ref}{ShakeMap Processing}}}}), it is assumed that this is primarily the
responsibility of the contributing seismic network.

For global and historic earthquake ShakeMap generation, we have developed scripts to
preprocess various forms of seismic waveform (as well as macroseismic) data which are
openly available around the world. For example, we provide a Python
script \href{https://github.com/mhearne-usgs/smtools}{getstrong.py}
that runs independently of ShakeMap, as described in the {\hyperref[software_guide:software\string-guide]{\crossref{\DUrole{std,std-ref}{Software \& Implementation Guide}}}}.

For illustrative purposes, we describe the data acquisition for the seismic system in
Southern California, a component of the California Integrated Seismic Network (\href{http://www.cisn.org}{CISN}).
For perspective, as of 2015, there were nearly 800 real-time stations jointly operated with
a collaboration between the USGS and the California Institute of Technology (Caltech). In addition, the California Geological Survey (CGS)
contributes nearly 350 strong-motion stations in near real-time, utilizing an automated
telephone dial-up procedure ({\hyperref[references:shakal1998]{\crossref{\DUrole{std,std-ref}{Shakal et al, 1998}}}}), and the USGS National Strong Motion
Instrumentation Program (NSMP) contributes dial-up station parameters as well, with
nearly 50 stations in Southern California alone.  Lastly, the
\href{http://earthquake.usgs.gov/monitoring/netquakes/map}{``NetQuakes''} program, a
relatively low-cost seismograph that the USGS installs in homes, businesses, buildings, and
schools, contributes close to 100 additional stations in Southern California.

Generation of ShakeMap in Southern California is automatic, triggered by the event
associator of the seismic network.  Within the first two minutes of an earthquake, ground-motion
parameters are available from the USGS-Caltech component of the network, and
within several minutes most of the important near-source CGS and NSMP stations contribute; a
more complete CGS and NSMP contribution is available within approximately
ten minutes of an event.  Initial maps are made with the real-time component of CISN as
well as any of the available dial-sites, and they are updated automatically as more data
are acquired.


\subsection{Derived Parametric Ground-motion Values}
\label{tg_parameters:derived-parametric-ground-motion-values}
Parametric data from stations serving ShakeMap should include peak ground acceleration
(PGA), peak ground velocity (PGV), and peak response spectral acceleration amplitudes
(at 0.3, 1.0, and 3.0 sec).  Often, parametric values are derived continuously, using
recursive time-domain filtering as described by {\hyperref[references:kanamori1999]{\crossref{\DUrole{std,std-ref}{Kanamori et al. (1999)}}}}.  Otherwise
parameters are derived from post-processing as described by {\hyperref[references:shakal1998]{\crossref{\DUrole{std,std-ref}{Shakal et al. (1998)}}}} and
{\hyperref[references:converse1992]{\crossref{\DUrole{std,std-ref}{Converse and Brady (1992)}}}}.

ShakeMap will run successfully with no (or limited) parametric data, for example if only
PGA values are available at each station. Default logic is employed to provide reasonable
behavior for estimating intensities from PGA alone, bias correction, and interpolation
(see following sections). Likewise, for smaller-magnitude earthquakes, spectral values
can be noisy, so operators often omit the generation spectral maps below a lower
magnitude threshold (about M4); this can be done with simple command-line options.

For all maps and products, the motions depicted are peak values as observed; that is, the
maximum value observed on the two horizontal components of motion.  Many engineers
are accustomed to analyses employing the geometric mean of the horizontal peak-ground
motions, but that parameter is not computed by ShakeMap. More description and justification
for this strategy is given in the section {\hyperref[tg_choice_of_parameters:peak\string-values\string-vs\string-mean]{\crossref{\DUrole{std,std-ref}{Use of Peak Values Rather than Mean}}}}. It should be noted,
however, that conversions from peak to geometric mean {[}or orientation-independent
parameterizations ({\hyperref[references:boore2010]{\crossref{\DUrole{std,std-ref}{Boore, 2010}}}}){]} are available
(e.g., {\hyperref[references:beyer2006]{\crossref{\DUrole{std,std-ref}{Beyer and Bommer, 2006}}}}).


\subsection{Macroseismic Intensity}
\label{tg_parameters:macroseismic-intensity}
ShakeMap also (optionally) accepts input data in the form of observed macroseismic
intensity (MMI, MCS, etc.). As with peak ground motion parameters from seismic
stations, ShakeMap expects specific file formats (XML) and site metadata for
macroseismic data (see the {\hyperref[software_guide:sm35\string-software\string-guide]{\crossref{\DUrole{std,std-ref}{Software Guide}}}}).

Intensity data can fill important gaps where ground-motion recordings are not available,
and often provide the only control in sparsely instrumented areas. This is particularly true
for historic earthquakes, for which macroseismic data provide important constraints on
shaking intensities. As later discussed, the ShakeMap Atlas ({\hyperref[references:allen2008]{\crossref{\DUrole{std,std-ref}{Allen et al., 2008}}}}, {\hyperref[references:allen2009a]{\crossref{\DUrole{std,std-ref}{2009a}}}};
{\hyperref[references:garcia2012a]{\crossref{\DUrole{std,std-ref}{Garcia et al., 2012a}}}}) is a collection of important historic earthquake shaking maps which
are now widely used for scientific analyses and for loss model calibration (e.g., {\hyperref[references:wald2008]{\crossref{\DUrole{std,std-ref}{Wald et
al., 2008}}}}; {\hyperref[references:jaiswal2010]{\crossref{\DUrole{std,std-ref}{Jaiswal and Wald, 2010}}}}; {\hyperref[references:pomonis2011]{\crossref{\DUrole{std,std-ref}{Pomonis and So, 2011)}}}}.

The most common source for immediate post-earthquake intensity data is the USGS's
“Did You Feel It?” (DYFI) system ({\hyperref[references:wald2011c]{\crossref{\DUrole{std,std-ref}{Wald et al., 2011}}}}), though similar systems are
available in several countries. However, traditionally assigned intensities may be used as
well. DYFI data can be programmatically retrieved from the USGS's database and
formatted for ShakeMap input using the ShakeMap program \emph{getdyfi}, making it
especially easy to incorporate into the ShakeMap data input stream.

Macroseismic intensity data can also be an important constraint on peak ground motions,
since ground motion amplitudes can be derived from intensity through the use of a suitable Ground-Motion/Intensity
Conversion Equation (GMICE). Because a GMICE represents a statistical (probabilistic)
relationship, the conversion to and from intensity has a higher uncertainty than direct
ground-motion observation. ShakeMap accounts for this higher uncertainty by down-
weighting converted observations in the interpolation process, as discussed in the
{\hyperref[tg_processing:sec\string-interpolation]{\crossref{\DUrole{std,std-ref}{Interpolation}}}} section.

A variety of GMICEs are available with the ShakeMap software distribution, both for
MMI---based on {\hyperref[references:wald1999b]{\crossref{\DUrole{std,std-ref}{Wald, et al. (1999b)}}}}, {\hyperref[references:worden2012]{\crossref{\DUrole{std,std-ref}{Worden, et al. (2012)}}}},
and {\hyperref[references:atkinson2007]{\crossref{\DUrole{std,std-ref}{Atkinson and Kaka (2007)}}}}, among others---and for MCS---based on {\hyperref[references:faenza2010]{\crossref{\DUrole{std,std-ref}{Faenza and Michilini (2010)}}}}. Operators are
encouraged to explore the need to develop their own relationships based on data covering
their own operational area as GMICEs have been shown to have regional dependencies
(e.g., {\hyperref[references:caprio2015]{\crossref{\DUrole{std,std-ref}{Caprio et al., 2015}}}}). A complete list of GMICEs currently employed by ShakeMap is
provided in the {\hyperref[software_guide:sm35\string-software\string-guide]{\crossref{\DUrole{std,std-ref}{Software Guide}}}}.

We have implemented a convention for maps and regression plots that seismic stations
are represented with triangles and macroseismic data are depicted with circles (see \hyperref[tg_parameters:figure1-1]{Figure  \ref*{tg_parameters:figure1-1}}, for example). This convention is forward-looking: not all seismic networks were
currently following this convention at the time of this writing.
\begin{figure}[htbp]\begin{flushleft}
\capstart

\includegraphics{{Figure_1_1}.png}
\caption{Intensity ShakeMap from the 2014 M6.0 American Canyon (Napa Valley),
CA earthquake. Strong motion data (triangles) and intensity data (circles) are color-coded
according to their intensity value, either as observed (for macroseismic data) or as converted
by {\hyperref[references:wald1999b]{\crossref{\DUrole{std,std-ref}{Wald et al. (1999b)}}}} as shown in the legend. The north-south black line indicates the
fault location, which nucleated near the epicenter (red star). Note: Map Version Number reflects
separate offline processing for this Manual.}\label{tg_parameters:figure1-1}\label{tg_parameters:id1}\end{flushleft}\end{figure}

\hyperref[tg_parameters:figure-hawaii-interactive]{Figure  \ref*{tg_parameters:figure-hawaii-interactive}} shows a different representation of the
intensity map  on the newer, interactive maps on the USGS web site.
\begin{figure}[htbp]\begin{flushleft}
\capstart

\scalebox{0.900000}{\includegraphics{{Hawaii_interactive}.png}}
\caption{Intensity ShakeMap from the 2006 M6.7 Kahola Bay, HI earthquake.
Contours indicate intensities; strong motion data (triangles) and intensity data (circles) are color-
coded according to their intensity value, either as observed (for macroseismic data) or as
converted by {\hyperref[references:worden2012]{\crossref{\DUrole{std,std-ref}{Worden et al. (2012)}}}}. Inset on lower map shows
pop-up station information.}\label{tg_parameters:figure-hawaii-interactive}\label{tg_parameters:id2}\end{flushleft}\end{figure}


\section{Ground Motion and Intensity Predictions}
\label{tg_predictions::doc}\label{tg_predictions:sec-tg-predictions}\label{tg_predictions:ground-motion-and-intensity-predictions}
In areas distant from the control of seismic instrumentation or reported intensity, ground
motions must be estimated using the available earthquake source parameters and GMPEs
or Intensity Prediction Equations (IPEs). GMPEs are available for a wide range of magnitudes, source mechanisms,
and tectonic settings. IPEs are still comparatively uncommon, with only a handful of
published relations, focused on active tectonic and stable shield (cratonic) environments
(e.g., {\hyperref[references:atkinson\string-wald2007]{\crossref{\DUrole{std,std-ref}{Atkinson and Wald, 2007}}}}; {\hyperref[references:allen2012]{\crossref{\DUrole{std,std-ref}{Allen et al., 2012}}}}). To supplement the available IPEs, we
have developed a ``virtual IPE'' which is a combination of the operator's selected GMPE
and Ground Motion/Intensity Conversion Equation (GMICE), which work together to
present the same interface and behaviors as a direct IPE, while being available for a
wider range of regional and tectonic environments.

We describe the way ShakeMap employs ground-motion and intensity predictions in
{\hyperref[tg_processing:sec\string-shakemap\string-processing]{\crossref{\DUrole{std,std-ref}{ShakeMap Processing}}}}. An up-to-date list of the GMPEs and IPEs available for ShakeMap can be
found in the {\hyperref[software_guide:sm35\string-software\string-guide]{\crossref{\DUrole{std,std-ref}{Software Guide}}}}.


\section{ShakeMap Processing}
\label{tg_processing::doc}\label{tg_processing:sec-shakemap-processing}\label{tg_processing:shakemap-processing}
As discussed in the {\hyperref[users_guide:users\string-guide]{\crossref{\DUrole{std,std-ref}{User's Guide}}}}, ShakeMap produces a range of output products.
However, ShakeMap's primary outputs are grids of interpolated ground motions, from
which the other grids, contours, and maps are derived. Interpolated grids are produced for
PGA, PGV, macroseismic intensity (we will hereafter refer to macroseismic intensity
as ``MMI'' for Modified Mercalli Intensity, but other intensity measures are supported),
and (optionally) pseudo-spectral accelerations at 0.3, 1.0, and 3.0 sec. Attendant grids of
shaking-parameter uncertainty and Vs30, are also produced as separate
products or for later analyses of each intermediate processing step, if so desired.

The ShakeMap program responsible for producing the interpolated grids is called
``\emph{grind}”.  This section is a description of the way \emph{grind} works, and some of the
configuration parameters and command-line flags that control specific functionality. (For
a complete description of configuring and running \emph{grind}, see the {\hyperref[software_guide:sm35\string-software\string-guide]{\crossref{\DUrole{std,std-ref}{Software Guide}}}} and the
configuration file \emph{grind.conf}.)

Below is an outline of the ShakeMap processing workflow. \hyperref[tg_processing:figure-processing]{Figure  \ref*{tg_processing:figure-processing}} provides
a schematic of the key processing steps.
\begin{enumerate}
\item {} 
Data Preparation
\begin{enumerate}
\item {} 
Remove flagged stations

\item {} 
Convert intensities to peak ground motions (PGMs) and vice versa

\item {} 
Correct data to ``rock'' using Vs30-based amplification terms

\item {} 
Remove estimated basin response (optional)

\end{enumerate}

\item {} 
Correct earthquake bias with respect to the chosen GMPE
\begin{enumerate}
\item {} 
Remove the effect of directivity (optional)

\item {} 
Compute bias

\item {} 
Flag outliers

\item {} 
Repeat the previous two steps until no outliers are found

\item {} 
Create bias-adjusted GMPE estimates at each station location and for the entire output grid
(optionally, apply directivity)

\end{enumerate}

\item {} 
Interpolate ground motions to a uniform grid

\item {} 
Amplify ground motions
\begin{enumerate}
\item {} 
Basin amplifications (optional)

\item {} 
Vs30 site amplifications

\end{enumerate}

\end{enumerate}
\begin{figure}[htbp]\begin{flushleft}
\capstart

\includegraphics{{ProcessingFigure}.pdf}
\caption{A schematic of the basic ShakeMap ground motion interpolation scheme.}\label{tg_processing:figure-processing}\label{tg_processing:id2}\end{flushleft}\end{figure}

The sections that follow provide a more complete description of the processing steps outlined
above.


\subsection{Data preparation}
\label{tg_processing:data-preparation}
The first step in processing is the preparation of the parametric data. As discussed in the
{\hyperref[software_guide:sm35\string-software\string-guide]{\crossref{\DUrole{std,std-ref}{Software Guide}}}}, ground motion amplitudes are provided to ShakeMap in the form of
Extended Markup Language (XML) files. Note that we describe here the behavior of
\emph{grind} with respect to the input XML file(s). The programs that produce the input XML
(be it \emph{db2xml}, others, or the network operator's custom codes) will have their own rules
as to what is included in the input.

In our presentation here, the term “station” refers to a single seismographic station
denoted with a station ID (i.e., a code or number). In current practice, station IDs often
consist of a network identifier concatenated (using a “.”) with the station ID (e.g.,
“CI.JVC” or “CE.50281”).

Each station may have one or more “channels,” each of which is denoted by an ID code
(often called a “seedchan”). The last character of the ID is assumed to be the orientation
of the instrument (east-west, north-south, vertical). ShakeMap only
uses the peak horizontal component. Thus, ShakeMap does not consider amplitudes with a “Z” as the
final character, though it does carry the vertical amplitude
values through to the output station files. Note that some
stations in some networks are given orientations of “1”, “2”, and “3” (rather than the more
standard “N”, “E”, and “Z”), where any of the components may be vertical. Because of the
non-standardized nature of these component labels, ShakeMap does not attempt to
discern their orientation and assumes that they are all horizontal. This can lead to inaccuracies---it becomes the network operator's responsibility to ensure that the vertical channel
is either excluded or labeled with a “Z” before the data are presented to ShakeMap.
Similarly, many networks co-locate broadband instruments with strong-motion
instruments and produce PGMs for both. Again, it is the network operator's responsibility
to select the instrument that best represents the data for the PGMs in question. Aside from
the station flagging discussed below, ShakeMap makes no attempt to discern which of a
set of components is superior, it will simply use the largest value it finds (i.e., if
ShakeMap sees channels “HNE” and “HHE” for the same station, it will simply use the
larger of the two PGMS without regard to the possibility that one may be off-scale or
below-noise).

Currently, ShakeMap is location-code agnostic. Because the current SNCL (Station,
Network, Channel, Location) specification defines the location code as a pure identifier
(i.e., it should have no meaning), it is impossible to anticipate all the ways it may be used.
Therefore, if a network-station combo has multiple instruments at multiple locations, the
data provider should identify each location as a distinct station for ShakeMap XML input
purposes (by, for example, including the location code as part of the station identifier,
N.S.L---e.g., `CI.JVC.01'). If the network uses the location codes in another manner, it is
up to the operator to generate a station/component data structure that ShakeMap will
handle correctly.

Finally, each channel may produce one or more amplitudes (e.g., PGA, PGV, pseudo-
spectral acceleration). Note that these amplitudes should always be supplied by the
network as positive values, regardless of the direction of the peak motion. The amplitudes
for all stations and channels are collected and reported, but only the peak horizontal
amplitude of each ground-motion parameter is used by ShakeMap.

The foregoing is not intended to be a complete description of the requirements for the
input data. Please see the relevant section of the {\hyperref[software_guide:sm35\string-software\string-guide]{\crossref{\DUrole{std,std-ref}{Software Guide}}}}
for complete information.


\subsubsection{Flagged Stations}
\label{tg_processing:flagged-stations}
If the “flag” attribute of any amplitude in the input XML is non-null and non-zero, then
all components of that station are flagged as unusable. The reasoning here is that for a
given data stream, the typical network errors (telemetry glitch, incomplete data, off-scale
or below-noise data, etc.) will affect all of the parameters (as they are typically all
derived from the same data stream), and it is therefore impossible to determine the peak
horizontal component of any ground-motion parameter. This restriction is not without its
detractors, however, and we may revisit it at a future date.

MMI data are treated in much the same way; however, there is typically only one
``channel'' and one parameter (i.e., intensity).

ShakeMap presents flagged stations as
open, unfilled triangles on maps and on regression plots. In contrast, unflagged stations
are color coded by network or, optionally, by their amplitudes via their converted intensity
value, as shown in \hyperref[tg_processing:figure1-2]{Figure  \ref*{tg_processing:figure1-2}}. Flagged stations are also indicated as such within tables
produced for ShakeMap webpage consumption, e.g., the \emph{stations.xml} file.
\begin{figure}[htbp]\begin{flushleft}
\capstart

\includegraphics{{Figure_1_2}.png}
\caption{Peak acceleration ShakeMap from the Aug. 24, 2014, M6.0 American Canyon (Napa
Valley), California earthquake. Strong motion data (triangles) and intensity data (circles) are
color-coded according to their intensity value, either as observed (for macroseismic data) or as
converted by {\hyperref[references:wald1999b]{\crossref{\DUrole{std,std-ref}{Wald et al. (1999b)}}}} as shown in the legend.
The north-south black line
indicates the fault location, which nucleated near the epicenter (red star). Note: Map Version
Number reflects separate offline processing for this Manual.}\label{tg_processing:figure1-2}\label{tg_processing:id3}\end{flushleft}\end{figure}


\subsubsection{Converting MMI to PGM and PGM to MMI}
\label{tg_processing:converting-mmi-to-pgm-and-pgm-to-mmi}
Once the input data have been read and peak amplitudes assigned for each station (which
may be null if the data are flagged), intensities are derived from the peak
amplitudes and peak amplitudes are derived from the intensities using the GMICEs
configured (see the parameters `pgm2mi' and `mi2pgm' in \emph{grind.conf}). Small values of
observed intensities (MMI \textless{} III for PGA, and MMI \textless{} IV for other parameters) are not
converted to PGM for inclusion in the PGM maps. Our testing indicated that including
these low intensities introduced a significant source of error in the interpolation, likely
due to the very wide range (and overlap) of ground motions that can produce MMIs lower than III or IV.


\subsubsection{Site Corrections}
\label{tg_processing:site-corrections}
Near-surface conditions can have a substantial effect on ground motions. Ground motions
at soft-soil sites, for instance, will typically be amplified relative to sites on bedrock.
Because we wish to interpolate sparse data to a grid over which site characteristics may
vary greatly, we first remove the effects of near-surface amplification from our data,
perform the interpolation to a uniform grid at bedrock conditions, and then apply the site
amplifications to each point in the grid, based on each site's characteristics.

A commonly used proxy used to account for site effects (e.g., {\hyperref[references:borcherdt1994]{\crossref{\DUrole{std,std-ref}{Borcherdt, 1994}}}}) is Vs30,
the time-averaged shear wave velocity to 30 meters depth. Vs30 is also a fundamental
explanatory variable for modern GMPEs (e.g., {\hyperref[references:abrahamson2014]{\crossref{\DUrole{std,std-ref}{Abrahamson et al., 2014}}}}).  Since the use
of GMPEs for ground motion estimation is fundamental to ShakeMap, we follow this
convention and use Vs30-based amplification terms to account for site amplification.
In {\hyperref[future_directions:future\string-directions]{\crossref{\DUrole{std,std-ref}{Future Directions}}}}, we suggest alternative approaches that require additional
site information beyond Vs30.


\paragraph{Site Characterization Map}
\label{tg_processing:site-characterization-map}
Each region wishing to implement ShakeMap should have a Vs30 map that covers the
entire area they wish to map.  Using the Jan. 17, 1994, M6.7 Northridge, California
earthquake ShakeMap as an example (\hyperref[tg_processing:figure1-4]{Figure  \ref*{tg_processing:figure1-4}}), we present,
in \hyperref[tg_processing:figure1-5]{Figure  \ref*{tg_processing:figure1-5}}, the Vs30 map used.
Until 2015, the California site-condition map was based on geologic base maps as
introduced by {\hyperref[references:wills2000]{\crossref{\DUrole{std,std-ref}{Wills et al. (2000)}}}}, and modified by Howard Bundock and Linda Seekins
of the USGS at Menlo Park (H. Bundock, written comm., 2002). The Wills et
al. map extent is
that of the State boundary; however, ShakeMap requires a rectangular grid, so fixed
velocity regions were inserted to fill the grid areas representing the ocean and land
outside of California. Unique values were chosen to make it easy to replace those values
in the future. The southern boundary of the Wills et al. map coincides with the U.S.A./Mexico
border.  However, due to the abundant seismic activity in Imperial Valley and northern
Mexico, we have continued the trend of the Imperial Valley and Peninsular Ranges south
of the border by approximating the geology based on the topography; classification BC
was assigned to sites above 100m in elevation and CD was assigned to those below 100m.
This results in continuity of our site correction across the international border.
\begin{figure}[htbp]\begin{flushleft}
\capstart

\includegraphics{{Figure_1_4}.png}
\caption{PGA ShakeMap reprocessed with data from the 1994 M6.7 Northridge, CA earthquake with a finite
fault (red rectangle), strong motion data (triangles), and intensity data (circles). Stations and
macroseismic data are color-coded according to their intensity value, either as observed (for
macroseismic data) or as converted by {\hyperref[references:worden2012]{\crossref{\DUrole{std,std-ref}{Worden et al. (2012)}}}} and indicated by the scale
shown.}\label{tg_processing:figure1-4}\label{tg_processing:id4}\end{flushleft}\end{figure}
\begin{figure}[htbp]\begin{flushleft}
\capstart

\includegraphics{{vs30}.pdf}
\caption{Vs30 Map produced as a byproduct of ShakeMap for the 1994 M6.7 Northridge, CA earthquake. The
finite fault is shown as a red rectangle; strong motion data (triangles) and intensity data (circles)
are transparent to see site conditions. The legend indicates the range of color-coded Vs30 values
in m/sec.}\label{tg_processing:figure1-5}\label{tg_processing:id5}\end{flushleft}\end{figure}

Other ShakeMap operators have employed existing geotechnically- or geologically-based
Vs30 maps, or have developed their own Vs30 map for the area covered by their
ShakeMap. For regions lacking such maps, including most of globe, operators often
employ the approach of {\hyperref[references:wald2007]{\crossref{\DUrole{std,std-ref}{Wald and Allen (2007)}}}}, revised by {\hyperref[references:allen2009b]{\crossref{\DUrole{std,std-ref}{Allen and Wald, (2009b)}}}},
which provides estimates of Vs30 as a function of more readily available topographic
slope data. Wald and Allen's slope-based Vs30-mapping proxy is employed by the Global
ShakeMap (GSM) system.

Recent developments by {\hyperref[references:wald2011a]{\crossref{\DUrole{std,std-ref}{Wald et al. (2011d)}}}} and {\hyperref[references:thompson2012]{\crossref{\DUrole{std,std-ref}{Thompson et al. (2012}}}}; {\hyperref[references:thompson2014]{\crossref{\DUrole{std,std-ref}{2014}}}}) provide a
basis for refining Vs30 maps when Vs30 data constraints are abundant. Their method
employs not only geologic units and topographic slope, but also explicitly constrains map
values near Vs30 observations using kriging-with-a-trend to introduce the level of spatial
variations seen in the Vs30 data ({\hyperref[references:thompson2014]{\crossref{\DUrole{std,std-ref}{Thompson et al., 2014}}}}).  An example of Vs30 for
California using this approach is provided in \hyperref[tg_processing:figure1-6]{Figure  \ref*{tg_processing:figure1-6}}. Thompson et al. describe how
differences among Vs30 base maps translate into variations in site amplification in
ShakeMap.
\begin{figure}[htbp]\begin{flushleft}
\capstart

\includegraphics{{Figure_1_6}.png}
\caption{Revised California Vs30 Map ({\hyperref[references:thompson2014]{\crossref{\DUrole{std,std-ref}{Thompson et al., 2014}}}}). This map combines geology,
topographic slope, and constraints of map values near Vs30 observations using kriging-with-a-trend.
Inset shows Los Angeles region, with Los Angeles Basin indicating low Vs30 velocities.}\label{tg_processing:figure1-6}\label{tg_processing:id6}\end{flushleft}\end{figure}

{\hyperref[references:worden2015]{\crossref{\DUrole{std,std-ref}{Worden et al. (2015)}}}} further consolidate readily available Vs30 map grids used for
ShakeMaps at regional seismic networks of the ANSS with background, Thompson et
al.'s California Vs30 map, and the topographic-based Vs30 proxy to develop a
consistently scaled mosaic of \href{https://github.com/cbworden/earthquake-global\_vs30}{Vs30 maps for the globe}
with smooth transitions from tile to tile.  It is
anticipated that aggregated Vs30 data provided by
{\hyperref[references:yong2015]{\crossref{\DUrole{std,std-ref}{Yong et al. (2015)}}}} will facilitate further map development of other portions of the U.S.


\paragraph{Amplification Factors}
\label{tg_processing:amplification-factors}
ShakeMap provides two operator-selectable methods for determining the factors used to
amplify and de-amplify ground motions based on Vs30. The first is to apply the
frequency- and amplitude-dependent factors, such as those determined by
{\hyperref[references:borcherdt1994]{\crossref{\DUrole{std,std-ref}{Borcherdt (1994)}}}} or {\hyperref[references:seyhan2014]{\crossref{\DUrole{std,std-ref}{Seyhan and Stewart (2014)}}}}.
By default, amplification of PGA employs Borcherdt's short-period factors; PGV uses mid-period
factors; and PSA at 0.3, 1.0, and 3.0 sec uses the short-, mid-, and long-period factors,
respectively. The second method uses the site correction terms supplied with the user's
chosen GMPE (if such terms are supplied for that GMPE). The differences between these
choices and their behavior with respect to other user-configurable parameters are
discussed in the {\hyperref[software_guide:software\string-guide]{\crossref{\DUrole{std,std-ref}{Software \& Implementation Guide}}}}.


\paragraph{Correct Data to ``Rock''}
\label{tg_processing:correct-data-to-rock}
If, as is usually the case, the operator has opted to apply site amplification (via the \emph{-qtm}
option to \emph{grind}), the ground-motion observations are corrected (de-amplified) to ``rock”.
(The Vs30 of ``rock'' is specified with the parameter `smVs30default' in \emph{grind.conf}.) See
the section ``Site Corrections'' in the {\hyperref[software_guide:sm35\string-software\string-guide]{\crossref{\DUrole{std,std-ref}{Software Guide}}}} for
a complete discussion of the way
site amplifications are handled and the options for doing so.

Note that Borcherdt-style corrections do not handle PGV directly, so PGV is converted to
1.0 sec PSA (using {\hyperref[references:newmark1982]{\crossref{\DUrole{std,std-ref}{Newmark and Hall, 1982)}}}}, (de)amplified using the mid-period
Borcherdt terms, and then converted back to PGV. The Newmark and Hall conversion is
entirely linear and reversible, so while the conversion itself is an approximation, no bias
or uncertainty remains from the conversion following a ``round trip'' from site to bedrock
back to site.

Because there are no well-established site correction terms for MMI, when Borcherdt-
style corrections are specified, ShakeMap converts MMI to PGM, applies the
(de)amplification to PGM using the Borcherdt terms, then converts the PGMs back to
MMI.

\hyperref[tg_processing:figure1-7]{Figure  \ref*{tg_processing:figure1-7}} and \hyperref[tg_processing:figure1-8]{Figure  \ref*{tg_processing:figure1-8}} show shaking estimates (for PGA and
MMI, respectively) before site correction (upper left) and after
(upper right). Without site correction, ground
motion attenuation is uniform as a function of hypocentral distance (in the absence of a
finite fault model as in panels A through C) and fault distance (as in D); with site correction,
the correlation of amplitudes with the Vs30 map (and also topography) are more
apparent. This distinction is important: often complexity in ShakeMap's peak ground
motions and intensity patterns are driven by site terms, rather than variability due to
shaking observations.
\begin{figure}[htbp]\begin{flushleft}
\capstart

\includegraphics{{Figure_1_7}.png}
\caption{ShakeMap peak ground acceleration maps for the 1994 M6.7 Northridge, CA earthquake without
strong motion or intensity data. A) Hypocenter only, without site amplification; B) Hypocenter,
site amplification added; C) Hypocenter only, but with median distance correction added; and D)
Finite fault (red rectangle) added.}\label{tg_processing:figure1-7}\label{tg_processing:id7}\end{flushleft}\end{figure}
\begin{figure}[htbp]\begin{flushleft}
\capstart

\includegraphics{{Figure_1_8}.png}
\caption{ShakeMap intensity maps for the 1994 M6.7 Northridge, CA earthquake without strong
motion or intensity data. A) Hypocenter only, without site amplification; B) Hypocenter, site
amplification added; C) Hypocenter only, but with median distance correction added; and D)
Finite fault (red rectangle) added.}\label{tg_processing:figure1-8}\label{tg_processing:id8}\end{flushleft}\end{figure}

As the final step in correcting the observations to ``rock,'' if basin amplification is
specified (with the \emph{-basement} flag), the basin amplifications are removed from the PGM
data. Currently, basin amplifications are not applied to MMI.


\subsection{Event Bias}
\label{tg_processing:sec-event-bias}\label{tg_processing:event-bias}
ShakeMap uses ground motion prediction equations (GMPEs) and, optionally, intensity prediction
equations (IPEs) to supplement sparse data in its interpolation and estimation of ground
motions. If sufficient data are available, we compute an event bias that effectively
removes the inter-event uncertainty from the selected GMPE (IPE). This approach has
been shown to greatly improve the quality of the ShakeMap ground motion estimates (for
details, see {\hyperref[references:worden2012]{\crossref{\DUrole{std,std-ref}{Worden et al., 2012}}}}).

The bias-correction procedure is relatively straightforward: the magnitude of the
earthquake is adjusted so as to minimize the misfit between the observational data and
estimates at the observation points produced by the GMPE. If the user has chosen to use
the directivity correction (with the \emph{-directivity} flag), directivity is applied to the
estimates.

In computing the total misfit, primary observations (i.e., ground-motion observations from
seismic stations or MMI observations from \emph{Did You Feel It?} or field surveys) are weighted as
if they were GMPE
predictions, whereas converted observations (i.e., primary obsertations of one type converted to
another type, such as ground-motion observations converted to MMI) are down-weighted by treating
them as if they were GMPE predictions converted using the GMICE (i.e., primary observations are
given full weighting, whereas the converted observations are given a substantially lower
weight.) Once a bias has been obtained, we flag (as outliers) any data that exceed a user-
specified threshold (often three times the sigma of the GMPE at the obsertation point). The bias is then
recalculated and the flagging is repeated until no outliers are found. The flagged outliers
are then excluded from further processing (though the operator can choose to modify the
outlier criteria or impose their inclusion).

(There are a number of configuration parameters that affect the bias computation and the
flagging of outliers---see the {\hyperref[software_guide:sm35\string-software\string-guide]{\crossref{\DUrole{std,std-ref}{Software Guide}}}} and \emph{grind.conf}
for a complete discussion of these parameters.)

The bias-adjusted GMPE is then used to create estimates for the entire output grid.  If the user
has opted to include directivity effects, they are applied to these
estimates. See the {\hyperref[tg_processing:sec\string-interpolation]{\crossref{\DUrole{std,std-ref}{Interpolation}}}} section for the way the GMPE-based estimates are used.

The Northridge earthquake ShakeMap provides an excellent example of the effect
of bias correction. Overall, the ground motions for the Northridge earthquake exceed
average estimates of existing GMPEs---in other words, it has a significant positive
inter-event bias term (see \hyperref[tg_processing:nr-pga-regr]{Figure  \ref*{tg_processing:nr-pga-regr}} and \hyperref[tg_processing:nr-pgv-regr]{Figure  \ref*{tg_processing:nr-pgv-regr}},
which show PGA and PGV, respectively).
\begin{figure}[htbp]\begin{flushleft}
\capstart

\includegraphics{{northridge_pga_regr}.pdf}
\caption{Plot of Northridge earthquake PGAs as a function of distance. The triangles
show recorded ground motions; the red line shows the unbiased {\hyperref[references:ba2008]{\crossref{\DUrole{std,std-ref}{Boore and Atkinson
(2008)}}}} (BA08) GMPE; the dark green lines show BA08 following the bias
correction described in the text; the faint green lines show the bias-adjusted GMPE +/- three
standard deviations.}\label{tg_processing:nr-pga-regr}\label{tg_processing:id9}\end{flushleft}\end{figure}
\begin{figure}[htbp]\begin{flushleft}
\capstart

\includegraphics{{northridge_pgv_regr}.pdf}
\caption{Plot of Northridge earthquake PGVs as a function of distance. The triangles
show recorded ground motions; the red line shows the unbiased {\hyperref[references:ba2008]{\crossref{\DUrole{std,std-ref}{Boore and Atkinson
(2008)}}}} (BA08) GMPE; the dark green lines show BA08 following the bias
correction described in the text; the faint green lines show the bias-adjusted GMPE +/- three
standard deviations.}\label{tg_processing:nr-pgv-regr}\label{tg_processing:id10}\end{flushleft}\end{figure}

The ShakeMap bias correction accommodates this behavior once a
sufficient number of PGMs or intensity data are added (e.g., \hyperref[tg_processing:figure1-9]{Figure  \ref*{tg_processing:figure1-9}}
and \hyperref[tg_processing:figure1-10]{Figure  \ref*{tg_processing:figure1-10}} A and C,
show before and after bias correction, respectively). The addition of
the stations provides direct shaking constraints at those
locations; the bias correction additionally
affects the map wherever ground motion estimates dominate (i.e., away from the stations).
\begin{figure}[htbp]\begin{flushleft}
\capstart

\includegraphics{{Figure_1_9}.png}
\caption{PGA ShakeMaps for the Northridge earthquake, showing effects of adding
strong motion and intensity data. A) Finite fault only (red rectangle), no data; B) Strong motion
stations (triangles) only; C) Finite Fault and strong motion stations (triangles); D) Finite Fault
strong motion stations (triangles) and macroseismic data (circles). Notes: Stations and
macroseismic observations are color-coded to their equivalent intensity using {\hyperref[references:worden2012]{\crossref{\DUrole{std,std-ref}{Worden et al.
(2012)}}}}.}\label{tg_processing:figure1-9}\label{tg_processing:id11}\end{flushleft}\end{figure}
\begin{figure}[htbp]\begin{flushleft}
\capstart

\includegraphics{{Figure_1_10}.png}
\caption{Intensity ShakeMaps for the Northridge earthquake, showing effects of
adding strong motion and intensity data. A) Finite fault only (red rectangle), no data; B) Strong
motion stations (triangles) only; C) Finite Fault and strong motion stations (triangles); D) Finite
Fault strong motion stations (triangles) and macroseismic data (circles). Notes: (D) is the best
possible constrained representation for this earthquake. The finite fault model without data (A) is not
bias-corrected; for the Northridge earthquake, the inter-event biases are positive, indicating higher
than average ground shaking for M6.7; thus, the unbiased map (A) tends to under-predict shaking
shown in the data-rich, best-constrained map (D).}\label{tg_processing:figure1-10}\label{tg_processing:id12}\end{flushleft}\end{figure}


\subsection{Interpolation}
\label{tg_processing:sec-interpolation}\label{tg_processing:interpolation}
The interpolation procedure is described in detail in {\hyperref[references:worden2012]{\crossref{\DUrole{std,std-ref}{Worden et al. (2010)}}}}. Here we
summarize it only briefly.

To compute an estimate of ground motion at a given point in the latitude-longitude grid,
ShakeMap finds an uncertainty-weighted average of 1) direct observations of ground
motion or intensity, 2) direct observations of one type converted to another type (i.e.,
PGM converted to MMI, or vice versa), and 3) estimates from a GMPE or IPE. Note that
because the output grid points are some distance from the observations, we use a spatial
correlation function to obtain an uncertainty for each observation when evaluated at the
outpoint point. The total uncertainty at each point is a function of the uncertainty of the
direct observations obtained with the distance-to-observation uncertainty derived from
the spatial correlation function, and that of the GMPE or IPE.

In the case of direct ground-motion observations, the uncertainty at the observation site is
assumed to be zero, whereas at the ``site'' of a direct intensity observation, it is assumed to
have a non-zero uncertainty due to the spatially averaged nature of intensity assignments.
The uncertainty for estimates from
GMPEs (and IPEs) is the stated uncertainty given in the generative publication or
document. The GMPE uncertainty is often spatially constant, but this is not always the
case, especially with more recent GMPEs.

For converted observations, a third uncertainty is combined with zero-distance
uncertainty and the uncertainty due to spatial separation: the uncertainty associated with
the conversion itself (i.e., the uncertainty of the GMICE). This additional uncertainty
results in the converted observations being down-weighted in the average, relative to the
primary observations.

Because a point in the output may be closer to or farther from the source than a nearby
contributing observation, the observation is scaled by the ratio of the GMPE (or IPE)
estimates at the output point and the observation point:
\phantomsection\label{tg_processing:equation-equation1}\begin{align}\label{tg_processing-equation1}\begin{aligned}
\begin{split}Y_{obs,xy} = Y_{obs} \times \left(\frac{Y_{\text{GMPE},xy}}{Y_{\text{GMPE},obs}}\right),\end{split}\end{aligned}\end{align}
where \(Y_{obs}\) is the observation, and \(Y_{\text{GMPE},obs}\) and \(Y_{\text{GMPE},xy}\) are the ground-motion predictions
at the observation point and the point (\emph{x},*y*), respectively. This scaling is also applied to
the converted observations with the obvious substitutions. Note that the application of
this term also accounts for any geometric terms (such as directivity or source geometry)
that the ground-motion estimates may include.

The formula for the estimated ground motion at a point (\emph{x},*y*) is then given by:
\phantomsection\label{tg_processing:equation-equation2}\begin{align}\label{tg_processing-equation2}\begin{aligned}
\begin{split}\overline{Y_{xy}} = \frac{\displaystyle\frac{Y_{\text{GMPE},xy}}{\sigma_{\text{GMPE}}^2} + \displaystyle\sum_{i=1}^{n}\frac{Y_{obs,xy,i}}{\sigma_{obs,xy,i}^2} + \displaystyle\sum_{j=1}^{n}\frac{Y_{conv,xy,j}}{\sigma_{conv,xy,j}^2}}{\displaystyle\frac{1}{\sigma_{\text{GMPE}}^2} + \displaystyle\sum_{i=1}^{n}\frac{1}{\sigma_{obs,xy,i}^2} + \displaystyle\sum_{j=1}^{n}\frac{1}{\sigma_{conv,xy,j}^2}},\end{split}\end{aligned}\end{align}
where \(Y_{\text{GMPE},xy}\) and \(\sigma_{\text{GMPE}}^2\) are the amplitude and variance, respectively, at the point (\emph{x},*y*)
as given by the GMPE, \(Y_{obs,xy,i}\) are the observed amplitudes scaled to the point (\emph{x},*y*),
\(\sigma_{obs,xy,i}^2\) is the variance associated with observation \emph{i} at the point (\emph{x},*y*), \(Y_{conv,xy,j}\) are the
converted amplitudes scaled to the point (\emph{x},*y*), and \(\sigma_{conv,xy,j}^2\) is the variance associated
with converted observation \emph{j} at the point (\emph{x},*y*).

We can then compute the estimated IM (Equation \eqref{tg_processing-equation2}) for every point in the output grid. Note that the
reciprocal of the denominator of Equation \eqref{tg_processing-equation2} is the total variance at each point---a useful
byproduct of the interpolation process. Again, {\hyperref[references:worden2010]{\crossref{\DUrole{std,std-ref}{Worden et al. (2010)}}}} provides additional
details.


\subsection{Amplify Ground Motions}
\label{tg_processing:id1}\label{tg_processing:amplify-ground-motions}
At this point, ShakeMap has produced interpolated grids of ground motions (and
intensities) at a site class specified as ``rock.'' If the operator has specified the \emph{-basement}
option to \emph{grind} (and supplied the necessary depth-to-basement file), a
basin amplification function (currently {\hyperref[references:field2000]{\crossref{\DUrole{std,std-ref}{Field et al., 2000}}}}) is applied to the grids. Then, if the user has
specified \emph{-qtm}, site amplifications are applied to the grids, creating the final output.


\subsection{Differences in Handling MMI}
\label{tg_processing:differences-in-handling-mmi}
The processing of MMI is designed to be identical to the processing of PGM; however, a
few differences remain:
\begin{enumerate}
\item {} 
As of this writing, there are no spatial correlation functions available for MMI.
We are working on developing one, but it is not complete. We currently use the
spatial correlation function for PGA as a proxy for MMI, though this approach is
not optimal.

\item {} 
Because there are relatively few IPEs available at this time, we have introduced
the idea of a virtual IPE (VIPE). If the user does not specify an IPE in \emph{grind.conf},
\emph{grind} will use the configured GMPE in combination with the GMICE to simulate
the functionality of an IPE. In particular, the bias is computed as a magnitude
adjustment to the VIPE to produce the best fit to the intensity observations (and
converted observations) as described in {\hyperref[tg_processing:sec\string-event\string-bias]{\crossref{\DUrole{std,std-ref}{Event Bias}}}}; and the
uncertainty of the VIPE is the combined uncertainty of the GMPE and the
GMICE.

\item {} 
As mentioned above, intensity observations are given an inherent uncertainty
because of the nature of their assignment. Our research has shown that this
uncertainty amounts to a few tenths of an intensity unit, but it varies with the
number of responses within the averaged area. Research in this area is incomplete,
and intensity data can contain both ``Did You Feel It?'' data and traditionally
assigned intensities. Because of these considerations, we currently use a
conservative value of 0.5 intensity units for the inherent uncertainty.

\item {} 
The directivity function we use ({\hyperref[references:rowshandel2010]{\crossref{\DUrole{std,std-ref}{Rowshandel, 2010}}}}) does not have terms for
intensity. This is not a problem when using the VIPE, since we can apply the
directivity function to the output of the encapsulated GMPE before converting to
intensity. But when a true IPE is used there is currently no explicit way to apply our
directivity function.  In these cases, we use the VIPE to
compute two intensity grids: one with and one without directivity applied. We
then subtract the former from the latter to produce a grid of directivity adjustment
factors. That grid is then added to the output of the
IPE. We use the same procedure when creating estimates at observation locations
for computing the bias.

\item {} 
As mentioned above, we currently have no function for applying basin
amplification to the intensity data. We hope to remedy this shortly with a solution
similar to item 4, above, where we apply the basin effects through a
VIPE. In practice, the main areas where basin depth models are
available are also those within which station density is great
(e.g., Los Angeles and San Francisco, California).

\end{enumerate}


\subsection{Fault Considerations}
\label{tg_processing:fault-considerations}
Small-to-moderate earthquakes can be effectively characterized as a point source, with
distances being calculated from the hypocenter (or epicenter, depending on the GMPE).
But accurate prediction of ground motions from larger earthquakes requires knowledge of
the fault geometry. This is because ground motions attenuate with
distance from the source (i.e. fault), but for a spatially extended source, that distance can be quite different
from the distance to the hypocenter. Most GMPEs are developed using earthquakes with
well-constrained fault geometry, and therefore are not suitable for prediction of ground
motions from large earthquakes when only a point source is available. As discussed in the
{\hyperref[tg_processing:sec\string-median\string-distance]{\crossref{\DUrole{std,std-ref}{next section}}}}, we handle this common situation by using terms that modify the
distance calculation to accommodate the unknown fault geometry. We also allow the
operator to specify a finite fault, as discussed in sections {\hyperref[tg_processing:sec\string-fault\string-dimensions]{\crossref{\DUrole{std,std-ref}{Fault Dimensions}}}}
and {\hyperref[tg_processing:sec\string-directivity]{\crossref{\DUrole{std,std-ref}{Directivity}}}}.


\subsubsection{Median Distance and Finite Faults}
\label{tg_processing:median-distance-and-finite-faults}\label{tg_processing:sec-median-distance}
As discussed in the {\hyperref[software_guide:sm35\string-software\string-guide]{\crossref{\DUrole{std,std-ref}{Software Guide}}}}, the user may specify a
finite fault to guide the
estimates of the GMPE, but often a fault model is not available for some time following
an earthquake. For larger events, this becomes problematic because the
distance-to-source
term ShakeMap provides to the GMPE in order to predict ground motions comes
from the hypocenter (or epicenter, depending on the GMPE) rather than the actual rupture
plane (or its surface projection), and for a large fault, these distances can be quite
different. For a non-point source, in fact, the hypocentral distance
can equal the closest distance, but it can also be significantly greater than the
closest distance.

ShakeMap addresses this issue by introducing the concept of median distance. Following
a study by {\hyperref[references:epri2003]{\crossref{\DUrole{std,std-ref}{EPRI (2003)}}}}, we assume that an unknown fault of appropriate size could have
any orientation, and we use EPRI's equations to compute the distance that produces
the median ground motions of
all the possible fault orientations that pass through the hypocenter. (Thus, the term
``median distance'' is a bit of a misnomer; it is more literally ``distance of median ground
motion.'') Thus, for each point for which we want ground motion estimates, we compute
this distance and use it as input to the GMPE. We also adjust the uncertainty of the
estimate to account for the larger uncertainty associated with this situation. This feature
automatically applies for earthquake magnitudes \textgreater{}= 5, but may be disabled with the \emph{grind} flag
\emph{-nomedian}.

Ideally, GMPE developers would always regress not only for fault distance, but also for
hypocentral distance. If this were done routinely, we would be able to initially use
the hypocentral-distance regression coefficients and switch to fault-distance terms as the
fault geometry was established. The median-distance approximation described above
could then be discarded.

{\hyperref[references:bommer2012]{\crossref{\DUrole{std,std-ref}{Bommer and Akkar (2012)}}}} have made the case for deriving both sets of coefficients:
``The most simple, consistent, efficient and elegant solution to this problem is for all
ground-motion prediction equations to be derived and presented in pairs of models, one
using the analysts' preferred extended source metric ... ---and another using a point-
source metric, for which our preference would be hypocentral distance,
Rhyp''. Indeed, {\hyperref[references:akkar2014]{\crossref{\DUrole{std,std-ref}{Akkar et al. (2014)}}}} provide such multiple coefficients
for their GMPEs for the Middle East and Europe. However, despite its utility, this
strategy has not been widely adopted among the requirements for modern GMPEs (e.g.,
{\hyperref[references:powers2008]{\crossref{\DUrole{std,std-ref}{Powers et al., 2008}}}}; {\hyperref[references:abrahamson2008]{\crossref{\DUrole{std,std-ref}{Abrahamson et al., 2008}}}};
{\hyperref[references:abrahamson2014]{\crossref{\DUrole{std,std-ref}{2014}}}}).

The hypocentral- or median-distance correction is not a trivial consideration. Note that for
Northridge, even when the fault is unknown and there are no data, the median-distance
correction (\hyperref[tg_processing:figure1-7]{Figure  \ref*{tg_processing:figure1-7}} and \hyperref[tg_processing:figure1-8]{Figure  \ref*{tg_processing:figure1-8}}, panels B and C)
brings the shaking estimates more in line
with those constrained by knowledge of the fault. As mentioned earlier, the shaking for
this event exhibits a positive inter-event bias term, so even with the fault location
constrained, estimates still tend to under-predict the actual recordings, on average.

While the effect of this correction for the Northridge earthquake example is noticeable,
for events with larger magnitudes, and thus larger rupture areas, the median-distance
correction is crucial before
constraints on rupture geometry are available (from finite-fault modeling, aftershock
distribution, observations of surface slip, etc.) For example, in the case of the 1994
Northridge earthquake, the dimensions of the rupture are constrained from analyses of
the earthquake source (e.g., {\hyperref[references:wald1996]{\crossref{\DUrole{std,std-ref}{Wald et al., 1996}}}}).


\subsubsection{Fault Dimensions}
\label{tg_processing:fault-dimensions}\label{tg_processing:sec-fault-dimensions}
The {\hyperref[software_guide:sm35\string-software\string-guide]{\crossref{\DUrole{std,std-ref}{Software Guide}}}} describes the format for specifying a
fault. Essentially, ShakeMap
faults are one or more (connected or disconnected) planar quadrilaterals. The fault
geometry is used by ShakeMap to compute fault-to-site distances for the GMPE, IPE, and
GMICE, as well as to visualize the fault geometry in map view (for example, see red-line
rectangles in \hyperref[tg_processing:figure1-7]{Figure  \ref*{tg_processing:figure1-7}} and \hyperref[tg_processing:figure1-8]{Figure  \ref*{tg_processing:figure1-8}}). Examples of
fault-based distances include the distance to the surface projection of the
fault (for the so-called Joyner-Boore, or JB, distance), and the distance to the rupture plane.

While a finite fault is important for estimating the shaking from larger earthquakes, it is
typically not necessary to develop an extremely precise fault model, or to know the
rupture history that specifies the rupture propagation and slip distribution.
One or two fault planes usually suffice, except for very large or complex
surface-rupturing faults. In the immediate aftermath of a large earthquake, a first-order
fault model based on tectonic environment, known faults, aftershock distribution, and
empirical estimates based on the magnitude (e.g., {\hyperref[references:wells1994]{\crossref{\DUrole{std,std-ref}{Wells and Coppersmith, 1994}}}}) is often
sufficient to greatly improve the ShakeMap estimates in poorly instrumented areas. In
many cases, this amounts to an ``educated guess”, and seismological expertise and
intuition are extremely helpful. Later refinements to the faulting geometry may or may
not fundamentally change the shaking pattern, depending on the density of near-source
observations. As we show in a later section, dense observations greatly diminish the
influence of the estimated ground motion at each grid point, obviating the need for precise
fault geometries in such estimates.


\subsubsection{Directivity}
\label{tg_processing:sec-directivity}\label{tg_processing:directivity}
Another way in which a finite fault may affect the estimated ground motions is through
directivity. Where a finite fault has been defined in ShakeMap, one may choose to apply
a correction for rupture directivity. We use the approach developed
by {\hyperref[references:rowshandel2010]{\crossref{\DUrole{std,std-ref}{Rowshandel (2010)}}}}
for the NGA GMPEs (note: caution should be exercised when applying this directivity
function to non-NGA GMPEs; in addition, other directivity models give significantly different
results, which is an indication that there is a high degree of uncertainty in these models).
For the purposes of this calculation, we assume a
constant rupture over the fault surface. While the directivity effect is secondary to fault
geometry, it can make a not-insignificant difference in the near-source ground motions
up-rupture or along-strike from the hypocenter.

An example of the effect of the {\hyperref[references:rowshandel2010]{\crossref{\DUrole{std,std-ref}{Rowshandel (2010)}}}} directivity term is shown clearly in
\hyperref[tg_processing:figure1-13]{Figure  \ref*{tg_processing:figure1-13}} for a hypothetical strike-slip faulting scenario along the Hayward Fault in the East Bay
area of San Francisco. Unilateral rupture southeastward results in stronger shaking,
particularly along the southern edge of the rupture. The frequency dependence of the
directivity terms provided by {\hyperref[references:rowshandel2010]{\crossref{\DUrole{std,std-ref}{Rowshandel (2010)}}}} can be examined in detail by viewing
the intermediate grids produced and stored in the ShakeMap output
directory. In general, longer-period IMs (PGV, PSA1.0 and PSA3.0,
and MMI) are more strongly affected by the directivity function
employed.
\begin{figure}[htbp]\begin{flushleft}
\capstart

\includegraphics{{Figure_1_13}.png}
\caption{ShakeMap scenario intensity (top) and PGV (bottom) maps for the hypothetical M7.05
Hayward Fault, CA, earthquake: A) Intensity, No directivity; B) Intensity, Directivity added; C)
PGV, No Directivity; and D) PGV, Directivity added.}\label{tg_processing:figure1-13}\label{tg_processing:id13}\end{flushleft}\end{figure}


\subsection{Additional Considerations}
\label{tg_processing:additional-considerations}
There are a great number of details and options when running \emph{grind}. Operators should
familiarize themselves with \emph{grind} ’s behavior by reading the {\hyperref[software_guide:sm35\string-software\string-guide]{\crossref{\DUrole{std,std-ref}{Software Guide}}}},
the configuration file (\emph{grind.conf}), and the program's
self-documentation (run “\emph{grind -help}”).
Below are a few other characteristics of \emph{grind} that the operator should be familiar with.


\subsubsection{User-supplied Estimate Grids}
\label{tg_processing:user-supplied-estimate-grids}
Much of the discussion above was centered on the use of GMPEs (and IPEs) and getting
the best set of estimates from them (through bias, basin corrections, finite faults, and directivity).
But the users may also supply their own grids of estimates for any or all of the ground motion
parameters. ShakeMap is indifferent as to the way these estimates are generated, as long
as they appear in a GMT grid in the event's input directory, they will be used in place of
the GMPE's estimates. (See the {\hyperref[software_guide:sm35\string-software\string-guide]{\crossref{\DUrole{std,std-ref}{Software Guide}}}} for the
specifications of these input
files.) If available, the user should also supply grids of uncertainties for the corresponding
parameters---if not, ShakeMap will use the uncertainties produced by
the GMPE.

User-supplied input motions allow the user to employ more sophisticated numerical ground-motion
modeling techniques, ones that may include, for example, fault-slip distribution and 3D propagation
effects not achievable using empirical GMPEs. The PGM
output grid of such calculations can be rendered with ShakeMap, allowing users to
visualize and employ familiar ShakeMap products. For instance, see the
ShakeCast scenario described in {\hyperref[shakemap_applications:sec\string-shakemap\string-applications]{\crossref{\DUrole{std,std-ref}{Applications of ShakeMap}}}}.


\subsubsection{Uncertainty}
\label{tg_processing:uncertainty}
As mentioned above, some of the products of \emph{grind} are grids of uncertainty for each
parameter. This uncertainty is the result of a weighted average combination of the
uncertainties of the various inputs (observations, converted observations, and estimates)
at each point in the output. These gridded uncertainties are provided in the file
\emph{uncertainty.xml} (see {\hyperref[products:sec\string-interpolated\string-grid\string-file]{\crossref{\DUrole{std,std-ref}{Interpolated Ground Motion Grids}}}} for a description of the
file format).

Because we also know the GMPE uncertainty over the grid, we can compute the ratio of
the total ShakeMap uncertainty to the GMPE uncertainty. For the purposes of computing
this uncertainty ratio, we use PGA as the reference IM.

Continuing with the Northridge earthquake ShakeMap example, \hyperref[tg_processing:figure1-12]{Figure  \ref*{tg_processing:figure1-12}} presents the
uncertainty maps for a variety of constraints.
\begin{figure}[htbp]\begin{flushleft}
\capstart

\includegraphics{{Figure_1_12}.png}
\caption{ShakeMap uncertainty maps for the Northridge earthquake showing
effect of adding a finite fault and strong motion data. Color-coded legend shows uncertainty ratio,
where `1.0' indicates 1.0 times the GMPE's sigma. A) Hypocenter (black star) only; B) Finite fault
(red rectangle) added but no data (mean uncertainty is 1.00 at all locations since the site-to-source
distance is constrained); C) Hypocenter and strong motion stations (triangles) only (adding stations
reduces overall uncertainty); and D) Finite fault and strong motion stations. Note: Average
uncertainty is computed by averaging uncertainty at grids that lie within the MMI = VI contour
(bold contour line), so panel (D) is marginally higher than (C) despite added constraint (fault model).
For more details see {\hyperref[references:wald2008]{\crossref{\DUrole{std,std-ref}{Wald et al. (2008)}}}} and {\hyperref[references:worden2010]{\crossref{\DUrole{std,std-ref}{Worden et al. (2010)}}}}.}\label{tg_processing:figure1-12}\label{tg_processing:id14}\end{flushleft}\end{figure}

For a purely predictive map (of small magnitude), the uncertainty ratio will be 1.0
everywhere. For larger magnitude events, when a finite fault is not available, the
ShakeMap uncertainty is greater than the nominal GMPE uncertainty (as discussed above
in the section {\hyperref[tg_processing:sec\string-median\string-distance]{\crossref{\DUrole{std,std-ref}{Median Distance and Finite Faults}}}}), leading to a ratio greater than 1.0 in
the near-fault areas and diminishing with distance. When a finite fault is available, the
ratio returns to 1.0. In areas where data are available, the ShakeMap uncertainty is less
than that of the GMPE (see the section ``Interpolation,'' above), resulting in a ratio less
than 1.0. A grid of the uncertainty ratio (and the PGA uncertainty) is provided in the
output file \emph{grid.xml} (see {\hyperref[products:sec\string-interpolated\string-grid\string-file]{\crossref{\DUrole{std,std-ref}{Interpolated Ground Motion Grids}}}} for a description of
this file). The uncertainty ratio is the basis for the uncertainty maps and the grading
system described in the {\hyperref[users_guide:users\string-guide]{\crossref{\DUrole{std,std-ref}{User's Guide}}}}.


\section{Representing Macroseismic Intensity on Maps}
\label{tg_intensity::doc}\label{tg_intensity:representing-macroseismic-intensity-on-maps}\label{tg_intensity:sec-tg-intensity}
{\hyperref[references:wald1999b]{\crossref{\DUrole{std,std-ref}{Wald et al. (1999b)}}}} relates recorded ground motions to Modified Mercalli Intensities in
California. While not the first work of its type, Wald et al. had the advantage of using
several earthquakes that were both very well surveyed for MMI, and also well
instrumented for recorded ground motions. By relating the ground motions to MMI, Wald
et al. made possible the rapid calculation of expected intensities from recorded ground
motions. These “instrumental intensities” could be interpolated over an area and
represented on a map.

As part of the original implementation of ShakeMap, {\hyperref[references:wald1999a]{\crossref{\DUrole{std,std-ref}{Wald, et al. (1999a)}}}} developed a
color scale to represent expected intensities over the mapped area. This scale gives users
an intuitive, easy-to-understand depiction of the ground shaking for a given earthquake.
By mapping intensity to color, we also make the hardest-hit areas stand out for
emergency responders and members of the media. Along with the color scale, we
developed simplified two-word descriptions of the felt intensity as well as the likely
damage. These abridged descriptions are not meant to replace
more comprehensive descriptions provided in the MMI (e.g.,
{\hyperref[references:dewey2000]{\crossref{\DUrole{std,std-ref}{Dewey et al., 2000}}}}; {\hyperref[references:dewey1995]{\crossref{\DUrole{std,std-ref}{Dewey et al., 1995}}}}),
or EMS-98 ({\hyperref[references:grunthal1998]{\crossref{\DUrole{std,std-ref}{Grunthal et al., 1998}}}}) scales; however, they offer convenient description for
our purposes.

By relating recorded peak ground motions to Modified Mercalli Intensities, we can
generate instrumental intensities within a few minutes of an earthquake. In the current
ShakeMap system, these instrumental intensities can be combined with direct measures
of intensity (from “Did You Feel It?”, for example) and interpolated across the affected area. With the
color-coding and two-word text descriptors, we can adequately describe the associated
perceived shaking and potential damage consistent with both human response and
damage assessments of past earthquakes to characterize the shaking from just-occurred
earthquakes.


\subsection{Color Palette for the ShakeMap Instrumental Intensity Scale}
\label{tg_intensity:color-palette-for-the-shakemap-instrumental-intensity-scale}
The color coding for the Instrumental Intensity map uses a standard rainbow palette (see Table
1.1).  The ``cool'' to ``hot'' color scheme is familiar to most and is readily recognizable,
as it is used as a standard in many fields (for example, see USA Today's daily temperature
maps of the U.S.).  Note that we do not believe intensity II and III can be consistently
distinguished from ground-motions alone, so they are grouped together, see \hyperref[tg_intensity:figure1-16]{Figure  \ref*{tg_intensity:figure1-16}}. In
addition, we saturate intensity X+ with dark red; observed ground motions alone are not
sufficient to warrant any higher intensities, given that the available empirical relationships
do not have any values of intensity greater than IX. In recent years, the USGS has limited
observed MMIs to IX, reserving intensity X for possible future
observations (see {\hyperref[references:dewey1995]{\crossref{\DUrole{std,std-ref}{Dewey et al., 1995}}}}, for more details); the USGS no
longer assigns intensity XI and XII. We note that there were only
only two intensity-IX assignments for the 1994 Northridge earthquake ({\hyperref[references:dewey1995]{\crossref{\DUrole{std,std-ref}{Dewey et al., 1995}}}}),
and only two or three proper intensity-IX assignments for the
1989 Loma Prieta earthquake (J. Dewey, 2015, personal communication).

\begin{tabulary}{\linewidth}{|L|L|L|L|L|L|L|L|}
\hline
\textsf{\relax 
Intensity
} & \textsf{\relax 
Red
} & \textsf{\relax 
Green
} & \textsf{\relax 
Blue
} & \textsf{\relax 
Intensity
} & \textsf{\relax 
Red
} & \textsf{\relax 
Green
} & \textsf{\relax 
Blue
}\\
\hline
0
 & 
255
 & 
255
 & 
255
 & 
1
 & 
255
 & 
255
 & 
255
\\
\hline
1
 & 
255
 & 
255
 & 
255
 & 
2
 & 
191
 & 
204
 & 
255
\\
\hline
2
 & 
191
 & 
204
 & 
255
 & 
3
 & 
160
 & 
230
 & 
255
\\
\hline
3
 & 
160
 & 
230
 & 
255
 & 
4
 & 
128
 & 
255
 & 
255
\\
\hline
4
 & 
128
 & 
255
 & 
255
 & 
5
 & 
122
 & 
255
 & 
147
\\
\hline
5
 & 
122
 & 
255
 & 
147
 & 
6
 & 
255
 & 
255
 & 
0
\\
\hline
6
 & 
255
 & 
255
 & 
0
 & 
7
 & 
255
 & 
200
 & 
0
\\
\hline
7
 & 
255
 & 
200
 & 
0
 & 
8
 & 
255
 & 
145
 & 
0
\\
\hline
8
 & 
255
 & 
145
 & 
0
 & 
9
 & 
255
 & 
0
 & 
0
\\
\hline
9
 & 
255
 & 
0
 & 
0
 & 
10
 & 
200
 & 
0
 & 
0
\\
\hline
10
 & 
200
 & 
0
 & 
0
 & 
13
 & 
128
 & 
0
 & 
0
\\
\hline\end{tabulary}


Table 1.1  Color Mapping Table for Instrumental Intensity. This is a portion of the
Generic Mapping Tools (GMT) ``cpt'' file. Color values for intermediate intensities
are linearly interpolated from the Red, Green, and Blue (RGB) values in columns 2-4
to columns 6-8.

\begin{DUlineblock}{0em}
\item[] 
\item[] 
\end{DUlineblock}

\includegraphics{{Figure_1_16_top}.png}
\begin{figure}[htbp]\begin{flushleft}
\capstart

\includegraphics{{Figure_1_16_bottom}.png}
\caption{Top: ShakeMap Instrumental Intensity Scale Legend: Color palette, two-word
text descriptors, and ranges of peak motions for Instrumental Intensities. Note that the peak
motions are applicable to {\hyperref[references:worden2012]{\crossref{\DUrole{std,std-ref}{Worden et al. (2012)}}}}; other Ground Motion/Intensity
Conversion Equations use the same color scale, but their ranges of peak motions will differ. Bottom:
Legend below PGV ShakeMap. The legend for below each ShakeMap is now map
(parameter-) and GMICE-specific as labeled. Color-coding of stations corresponds to their
intensity the GMICE (ground motion/intensity) relationship.}\label{tg_intensity:figure1-16}\label{tg_intensity:id1}\end{flushleft}\end{figure}

We drape the color-coded Instrumental Intensity values over the topography to maximize
the information available in terms of both geographic location and likely site conditions.
Topography serves as a simple yet effective proxy for examining site and basin
amplification, but we recognize that many users do not necessarily
benefit intuitively from having topography as a basemap.


\subsection{ShakeMap Instrumental Intensity Scale Text Descriptions}
\label{tg_intensity:shakemap-instrumental-intensity-scale-text-descriptions}
The estimated intensity map is usually wholly or partially derived from ground motions
recorded by seismic instruments, and represents intensities that are likely to have been
associated with the recorded ground motions. However, unlike conventional intensities,
the instrumental intensities are not based on observations of the earthquake’s effects on
people or structures. The terms ``perceived shaking'' and ``potential damage'' in the
ShakeMap legend are chosen for this reason; these intensities were not observed, but they
are consistent on average with intensities at these ranges of ground motions recorded in a
number of past earthquakes (see, for example, {\hyperref[references:wald1999b]{\crossref{\DUrole{std,std-ref}{Wald et al., 1999b}}}}; {\hyperref[references:worden2012]{\crossref{\DUrole{std,std-ref}{Worden et al., 2012}}}}).
Two-word descriptions of both shaking and damage levels are provided to summarize the
effects in an area; they were derived with careful consideration of the existing
descriptions in the Modified Mercalli definitions (L. Dengler and J. Dewey, written
communication, 1998, 2003).

The ShakeMap qualitative descriptions of shaking are intended to be consistent with the way
people perceive the shaking in earthquakes.  The descriptions for intensities up to VII are
constrained by the work of {\hyperref[references:dengler1998]{\crossref{\DUrole{std,std-ref}{Dengler and Dewey (1998)}}}}, in which they compared results of
telephone surveys with USGS MMI intensities for the 1994 Northridge earthquake. The
ShakeMap descriptions up to intensity VII may be viewed as a rendering of Dengler and
Dewey's Figure 7a.

The instrumental intensity map for the Northridge earthquake shares most of the notable
features of the Modified Mercalli map prepared by the USGS ({\hyperref[references:dewey1995]{\crossref{\DUrole{std,std-ref}{Dewey et al., 1995}}}}),
including the relatively high intensities near Santa Monica and southeast of the epicenter
near Sherman Oaks.  However, in general, the area of intensity IX on the instrumentally
derived intensity map is slightly larger than on the USGS Modified Mercalli intensity
map. This reflects the fact that although much of the Santa Susanna mountains, north and
northwest of the epicenter, were very strongly shaken, the region is also sparsely
populated, hence, observed intensities could not be determined there.  This is a fundamental
difference between observed and instrumentally derived intensities: instrumental
intensities will show high levels of strong shaking independent of the exposure of
populations and buildings, while observed intensities only represent intensities where there are
structures to damage and people to experience the earthquake.

The ShakeMap descriptions of felt shaking begin to lose meaning above intensity VII or VIII. In
the {\hyperref[references:dengler1998]{\crossref{\DUrole{std,std-ref}{Dengler and Dewey (1998)}}}} study, peoples' perception of shaking began to saturate in
the VII-VIII range, with more than half the people at VII-VIII and above
reporting the shaking as ``violent'' (on a scale from ``weak'' to ``violent”).  In the ShakeMap
descriptions, we intensified the descriptions of shaking with increases of intensity above
VII, because the evidence from instrumental data is that the shaking is stronger.  But we
know of no solid evidence that one could discriminate intensities higher than VII on the
basis of different individuals' descriptions of perceived shaking alone.

ShakeMap is not unique in describing intensity VI as corresponding to strong shaking. In
the 7-point Japanese macroseismic scale, for which intensity 4 is equivalent to MMI VI,
intensity 4 is described as ``strong.'' In the European Macroseismic Scale ({\hyperref[references:grunthal1998]{\crossref{\DUrole{std,std-ref}{Grunthel et al.,
1998}}}}) (EMS-98), which is compatible with MMI ({\hyperref[references:musson2010]{\crossref{\DUrole{std,std-ref}{Musson et al., 2010}}}}), the bullet description of intensity
V is  ``strong.''  Higher EMS-98 intensities are given bullet descriptions in terms of the
damage they produce, rather than the strength of perceived shaking.


\subsection{ShakeMap Intensity Scale and Peak Ground Motions}
\label{tg_intensity:shakemap-intensity-scale-and-peak-ground-motions}
The ShakeMap Instrumental Intensity Scale Legend provides the PGA and PGV
associated with the central value in each
intensity box (see \hyperref[tg_intensity:figure1-16]{Figure  \ref*{tg_intensity:figure1-16}}). For all current GMICEs, the ground motion scale is
logarithmic, with an increase of one intensity unit resulting from approximately a
doubling of peak ground motion. Nevertheless, each GMICE has its own mapping of
ground motion to intensity, and thus the values shown in the scale legend can vary,
depending on the GMICE chosen for the map in question. To avoid confusion, the
legends now have a citation in the lower left specifying which GMICE was used to
produce the map and scale. Note, however, that while the mapping of ground motion to
intensity varies, the mapping of color to intensity remains the same for all maps.

We note that the ShakeMap legends (e.g., \hyperref[tg_intensity:figure1-16]{Figure  \ref*{tg_intensity:figure1-16}}) have evolved slightly from the
earlier version of ShakeMap and the 2005 ShakeMap Manual. The PGMs tabulated are
no longer provided by (previously redundant) PGM ranges, but rather by the median
motions associated with the intensity on the scale.


\subsection{Color Coding Stations by Intensity}
\label{tg_intensity:color-coding-stations-by-intensity}
Traditionally, stations on the PGM ShakeMaps were color-coded to the seismic network
that provided them. More recent versions of ShakeMap, however, allow the operator to
color the stations with the intensity they produced, with each PGM parameter (e.g., PGA,
PGV, PSA03) using its own intensity correlation. \hyperref[tg_intensity:figure1-11]{Figure  \ref*{tg_intensity:figure1-11}} (and
many of the other figures throughout this guide) illustrates the color coding of stations
by their intensity values for several parameters. The operator can elect this option by
calling the program \emph{mapping} with the flag \emph{-pgminten}.
\begin{figure}[htbp]
\centering
\capstart

\scalebox{0.950000}{\includegraphics{{Figure_1_11}.png}}
\caption{ShakeMap for the 1994 M6.7 Northridge, CA earthquake with a finite fault (red
rectangle), strong motion data (triangles) and intensity data (circles). Stations and macroseismic
data are color coded according to their intensity value, either as observed (for macroseismic data)
or as converted by {\hyperref[references:worden2011]{\crossref{\DUrole{std,std-ref}{Worden et al. (2011)}}}} and indicated by the scales shown. Note:
Macroseismic data do not change colors from map to map, but seismic stations do, since the
estimated intensity conversion depends on which parameter is used.}\label{tg_intensity:figure1-11}\label{tg_intensity:id2}\end{figure}


\section{Discussion of Chosen Map Parameters (Intensity Measures)}
\label{tg_choice_of_parameters::doc}\label{tg_choice_of_parameters:tg-sec-choice-of-parameters}\label{tg_choice_of_parameters:discussion-of-chosen-map-parameters-intensity-measures}

\subsection{Use of Peak Values Rather than Mean}
\label{tg_choice_of_parameters:use-of-peak-values-rather-than-mean}\label{tg_choice_of_parameters:peak-values-vs-mean}
With ShakeMap, we chose to represent peak ground motions as recorded. We depict the
larger of the two horizontal components, rather than as either a vector sum, or as a
geometric mean value. The initial choice of peak values was necessitated by the fact that
roughly two-thirds of the TriNet (now the Southern California portion of CISN) strong-
motion data (the California Geological Survey, or CGS, data)
are delivered as peak values for individual components of
motion, that is, as parametric data rather than waveforms. This left two options: providing peak
values or median of the peak values---determining vector sums of the two horizontal
components was not an option, because the peak values on each component do not
necessarily occur at the same time.  A useful strategy going forward may be to employ
the 50th percentile of the response spectra over all non-redundant rotation angles
(RotD50; {\hyperref[references:boore2010]{\crossref{\DUrole{std,std-ref}{Boore, 2010}}}}), which is becoming a standard
for ``next-generation''
GMPEs ({\hyperref[references:abrahamson2014]{\crossref{\DUrole{std,std-ref}{Abrahamson et al., 2014}}}}), or on another
agreed-upon vector-component
calculation. (See {\hyperref[future_directions:future\string-directions]{\crossref{\DUrole{std,std-ref}{Future Directions}}}}). However, such changes would
require adoption of these
calculations on the part of each contributing seismic network, and would thus require
consensus (and substantial software upgrades) all around.

Despite the common use of mean values in attenuation relations and loss estimation, we
decided that computing and depicting median values, which effectively reduces
information and discards the largest values of shaking, was not acceptable. This is
particularly true for highly directional near-fault pulse-like ground motions, for which
peak velocities can be large on one component and small on the other.  Mean values for
such motions (particularly when determined in logarithmic space) can seriously underrepresent
the largest motion that a building may have experienced, and these pulse-like motions are
typically associated with the regions of greatest damage. Thus, we chose peak ground
motions as the parameters to be mapped.

{\hyperref[references:beyer2006]{\crossref{\DUrole{std,std-ref}{Beyer and Bommer (2006)}}}} provide statistical relationships
to convert among median and
peak parameters and between aleatory variability for different definitions of the
horizontal component of motion. This is useful when approximating alternative
components than those presented, but one must recognize that for any individual record,
these statistics may or may not be representative.

Initially, our use of PGA and PGV for estimating intensities was also simply practical.
We were retrieving only peak values from a large subset of the network, so it was
impractical to compute more specific ground-motion parameters, such as average-
response spectral values, kinetic energy, cumulative absolute velocities (CAV, {\hyperref[references:epri1991]{\crossref{\DUrole{std,std-ref}{EPRI,
1991}}}}), or the JMA intensity algorithm ({\hyperref[references:jma1996]{\crossref{\DUrole{std,std-ref}{JMA, 1996}}}}). However, because
near-source strong ground motions are often dominated by short-duration, pulse-like
ground motions (usually associated with source directivity), PGV appears to be a robust measure
of intensity for strong shaking. In other words, the kinetic energy (proportional to
velocity squared) available for damage is well characterized by PGV. In addition, the
close correspondence of the JMA intensities and peak ground velocity
indicates that our use of peak ground velocities for higher intensities was
consistent with the algorithm used by JMA. Work by {\hyperref[references:wu2003]{\crossref{\DUrole{std,std-ref}{Wu et al. (2003)}}}} indicates a very
good correspondence between PGV and damage for data collected on the island of Taiwan,
which included high-quality loss data and densely sampled strong-motion observations
for the 1999 Chi-Chi earthquake. More recent work on Ground-Motion/Intensity
Conversion Equations (GMICEs) (e.g., {\hyperref[references:atkinson2007]{\crossref{\DUrole{std,std-ref}{Atkinson and Kaka, 2007}}}}; {\hyperref[references:worden2012]{\crossref{\DUrole{std,std-ref}{Worden, et al., 2012}}}}) has also
confirmed the strong relationship between PGV and intensity.

Nonetheless, for large, distant earthquakes, peak motions may be less informative, and
spectral content and perhaps duration become key parameters.  Although we may eventually
adopt corrections for these situations, it is difficult to assign intensities in such cases. For
instance, it is difficult to assign the intensity in the zone of Mexico City where numerous
high-rises collapsed during the 1985 Michoacan earthquake. There was obviously high-
intensity shaking for high-rise buildings; however, most smaller buildings were
unaffected, suggesting a much lower intensity.  Whereas PGVs were
moderate and would imply intensity VIII, resonance and duration conspired to cause a
more substantial damage than peak values would indicate. Although this is, in part, a
shortcoming of using peak parameters alone, it is more a limitation imposed by
simplifying the complexity of ground motions into a single parameter. Therefore, in
addition to providing peak ground-motion values and intensity, we are also producing
spectral response maps (for 0.3, 1.0, and 3.0 sec). Users who can process this information
for loss estimation will have a clearer picture than can be provided with maps of PGA
and PGV alone. However, as discussed earlier, a simple intensity map is extremely useful
for the overwhelming majority of users, which includes the general public and many
people involved with the initial emergency response.


\chapter{User's Guide}
\label{users_guide::doc}\label{users_guide:users-guide}\label{users_guide:user-s-guide}
ShakeMap originated primarily as an internet-based system to provide
real-time displays of earthquake shaking.
Although the online color-coded intensity maps are the most visible result of
the ShakeMap system and are the most commonly accessed and downloaded
products, they are just one representation of the ShakeMap output. ShakeMap also
produces grids of peak ground acceleration (PGA) and velocity
amplitudes (PGV), peak
spectral response values (PSA), instrumental intensities, seismic station files, fault
files, regression plots, contours, metadata, and uncertainty
estimates. The ShakeMap webpages also serve a variety of formats, including
PDF, KML, XML, GeoJSON, HAZUS-MH, GIS files, ESRI Raster, and a host of other
formats and products for varied user needs and applications.

In this User's Guide, after some background, we present the range of ShakeMap
products and describe the available formats. The different automated mechanisms
to receive and utilize ShakeMap---including GIS Web Services, GeoJSON feeds,
and ShakeCast---are described and links are provided. Many users also
take advantage of real-time ShakeMaps, as well as older events and earthquake
planning scenarios, so we describe the three primary ShakeMap repositories:
real-time ShakeMaps, historic ShakeMap Atlas events, and collections of
Scenario ShakeMaps.

Next is an overview of the current ShakeMap users and applications. Beyond
quickly assessing the overall domain of shaking, ShakeMap is used across many
sectors: planning and response, loss estimation, financial services, education
and outreach, and engineering and seismological research. We provide examples of
each. The section that follows expands on how ShakeMap integrates with the related
systems of DYFI, ShakeCast, and PAGER. Lastly, and importantly, are {\hyperref[disclaimers:sec\string-disclaimers]{\crossref{\DUrole{std,std-ref}{Disclaimers}}}}
that users should be aware of and refer to when employing ShakeMap as
part of their post-earthquake decision making.


\section{Background}
\label{background::doc}\label{background:sec-background}\label{background:background}
Until the development of ShakeMap, the most common information available
immediately following a significant earthquake was typically its magnitude and
epicenter.  However, the damage pattern is not a simple function of these two
parameters alone, and more-detailed information must be provided to properly
assess the hazard.  For example, for the 1971 M6.7 San Fernando,
California earthquake, the northern San Fernando Valley was the region with the
most damage, even though it was more than 15km from the epicenter.  Likewise,
areas strongly affected by the 1989 M6.9 Loma Prieta and 1994 M6.7 Northridge, California,
earthquakes that were either distant from
the epicentral region or out of the immediate media limelight were not fully
appreciated until long after the initial reports of damage. The full extent of
damage from the 1995 M6.9 Kobe, Japan, earthquake was not recognized by
the central government in Tokyo until many hours later (e.g., {\hyperref[references:yamakawa1998]{\crossref{\DUrole{std,std-ref}{Yamakawa, 1998}}}}),
seriously delaying rescue and recovery efforts.

In contrast, a ShakeMap is a representation of actual
ground shaking produced by an earthquake. The
information it presents is different from the earthquake magnitude and epicenter
that are released after an earthquake, because ShakeMap focuses on the
ground shaking produced by the earthquake, rather than the parameters describing
the earthquake starting point (its hypocenter) and size (magnitude). So,
although an earthquake has one magnitude and one epicenter, it produces a range
of ground shaking levels at sites throughout the region, depending on distance
from the earthquake fault that ruptured, the rock and soil conditions at sites,
and variations in the propagation of seismic waves from the earthquake due to
complexities in the structure of the Earth's crust.

Part of the strategy for generating rapid-response ground motion maps was to
determine the best format for reliable presentation of the maps given the
diverse audience, which includes scientists, businesses, emergency response
agencies, media, and the general public.  In an effort to simplify and maximize
the flow of information to general users, we have developed a means of generating
not only PGA and PGV maps, but also an
instrumentally derived estimated Modified Mercalli Intensity (MMI) map.  This
“instrumental intensity” map makes it easier to relate the recorded
ground motions to the expected felt area and damaging shaking distribution. At the same time,
we preserve a full range of utilities of recorded ground-motion data by
producing maps of response spectral acceleration, which are not particularly
useful to the general public, yet which provide fundamental data for loss
estimation and earthquake engineering assessments.

As mentioned, ShakeMap provides maps of \textbf{peak} ground-acceleration, velocity, and spectral
acceleration as well as MMI. Intensity ShakeMaps
depict estimated intensities derived from peak ground motions as well as
(optionally) from reported intensities. Intensity maps make it easier to relate
the recorded and estimated ground motions to the expected felt and damage
distributions. Intensities are estimated from ground shaking
using equations based on analyses
of intensities reported near recorded seismic stations for past
earthquakes, e.g., {\hyperref[references:wald1999b]{\crossref{\DUrole{std,std-ref}{Wald et al. (1999b)}}}} or
{\hyperref[references:worden2012]{\crossref{\DUrole{std,std-ref}{Worden et al. (2012)}}}}. The
legend on the ShakeMap indicates which relationship was used to estimate
intensities from ground motions and vice versa (see the ShakeMap
{\hyperref[technical_guide:technical\string-guide]{\crossref{\DUrole{std,std-ref}{Technical Guide}}}} for more details).

Station locations are the best indicator of where the map is most accurate: near
seismic stations, the shaking is well constrained by data; far from such
stations, the shaking is estimated using standard seismological inferences and
geospatial interpolation. Details about the interpolation; uncertainty maps; and
codes for the seismic station components, network contributors, and potential
outlier or clipped flag codes are provided in the {\hyperref[technical_guide:technical\string-guide]{\crossref{\DUrole{std,std-ref}{Technical Guide}}}}. Peak
horizontal acceleration and spectral acceleration values are in units of
percent-g (or \%g, where g = acceleration due to the force of gravity = 981cm/s/s). The
peak values of the vertical components are not used in the construction of the
maps. Peak velocity values are given (in cm/s) at each station. Acceleration
spectra are the response of a 5\% critically damped, single-degree-of-freedom
oscillator to the recorded ground motion at three reference periods: 0.3, 1.0,
and 3.0 sec.


\section{Products and Formats}
\label{products::doc}\label{products:sec-products}\label{products:products-and-formats}
ShakeMap is fundamentally a geographic product, providing a spatial representation of
the potentially very complex shaking field associated with an earthquake. Because of
its complicated nature, we are required to generate numerous maps that portray
various aspects of the shaking that are customized for specific uses or
audiences.  For some uses, it is not the maps themselves but the components that make up
the ShakeMaps that are of interest in order to re-create or further customize the
maps or user-specific products. In this section, we further describe
these ShakeMap component products and the variety of maps and formats.

For each earthquake, all maps and associated products for that event are available
via the “Downloads” link on each earthquake-specific ShakeMap webpage.


\subsection{Input Files}
\label{products:input-files}\label{products:sec-input-files}
The downloadable products include sufficient information to reproduce the
ShakeMap. In particular, \emph{stationlist.xml} and the \emph{*\_fault.txt} file(s) provide the
input files, \emph{grid.xml} provides the Vs30 grid (see above), and \emph{info.xml}
provides the important configuration and processing parameters, including the
name(s) of the fault file(s).

\textbf{Station Lists}. The file \emph{stationlist.xml} contains the combined input data from
all of the original processing center’s input files in a ShakeMap-readable
format. The file may contain seismic station data, intensity data, or a
combination of both. The file also contains an event tag with the earthquake
source specifications.
See the ShakeMap {\hyperref[software_guide:software\string-guide]{\crossref{\DUrole{std,std-ref}{Software \& Implementation Guide}}}} for a complete specification of
the ShakeMap-input XML formats.

For reasons of backward compatibility, we also provide \emph{stationlist.txt}. As with
\emph{grid.xyz}, the use of this file is deprecated and it may disappear in a future
release.

\textbf{Fault Files}. Fault files are named \emph{\textless{}something\textgreater{}\_fault.txt} and are listed in
\emph{info.xml}. Zero or more fault files may be present in the ShakeMap input
directory. See the ShakeMap {\hyperref[software_guide:software\string-guide]{\crossref{\DUrole{std,std-ref}{Software \& Implementation Guide}}}} for a complete specification
of the fault file format. For the purposes of reproducing the ShakeMap for an
earthquake, it is sufficient to copy the specified file(s) into the event’s
input directory.


\subsection{Output Files and Products}
\label{products:output-files-and-products}
The available ShakeMap products include (and each is described in more detail in the sections
that follow):
\begin{itemize}
\item {} \begin{description}
\item[{\textbf{Metadata and runtime information}}] \leavevmode\begin{itemize}
\item {} 
FGDC-compliant metadata

\item {} 
XML file of processing and constraints parameters, input data, output paramaters, timestamps, and versioning.

\end{itemize}

\end{description}

\item {} \begin{description}
\item[{\textbf{Static maps and plots (images)}}] \leavevmode\begin{itemize}
\item {} 
Macroseismic Intensity

\item {} 
Peak Ground Acceleration (PGA), Peak Ground Velocity (PGV), and Pseudo-Spectral Acceleration (PSA) (when appropriate)

\item {} 
Uncertainty maps

\item {} 
Regression (GMPE) plots

\item {} 
Station lists

\end{itemize}

\end{description}

\item {} 
\textbf{Interactive maps}

\item {} \begin{description}
\item[{\textbf{Grids of interpolated ground shaking}}] \leavevmode\begin{itemize}
\item {} 
XML grid of ground motions

\item {} 
XML grid of ground motions on “rock”

\item {} 
XML grid of ground-motion uncertainty

\item {} 
Text grid of ground motions (deprecated)

\end{itemize}

\end{description}

\item {} \begin{description}
\item[{\textbf{GIS files}}] \leavevmode\begin{itemize}
\item {} 
GIS Shapefiles

\item {} 
\href{http://www.fema.gov/hazus/}{HAZUS-MH®} Shapefiles

\item {} 
\href{http://www.esri.com}{ESRI} Raster Grid Files

\item {} 
\href{http://earth.google.com}{Google Earth} KML files

\item {} 
Contour Files

\end{itemize}

\end{description}

\end{itemize}


\subsubsection{Metadata and Runtime Information}
\label{products:metadata-and-runtime-information}
\textbf{Metadata}. FGDC-compliant geospatial metadata files are distributed via the
earthquake-specific ShakeMap webpage for each earthquake under the “Downloads”
page. The metadata are provided in text, HTML, and XML formats in the files
\emph{metadata.txt}, \emph{metadata.html}, and \emph{metadata.xml}, respectively.

Because the ShakeMap output grid is the fundamental derived product from the ShakeMap
processing, it is fully described in an accompanying metadata file following
Federal Geographic Data Committee (\href{https://www.fgdc.gov/}{FGDC}) standards
for geospatial information.  As described below, station amplitudes are provided
in separate ShakeMap station files; however, the complete metadata for the parametric data are
archived by the regional seismic networks and contributing strong-motion data
sources.

\textbf{Supplemental Information}. A second aggregation of important
earthquake-specific ShakeMap information is provided online in the file
\emph{info.xml}. This supplemental information provides a machine-readable (XML)
rundown of many important ShakeMap processing parameters. It includes
information about the data and fault input files; the source mechanism; the
GMPE, IPE, and GMICE selected; the type and source of the site amplifications;
the map boundaries; and important output information, including the bias and
maximum amplitude for each parameter. The \emph{info.xml} is critical for
understanding or replicating any particular ShakeMap.

\begin{notice}{note}{Note:}
Timestamps, versions of the ShakeMap software employed, event-specific parameters, and the version of the specific ShakeMap run are documented in the supplemental information provided in the \emph{info.xml} file.
\end{notice}


\subsubsection{Static Maps and Plots (Images)}
\label{products:static-maps-and-plots-images}
ShakeMap generates a number of static ground-motion maps and plots for various
parameters (intensity measures, or IMs). Most of these maps are available in
JPEG format, as well as zipped PostScript files that---as vector-based
images---are suitable for PDF conversion or editing. These maps are typically
generated automatically, limiting the format, extent, and features that can be
depicted. Nonetheless, these static maps are ShakeMap’s ``signature products'' and
serve as maps of record and for other purposes, as described below. Static maps
can be accessed and selected using tabs along the top of the USGS earthquake event
page, as shown in the example in \hyperref[products:napa-event-page]{Figure  \ref*{products:napa-event-page}}.
\begin{figure}[htbp]\begin{flushleft}
\capstart

\includegraphics{{Napa_Event_Page}.png}
\caption{Event page ShakeMap view for the 2014 M6.0 American Canyon (Napa Valley), CA
earthquake. The static instrumental intensity map is shown. Tabs above the map allow access and
comparison of different intensity measures (IMs), as well as the uncertainty map
and station list.}\label{products:napa-event-page}\label{products:id3}\end{flushleft}\end{figure}

\textbf{Intensity Maps}. Intensity images---typically of Modified Mercalli Intensity
(MMI), but potentially other intensity measures---are the most familiar ShakeMap
products. The main intensity map consists of a colored overlay of intensity with
the epicenter (and the causative fault, if supplied) prominently marked,
(usually) overlain upon the region’s topography, with other cultural and
geologic features (cities, roads, and active faults) plotted, depending on the
configuration of the ShakeMap system. A detailed scale of intensity is also
provided as described in detail in the {\hyperref[technical_guide:technical\string-guide]{\crossref{\DUrole{std,std-ref}{Technical Guide}}}}.

\begin{notice}{note}{Note:}
\textbf{ShakeMap Symbology}. It is a recent ShakeMap convention to depict seismic stations as \textbf{triangles} and intensity observations as \textbf{circles} (for cities) or \textbf{squares} (for geocoded boxes). On intensity maps, symbols are unfilled so that the underlying intensity values are visible. On peak ground motion maps, observations are (optionally) color coded to their amplitude according to the legend shown below each map. The epicenter is indicated with a \textbf{star}, and for larger earthquakes, the surface projection of the causative fault is shown with \textbf{black lines}.
\end{notice}

Strong motion and intensity data symbols default to ``see-through” (unfilled) mode for the
intensity map shown in \hyperref[products:napa-shakemap-cover]{Figure  \ref*{products:napa-shakemap-cover}} and are color-filled
for peak ground motion maps (\hyperref[products:napa-pga]{Figure  \ref*{products:napa-pga}}). ShakeMap operators may
chose to modify these defaults using alternative mapping configurations.
\begin{figure}[htbp]
\centering
\capstart

\includegraphics{{Napa_ShakeMap_cover}.png}
\caption{Intensity ShakeMap from the 2014 M6.0 American Canyon (Napa Valley), CA earthquake. Strong-motion data
(triangles) and intensity data (circles) default to ``see-through” mode for the intensity map. The
north-south black line indicates the fault location, and the epicenter is a red star. The intensity
color-coding either as observed (for macroseismic data) or as converted is derived from the conversion equations of {\hyperref[references:wald1999b]{\crossref{\DUrole{std,std-ref}{Wald et al. (1999b)}}}} as shown in the legend. Note: Map Version Number reflects separate offline processing for this Manual.}\label{products:napa-shakemap-cover}\label{products:id4}\end{figure}
\begin{figure}[htbp]
\centering
\capstart

\includegraphics{{Figure_1_2}.png}
\caption{Peak acceleration ShakeMap from the 2014 M6.0 American Canyon (Napa Valley), CA
earthquake. Strong-motion data (triangles) and intensity data (circles) are color-coded according to their intensity
value, either as observed (for macroseismic data) or as converted by
{\hyperref[references:wald1999b]{\crossref{\DUrole{std,std-ref}{Wald et al. (1999b)}}}} as shown in the
legend. The north-south black line indicates the fault location, which nucleated near the epicenter (red star). Note:
Map Version Number reflects separate offline processing for this Manual.}\label{products:napa-pga}\label{products:id5}\end{figure}

\textbf{Peak Ground Motion Maps.} ShakeMap generates static maps for PGA, PGV, and
Intensity, and optionally, three separate maps for PSA
(at 0.3, 1.0, and 3.0 sec). The PGM maps are distinct from the
intensity maps: shaking values on the former are colored image overlays; the latter are
PGM contours. On PGM maps, stations' fill colors
indicate the ground motion of the station converted to intensity or, optionally,
the identity of the seismic network data source. When the color indicates peak
ground motion, the values are converted to the intensity color scheme via the
selected ground-motion--intensity conversion equation (GMICE), and the
corresponding color scale bar is provided at the bottom of the map (see example
in \hyperref[products:napa-pga]{Figure  \ref*{products:napa-pga}}).


\subsubsection{Interactive Maps}
\label{products:interactive-maps}
Although the static ShakeMaps are useful, many of these products are more suitably
served as interactive maps which can be dynamically scaled (zoomed) and layered upon
with user-selected background and other overlays. The layers are provided via
GeoJSON, KML, GIS, Raster, and other formats. The USGS Earthquake Program Web
pages employ \href{http://leafletjs.com/}{Leaflet}, an open-source JavaScript
library that is suitable for mobile-friendly interactive maps (see, for example,
\hyperref[products:napa-contours]{Figure  \ref*{products:napa-contours}}). Many of the
interactive features are geared towards balancing the experience for both
desktop and mobile visitors (\hyperref[products:napa-mobile]{Figure  \ref*{products:napa-mobile}}). Since
the interactive maps are zoomable, it is convenient to select
individual stations to query station
information and amplitudes (see the example in \hyperref[products:napa-stationpopup]{Figure  \ref*{products:napa-stationpopup}}).
The interactive map also allows users to select and show/hide specific layers,
including seismic stations and DYFI geocoded intensity
stations (\hyperref[products:napa-dyfi]{Figure  \ref*{products:napa-dyfi}}).
\begin{figure}[htbp]\begin{flushleft}
\capstart

\includegraphics{{Napa_contours_station}.png}
\caption{Interactive ShakeMap for the 2014 M6.0 American Canyon, CA
earthquake. Contours indicate intensities; strong motion data (triangles) and intensity data (circles) are
color-coded according to their intensity value, either as observed (for macroseismic data) or as converted
by {\hyperref[references:worden2012]{\crossref{\DUrole{std,std-ref}{Worden et al. (2012)}}}}.}\label{products:napa-contours}\label{products:id6}\end{flushleft}\end{figure}
\begin{figure}[htbp]
\centering
\capstart

\scalebox{0.400000}{\includegraphics{{Napa_mobile_shakemap}.png}}
\caption{Mobile view of the interactive ShakeMap for the 2014 M6.0 American Canyon, CA
earthquake. Contours indicate intensities; strong motion data (triangles) are color-coded according to their intensity
value.}\label{products:napa-mobile}\label{products:id7}\end{figure}
\begin{figure}[htbp]\begin{flushleft}
\capstart

\includegraphics{{Napa_contours_station_popup}.png}
\caption{Interactive ShakeMap for the 2014 M6.0 American Canyon, CA
earthquake showing station information pop-up.}\label{products:napa-stationpopup}\label{products:id8}\end{flushleft}\end{figure}
\begin{figure}[htbp]\begin{flushleft}
\capstart

\includegraphics{{Napa_contours-stas-dyfi}.png}
\caption{Interactive ShakeMap for the 2014 M6.0 American Canyon, CA
earthquake. On the interactive map, reported (DYFI) intensities are geocoded and represented with
\textbf{squares} depicting the 1km grid area they occupy. Reported Intensities are color-coded according to their intensity
value, either as observed or as converted by {\hyperref[references:wald1999b]{\crossref{\DUrole{std,std-ref}{Wald et al. (1999b)}}}}.}\label{products:napa-dyfi}\label{products:id9}\end{flushleft}\end{figure}

The interactive maps may be accessed by clicking on the static ShakeMaps on the
USGS event pages (e.g., \url{http://earthquake.usgs.gov/earthquakes/eventpage/us10003zgz\#impact\_shakemap}).

\begin{notice}{note}{Note:}
Currently, interactive maps only portray contours of intensity. Other contours can be downloaded for users' programs, or overlain with the GIS or KML formats provided with each ShakeMap.
\end{notice}

\textbf{Uncertainty Maps}. As discussed in detail in the {\hyperref[technical_guide:technical\string-guide]{\crossref{\DUrole{std,std-ref}{Technical Guide}}}},
gridded uncertainty is available for all ground motion parameters. The ratio of
the ShakeMap PGA uncertainty to the GMPE’s uncertainty is also available (see
the section on {\hyperref[tg_processing:sec\string-interpolation]{\crossref{\DUrole{std,std-ref}{Interpolation}}}}).

We utilize the uncertainty ratio to produce a graded map of uncertainty. Where
the ratio is 1.0 (meaning the ShakeMap is purely predictive), the map is colored
white. Where the ratio is greater than 1.0 (meaning that the ShakeMap
uncertainty is high because of unknown fault geometry), the map shades toward
dark red, and where the uncertainty is less than 1.0 (because the presence of
data decreases the uncertainty), the map shades toward dark blue. These maps
provide a quick visual summary of the quality of ground-motion estimates over
the area of interest.

ShakeMaps are also given a letter grade based on the mean uncertainty ratio
within the area of the MMI-VI contour (on the theory that this is the area most
important to accurately represent). A ratio of 1.0 is given a grade of “C”; maps
with mean ratios greater than 1.0 get grades of “D” or “F”; ratios less than 1.0
earn grades of “B” or “A”. If the map does not contain areas of MMI \textgreater{}= VI, no
grade is assigned. See \hyperref[products:napa-urat]{Figure  \ref*{products:napa-urat}} for an example uncertainty map.
\begin{figure}[htbp]\begin{flushleft}
\capstart

\includegraphics{{Napa_urat_pga}.pdf}
\caption{ShakeMap uncertainty map for the 2014 M6.0 American Canyon, CA
earthquake. Color-coded legend shows uncertainty ratio, where “1.0”
indicates 1.0 times the GMPE’s sigma. The average uncertainty is
computed by averaging uncertainty at grid points that lie
within the MMI-VI contour (bold contour line). For more details, see
{\hyperref[references:wald2008]{\crossref{\DUrole{std,std-ref}{Wald et al. (2008)}}}},
{\hyperref[references:worden2010]{\crossref{\DUrole{std,std-ref}{Worden et al. (2010)}}}}, and the {\hyperref[technical_guide:technical\string-guide]{\crossref{\DUrole{std,std-ref}{Technical Guide}}}}.}\label{products:napa-urat}\label{products:id10}\end{flushleft}\end{figure}

\textbf{Regression (GMPE and Distance Attenuation) Plots.}

ShakeMap can also (optionally) produce graphs of the observational data plotted with the biased
and unbiased GMPE. For example, \hyperref[products:northridge-mi-regr-w-dyfi]{Figure  \ref*{products:northridge-mi-regr-w-dyfi}} shows
the 1994 M6.7 Northridge earthquake MMI data, and \hyperref[products:northridge-pga-regr-w-dyfi]{Figure  \ref*{products:northridge-pga-regr-w-dyfi}} shows the
PGA data and GMPE.
\begin{figure}[htbp]\begin{flushleft}
\capstart

\includegraphics{{northridge_mi_regr_w_dyfi}.pdf}
\caption{Plot showing the 1994 M6.7 Northridge, CA earthquake MMI data (seismic stations are yellow triangles;
DYFI observations are blue circles) plotted with the unbiased (red line) and biased
(green line) IPE. The dashed green lines show the biased IPE \(\pm\)3 standard deviations.}\label{products:northridge-mi-regr-w-dyfi}\label{products:id11}\end{flushleft}\end{figure}
\begin{figure}[htbp]\begin{flushleft}
\capstart

\includegraphics{{northridge_pga_regr_w_dyfi}.pdf}
\caption{Plot showing the 1994 M6.7 Northridge, CA earthquake PGA data (seismic stations are yellow triangles;
DYFI observations are blue circles) plotted with the unbiased (red line) and biased
(green line) GMPE. The dashed green lines show the biased GMPE \(\pm\)3 standard deviations.}\label{products:northridge-pga-regr-w-dyfi}\label{products:id12}\end{flushleft}\end{figure}


\subsubsection{Interpolated Ground Motion Grids}
\label{products:interpolated-ground-motion-grids}\label{products:sec-interpolated-grid-file}
As described in the {\hyperref[technical_guide:technical\string-guide]{\crossref{\DUrole{std,std-ref}{Technical Guide}}}}, the fundamental output product of the
ShakeMap processing system is a finely-sampled grid (nominally 1km
spacing) of latitude and longitude
pairs with associated amplitude values of shaking parameters at each point.
These amplitude values are derived by interpolation of a combination of the
recorded ground shaking observations and estimated amplitudes, with consideration
of site amplification at all interpolated points.  The resulting grid of
amplitude values provides the basis for generating color-coded intensity contour
maps, for further interpolation to infer shaking at selected locations, and for
generating GIS-formatted files for further analyses.

\textbf{XML Grid}. The ShakeMap XML grid file is the basis for nearly all ShakeMap
products, as well as for computerized post-processing in systems such as
ShakeCast and PAGER {[}see {\hyperref[related_systems:sec\string-related\string-systems]{\crossref{\DUrole{std,std-ref}{Related Systems}}}}{]}. The XML grid is available
as both plain text (\emph{grid.xml}) and compressed as a zip file
(\emph{grid.xml.zip}). As XML, the grid is meant to be self-describing; however, we describe the format
here for the sake of completeness.

After the XML header, the first line is the \emph{shakemap\_grid} tag:
\begin{quote}

\begin{Verbatim}[commandchars=\\\{\}]
\PYG{o}{\PYGZlt{}}\PYG{n}{shakemap\PYGZus{}grid} \PYG{n}{xsi}\PYG{p}{:}\PYG{n}{schemaLocation}\PYG{o}{=}\PYG{l+s+s2}{\PYGZdq{}}\PYG{l+s+s2}{http://earthquake.usgs.gov}
 \PYG{n}{http}\PYG{p}{:}\PYG{o}{/}\PYG{o}{/}\PYG{n}{earthquake}\PYG{o}{.}\PYG{n}{usgs}\PYG{o}{.}\PYG{n}{gov}\PYG{o}{/}\PYG{n}{eqcenter}\PYG{o}{/}\PYG{n}{shakemap}\PYG{o}{/}\PYG{n}{xml}\PYG{o}{/}\PYG{n}{schemas}\PYG{o}{/}\PYG{n}{shakemap}\PYG{o}{.}\PYG{n}{xsd}\PYG{l+s+s2}{\PYGZdq{}}
 \PYG{n}{event\PYGZus{}id}\PYG{o}{=}\PYG{l+s+s2}{\PYGZdq{}}\PYG{l+s+s2}{19940117123055}\PYG{l+s+s2}{\PYGZdq{}} \PYG{n}{shakemap\PYGZus{}id}\PYG{o}{=}\PYG{l+s+s2}{\PYGZdq{}}\PYG{l+s+s2}{19940117123055}\PYG{l+s+s2}{\PYGZdq{}} \PYG{n}{shakemap\PYGZus{}version}\PYG{o}{=}\PYG{l+s+s2}{\PYGZdq{}}\PYG{l+s+s2}{2}\PYG{l+s+s2}{\PYGZdq{}}
 \PYG{n}{code\PYGZus{}version}\PYG{o}{=}\PYG{l+s+s2}{\PYGZdq{}}\PYG{l+s+s2}{3.5.1446}\PYG{l+s+s2}{\PYGZdq{}} \PYG{n}{process\PYGZus{}timestamp}\PYG{o}{=}\PYG{l+s+s2}{\PYGZdq{}}\PYG{l+s+s2}{2015\PYGZhy{}10\PYGZhy{}30T20:38:19Z}\PYG{l+s+s2}{\PYGZdq{}}
 \PYG{n}{shakemap\PYGZus{}originator}\PYG{o}{=}\PYG{l+s+s2}{\PYGZdq{}}\PYG{l+s+s2}{us}\PYG{l+s+s2}{\PYGZdq{}} \PYG{n}{map\PYGZus{}status}\PYG{o}{=}\PYG{l+s+s2}{\PYGZdq{}}\PYG{l+s+s2}{RELEASED}\PYG{l+s+s2}{\PYGZdq{}} \PYG{n}{shakemap\PYGZus{}event\PYGZus{}type}\PYG{o}{=}\PYG{l+s+s2}{\PYGZdq{}}\PYG{l+s+s2}{ACTUAL}\PYG{l+s+s2}{\PYGZdq{}}\PYG{o}{\PYGZgt{}}
\end{Verbatim}
\end{quote}

Aside from schema information, the \emph{shakemap\_grid} tag provides the following attributes:
\begin{itemize}
\item {} 
\emph{event\_id}: Typically this is a string of numbers and/or letters with or without a network
ID prefix (e.g., “us100003ywp”), though in the case of major historic earthquakes, scenarios, or
other special cases it may be a descriptive string (for example, we have previously used the 1994
Northridge earthquake as an example, and its event\_id is “Northridge”).

\item {} 
\emph{shakemap\_id}: Currently the same as \emph{event\_id}, above.

\item {} 
\emph{shakemap\_version}: The version of this map, incremented each time a map is revised or reprocessed
and transferred.

\item {} 
\emph{code\_version}: The version of the ShakeMap software used to make the map.

\item {} 
\emph{process\_timestamp}: The date and time the event was processed.

\item {} 
\emph{shakemap\_originator}: The network code of the center that produced the map.

\item {} 
\emph{map\_status}: Currently always the string “RELEASED”, but other strings may be used in the future.

\item {} 
\emph{shakemap\_event\_type}: Either “ACTUAL” (for real earthquakes) or “SCENARIO” (for scenarios).

\end{itemize}

The next tag describes the earthquake source:
\begin{quote}

\begin{Verbatim}[commandchars=\\\{\}]
\PYG{o}{\PYGZlt{}}\PYG{n}{event} \PYG{n}{event\PYGZus{}id}\PYG{o}{=}\PYG{l+s+s2}{\PYGZdq{}}\PYG{l+s+s2}{Northridge}\PYG{l+s+s2}{\PYGZdq{}} \PYG{n}{magnitude}\PYG{o}{=}\PYG{l+s+s2}{\PYGZdq{}}\PYG{l+s+s2}{6.7}\PYG{l+s+s2}{\PYGZdq{}} \PYG{n}{depth}\PYG{o}{=}\PYG{l+s+s2}{\PYGZdq{}}\PYG{l+s+s2}{18}\PYG{l+s+s2}{\PYGZdq{}} \PYG{n}{lat}\PYG{o}{=}\PYG{l+s+s2}{\PYGZdq{}}\PYG{l+s+s2}{34.213000}\PYG{l+s+s2}{\PYGZdq{}}
 \PYG{n}{lon}\PYG{o}{=}\PYG{l+s+s2}{\PYGZdq{}}\PYG{l+s+s2}{\PYGZhy{}118.535700}\PYG{l+s+s2}{\PYGZdq{}} \PYG{n}{event\PYGZus{}timestamp}\PYG{o}{=}\PYG{l+s+s2}{\PYGZdq{}}\PYG{l+s+s2}{1994\PYGZhy{}01\PYGZhy{}17T12:30:55GMT}\PYG{l+s+s2}{\PYGZdq{}} \PYG{n}{event\PYGZus{}network}\PYG{o}{=}\PYG{l+s+s2}{\PYGZdq{}}\PYG{l+s+s2}{ci}\PYG{l+s+s2}{\PYGZdq{}}
 \PYG{n}{event\PYGZus{}description}\PYG{o}{=}\PYG{l+s+s2}{\PYGZdq{}}\PYG{l+s+s2}{Northridge}\PYG{l+s+s2}{\PYGZdq{}} \PYG{o}{/}\PYG{o}{\PYGZgt{}}
\end{Verbatim}
\end{quote}

Most of the attributes are self-explanatory:
\begin{itemize}
\item {} 
\emph{event\_id}: See above.

\item {} 
\emph{magnitude}: The earthquake magnitude.

\item {} 
\emph{depth}: The depth (in km) of the earthquake hypocenter.

\item {} 
\emph{lat/lon}: The latitude and longitude of the earthquake epicenter.

\item {} 
\emph{event\_timestamp}: The date and time of the earthquake.

\item {} 
\emph{event\_network}: The authoritative seismic network in which the earthquake occurred.

\item {} 
\emph{event\_description}: A string containing the earthquake name or a location string (e.g., “13 km SW of Newhall, CA”).

\end{itemize}

Following the event tag is the grid\_specification tag:
\begin{quote}

\begin{Verbatim}[commandchars=\\\{\}]
\PYG{o}{\PYGZlt{}}\PYG{n}{grid\PYGZus{}specification} \PYG{n}{lon\PYGZus{}min}\PYG{o}{=}\PYG{l+s+s2}{\PYGZdq{}}\PYG{l+s+s2}{\PYGZhy{}119.785700}\PYG{l+s+s2}{\PYGZdq{}} \PYG{n}{lat\PYGZus{}min}\PYG{o}{=}\PYG{l+s+s2}{\PYGZdq{}}\PYG{l+s+s2}{33.379666}\PYG{l+s+s2}{\PYGZdq{}} \PYG{n}{lon\PYGZus{}max}\PYG{o}{=}\PYG{l+s+s2}{\PYGZdq{}}\PYG{l+s+s2}{\PYGZhy{}117.285700}\PYG{l+s+s2}{\PYGZdq{}}
 \PYG{n}{lat\PYGZus{}max}\PYG{o}{=}\PYG{l+s+s2}{\PYGZdq{}}\PYG{l+s+s2}{35.046334}\PYG{l+s+s2}{\PYGZdq{}} \PYG{n}{nominal\PYGZus{}lon\PYGZus{}spacing}\PYG{o}{=}\PYG{l+s+s2}{\PYGZdq{}}\PYG{l+s+s2}{0.008333}\PYG{l+s+s2}{\PYGZdq{}} \PYG{n}{nominal\PYGZus{}lat\PYGZus{}spacing}\PYG{o}{=}\PYG{l+s+s2}{\PYGZdq{}}\PYG{l+s+s2}{0.008333}\PYG{l+s+s2}{\PYGZdq{}}
 \PYG{n}{nlon}\PYG{o}{=}\PYG{l+s+s2}{\PYGZdq{}}\PYG{l+s+s2}{301}\PYG{l+s+s2}{\PYGZdq{}} \PYG{n}{nlat}\PYG{o}{=}\PYG{l+s+s2}{\PYGZdq{}}\PYG{l+s+s2}{201}\PYG{l+s+s2}{\PYGZdq{}} \PYG{o}{/}\PYG{o}{\PYGZgt{}}
\end{Verbatim}
\end{quote}

The attributes are:
\begin{itemize}
\item {} 
\emph{lon\_min/lon\_max}: The boundaries of the grid in longitude.

\item {} 
\emph{lat\_min/lat\_max}: The boundaries of the grid in latitude.

\item {} 
\emph{nominal\_lon\_spacing}: The expected grid interval in longitude within the resolution of the
numeric format of the output.

\item {} 
\emph{nominal\_lat\_spacing}: The expected grid interval in latitude within the resolution of the
numeric format of the output.

\item {} 
\emph{nlon/nlat}: The number of grid points in longitude and latitude. The grid data table will
contain nlon times nlat rows.

\end{itemize}

Following the \emph{grid\_specification} tag will be a set of event-specific uncertainty tags:
\begin{quote}

\begin{Verbatim}[commandchars=\\\{\}]
\PYG{o}{\PYGZlt{}}\PYG{n}{event\PYGZus{}specific\PYGZus{}uncertainty} \PYG{n}{name}\PYG{o}{=}\PYG{l+s+s2}{\PYGZdq{}}\PYG{l+s+s2}{pga}\PYG{l+s+s2}{\PYGZdq{}} \PYG{n}{value}\PYG{o}{=}\PYG{l+s+s2}{\PYGZdq{}}\PYG{l+s+s2}{0.466260}\PYG{l+s+s2}{\PYGZdq{}} \PYG{n}{numsta}\PYG{o}{=}\PYG{l+s+s2}{\PYGZdq{}}\PYG{l+s+s2}{598}\PYG{l+s+s2}{\PYGZdq{}} \PYG{o}{/}\PYG{o}{\PYGZgt{}}
\PYG{o}{\PYGZlt{}}\PYG{n}{event\PYGZus{}specific\PYGZus{}uncertainty} \PYG{n}{name}\PYG{o}{=}\PYG{l+s+s2}{\PYGZdq{}}\PYG{l+s+s2}{pgv}\PYG{l+s+s2}{\PYGZdq{}} \PYG{n}{value}\PYG{o}{=}\PYG{l+s+s2}{\PYGZdq{}}\PYG{l+s+s2}{0.464209}\PYG{l+s+s2}{\PYGZdq{}} \PYG{n}{numsta}\PYG{o}{=}\PYG{l+s+s2}{\PYGZdq{}}\PYG{l+s+s2}{595}\PYG{l+s+s2}{\PYGZdq{}} \PYG{o}{/}\PYG{o}{\PYGZgt{}}
\PYG{o}{\PYGZlt{}}\PYG{n}{event\PYGZus{}specific\PYGZus{}uncertainty} \PYG{n}{name}\PYG{o}{=}\PYG{l+s+s2}{\PYGZdq{}}\PYG{l+s+s2}{mi}\PYG{l+s+s2}{\PYGZdq{}} \PYG{n}{value}\PYG{o}{=}\PYG{l+s+s2}{\PYGZdq{}}\PYG{l+s+s2}{0.624327}\PYG{l+s+s2}{\PYGZdq{}} \PYG{n}{numsta}\PYG{o}{=}\PYG{l+s+s2}{\PYGZdq{}}\PYG{l+s+s2}{598}\PYG{l+s+s2}{\PYGZdq{}} \PYG{o}{/}\PYG{o}{\PYGZgt{}}
\PYG{o}{\PYGZlt{}}\PYG{n}{event\PYGZus{}specific\PYGZus{}uncertainty} \PYG{n}{name}\PYG{o}{=}\PYG{l+s+s2}{\PYGZdq{}}\PYG{l+s+s2}{psa03}\PYG{l+s+s2}{\PYGZdq{}} \PYG{n}{value}\PYG{o}{=}\PYG{l+s+s2}{\PYGZdq{}}\PYG{l+s+s2}{0.436803}\PYG{l+s+s2}{\PYGZdq{}} \PYG{n}{numsta}\PYG{o}{=}\PYG{l+s+s2}{\PYGZdq{}}\PYG{l+s+s2}{594}\PYG{l+s+s2}{\PYGZdq{}} \PYG{o}{/}\PYG{o}{\PYGZgt{}}
\PYG{o}{\PYGZlt{}}\PYG{n}{event\PYGZus{}specific\PYGZus{}uncertainty} \PYG{n}{name}\PYG{o}{=}\PYG{l+s+s2}{\PYGZdq{}}\PYG{l+s+s2}{psa10}\PYG{l+s+s2}{\PYGZdq{}} \PYG{n}{value}\PYG{o}{=}\PYG{l+s+s2}{\PYGZdq{}}\PYG{l+s+s2}{0.534212}\PYG{l+s+s2}{\PYGZdq{}} \PYG{n}{numsta}\PYG{o}{=}\PYG{l+s+s2}{\PYGZdq{}}\PYG{l+s+s2}{595}\PYG{l+s+s2}{\PYGZdq{}} \PYG{o}{/}\PYG{o}{\PYGZgt{}}
\PYG{o}{\PYGZlt{}}\PYG{n}{event\PYGZus{}specific\PYGZus{}uncertainty} \PYG{n}{name}\PYG{o}{=}\PYG{l+s+s2}{\PYGZdq{}}\PYG{l+s+s2}{psa30}\PYG{l+s+s2}{\PYGZdq{}} \PYG{n}{value}\PYG{o}{=}\PYG{l+s+s2}{\PYGZdq{}}\PYG{l+s+s2}{0.577897}\PYG{l+s+s2}{\PYGZdq{}} \PYG{n}{numsta}\PYG{o}{=}\PYG{l+s+s2}{\PYGZdq{}}\PYG{l+s+s2}{594}\PYG{l+s+s2}{\PYGZdq{}} \PYG{o}{/}\PYG{o}{\PYGZgt{}}
\end{Verbatim}
\end{quote}

These tags provide the uncertainty for the ground motion parameters (natural log units
for all but intensity, which is in linear units) computed as a misfit from the
biased GMPE (IPE). This is equivalent to the intra-event uncertainty. The number of
stations contributing to each uncertainty is also provided. If the number of stations
falls below the minimum required to compute the bias, the uncertainty value will be
set to -1.

These lines are followed by a number of grid\_field tags:
\begin{quote}

\begin{Verbatim}[commandchars=\\\{\}]
\PYG{o}{\PYGZlt{}}\PYG{n}{grid\PYGZus{}field} \PYG{n}{index}\PYG{o}{=}\PYG{l+s+s2}{\PYGZdq{}}\PYG{l+s+s2}{1}\PYG{l+s+s2}{\PYGZdq{}} \PYG{n}{name}\PYG{o}{=}\PYG{l+s+s2}{\PYGZdq{}}\PYG{l+s+s2}{LON}\PYG{l+s+s2}{\PYGZdq{}} \PYG{n}{units}\PYG{o}{=}\PYG{l+s+s2}{\PYGZdq{}}\PYG{l+s+s2}{dd}\PYG{l+s+s2}{\PYGZdq{}} \PYG{o}{/}\PYG{o}{\PYGZgt{}}
\PYG{o}{\PYGZlt{}}\PYG{n}{grid\PYGZus{}field} \PYG{n}{index}\PYG{o}{=}\PYG{l+s+s2}{\PYGZdq{}}\PYG{l+s+s2}{2}\PYG{l+s+s2}{\PYGZdq{}} \PYG{n}{name}\PYG{o}{=}\PYG{l+s+s2}{\PYGZdq{}}\PYG{l+s+s2}{LAT}\PYG{l+s+s2}{\PYGZdq{}} \PYG{n}{units}\PYG{o}{=}\PYG{l+s+s2}{\PYGZdq{}}\PYG{l+s+s2}{dd}\PYG{l+s+s2}{\PYGZdq{}} \PYG{o}{/}\PYG{o}{\PYGZgt{}}
\PYG{o}{\PYGZlt{}}\PYG{n}{grid\PYGZus{}field} \PYG{n}{index}\PYG{o}{=}\PYG{l+s+s2}{\PYGZdq{}}\PYG{l+s+s2}{3}\PYG{l+s+s2}{\PYGZdq{}} \PYG{n}{name}\PYG{o}{=}\PYG{l+s+s2}{\PYGZdq{}}\PYG{l+s+s2}{PGA}\PYG{l+s+s2}{\PYGZdq{}} \PYG{n}{units}\PYG{o}{=}\PYG{l+s+s2}{\PYGZdq{}}\PYG{l+s+s2}{pctg}\PYG{l+s+s2}{\PYGZdq{}} \PYG{o}{/}\PYG{o}{\PYGZgt{}}
\PYG{o}{\PYGZlt{}}\PYG{n}{grid\PYGZus{}field} \PYG{n}{index}\PYG{o}{=}\PYG{l+s+s2}{\PYGZdq{}}\PYG{l+s+s2}{4}\PYG{l+s+s2}{\PYGZdq{}} \PYG{n}{name}\PYG{o}{=}\PYG{l+s+s2}{\PYGZdq{}}\PYG{l+s+s2}{PGV}\PYG{l+s+s2}{\PYGZdq{}} \PYG{n}{units}\PYG{o}{=}\PYG{l+s+s2}{\PYGZdq{}}\PYG{l+s+s2}{cms}\PYG{l+s+s2}{\PYGZdq{}} \PYG{o}{/}\PYG{o}{\PYGZgt{}}
\PYG{o}{\PYGZlt{}}\PYG{n}{grid\PYGZus{}field} \PYG{n}{index}\PYG{o}{=}\PYG{l+s+s2}{\PYGZdq{}}\PYG{l+s+s2}{5}\PYG{l+s+s2}{\PYGZdq{}} \PYG{n}{name}\PYG{o}{=}\PYG{l+s+s2}{\PYGZdq{}}\PYG{l+s+s2}{MMI}\PYG{l+s+s2}{\PYGZdq{}} \PYG{n}{units}\PYG{o}{=}\PYG{l+s+s2}{\PYGZdq{}}\PYG{l+s+s2}{intensity}\PYG{l+s+s2}{\PYGZdq{}} \PYG{o}{/}\PYG{o}{\PYGZgt{}}
\PYG{o}{\PYGZlt{}}\PYG{n}{grid\PYGZus{}field} \PYG{n}{index}\PYG{o}{=}\PYG{l+s+s2}{\PYGZdq{}}\PYG{l+s+s2}{6}\PYG{l+s+s2}{\PYGZdq{}} \PYG{n}{name}\PYG{o}{=}\PYG{l+s+s2}{\PYGZdq{}}\PYG{l+s+s2}{PSA03}\PYG{l+s+s2}{\PYGZdq{}} \PYG{n}{units}\PYG{o}{=}\PYG{l+s+s2}{\PYGZdq{}}\PYG{l+s+s2}{pctg}\PYG{l+s+s2}{\PYGZdq{}} \PYG{o}{/}\PYG{o}{\PYGZgt{}}
\PYG{o}{\PYGZlt{}}\PYG{n}{grid\PYGZus{}field} \PYG{n}{index}\PYG{o}{=}\PYG{l+s+s2}{\PYGZdq{}}\PYG{l+s+s2}{7}\PYG{l+s+s2}{\PYGZdq{}} \PYG{n}{name}\PYG{o}{=}\PYG{l+s+s2}{\PYGZdq{}}\PYG{l+s+s2}{PSA10}\PYG{l+s+s2}{\PYGZdq{}} \PYG{n}{units}\PYG{o}{=}\PYG{l+s+s2}{\PYGZdq{}}\PYG{l+s+s2}{pctg}\PYG{l+s+s2}{\PYGZdq{}} \PYG{o}{/}\PYG{o}{\PYGZgt{}}
\PYG{o}{\PYGZlt{}}\PYG{n}{grid\PYGZus{}field} \PYG{n}{index}\PYG{o}{=}\PYG{l+s+s2}{\PYGZdq{}}\PYG{l+s+s2}{8}\PYG{l+s+s2}{\PYGZdq{}} \PYG{n}{name}\PYG{o}{=}\PYG{l+s+s2}{\PYGZdq{}}\PYG{l+s+s2}{PSA30}\PYG{l+s+s2}{\PYGZdq{}} \PYG{n}{units}\PYG{o}{=}\PYG{l+s+s2}{\PYGZdq{}}\PYG{l+s+s2}{pctg}\PYG{l+s+s2}{\PYGZdq{}} \PYG{o}{/}\PYG{o}{\PYGZgt{}}
\PYG{o}{\PYGZlt{}}\PYG{n}{grid\PYGZus{}field} \PYG{n}{index}\PYG{o}{=}\PYG{l+s+s2}{\PYGZdq{}}\PYG{l+s+s2}{9}\PYG{l+s+s2}{\PYGZdq{}} \PYG{n}{name}\PYG{o}{=}\PYG{l+s+s2}{\PYGZdq{}}\PYG{l+s+s2}{STDPGA}\PYG{l+s+s2}{\PYGZdq{}} \PYG{n}{units}\PYG{o}{=}\PYG{l+s+s2}{\PYGZdq{}}\PYG{l+s+s2}{ln(pctg)}\PYG{l+s+s2}{\PYGZdq{}} \PYG{o}{/}\PYG{o}{\PYGZgt{}}
\PYG{o}{\PYGZlt{}}\PYG{n}{grid\PYGZus{}field} \PYG{n}{index}\PYG{o}{=}\PYG{l+s+s2}{\PYGZdq{}}\PYG{l+s+s2}{10}\PYG{l+s+s2}{\PYGZdq{}} \PYG{n}{name}\PYG{o}{=}\PYG{l+s+s2}{\PYGZdq{}}\PYG{l+s+s2}{URAT}\PYG{l+s+s2}{\PYGZdq{}} \PYG{n}{units}\PYG{o}{=}\PYG{l+s+s2}{\PYGZdq{}}\PYG{l+s+s2}{\PYGZdq{}} \PYG{o}{/}\PYG{o}{\PYGZgt{}}
\PYG{o}{\PYGZlt{}}\PYG{n}{grid\PYGZus{}field} \PYG{n}{index}\PYG{o}{=}\PYG{l+s+s2}{\PYGZdq{}}\PYG{l+s+s2}{11}\PYG{l+s+s2}{\PYGZdq{}} \PYG{n}{name}\PYG{o}{=}\PYG{l+s+s2}{\PYGZdq{}}\PYG{l+s+s2}{SVEL}\PYG{l+s+s2}{\PYGZdq{}} \PYG{n}{units}\PYG{o}{=}\PYG{l+s+s2}{\PYGZdq{}}\PYG{l+s+s2}{ms}\PYG{l+s+s2}{\PYGZdq{}} \PYG{o}{/}\PYG{o}{\PYGZgt{}}
\end{Verbatim}
\end{quote}

Each tag specifies a column in the grid table that follows.
\begin{itemize}
\item {} 
\emph{index}:  The column number where the specified parameter may be found. The first column is column “1.”

\item {} 
\emph{name}:   Description of the parameter in the given column.

\item {} 
\emph{LON}:    Longitude of the grid location (the “site”).

\item {} 
\emph{LAT}:    Latitude of the site.

\item {} 
\emph{PGA}:    Peak ground acceleration at the site.

\item {} 
\emph{PGV}:    Peak ground velocity.

\item {} 
\emph{MMI}:    Seismic intensity.

\item {} 
\emph{PSA03}:  0.3 sec pseudo-spectral acceleration.

\item {} 
\emph{PSA10}:  1.0 sec pseudo-spectral acceleration.

\item {} 
\emph{PSA30}:  3.0 sec pseudo-spectral acceleration.

\item {} 
\emph{STDPGA}: The standard error of PGA at the site (in natural log units).

\item {} 
\emph{URAT}:   The uncertainty ratio. The ratio STDPGA to the nominal standard error of the GMPE at the site (no units).

\item {} 
\emph{SVEL}:   The 30-meter shear wave velocity (Vs30) at the site.

\end{itemize}

The measurement units:
\begin{itemize}
\item {} 
\emph{dd}:         Decimal degrees.

\item {} 
\emph{pctg}:       Percent-g (i.e., nominal Earth gravity).

\item {} 
\emph{cms}:        Centimeters per second.

\item {} 
\emph{intensity}:  Generally Modified Mercalli Intensity, but potentially other intensity measures.

\item {} 
\emph{ms}:         Meters per second.

\item {} 
\emph{ln(pctg)}:   Natural log of percent-g.

\item {} 
\emph{ln(cms)}:    Natural log of centimeters per second.

\end{itemize}

The number of \emph{grid\_field} tags will vary: smaller-magnitude earthquakes may not
have the pseudo-spectral acceleration values; scenarios will not have STDPGA or
URAT; and maps that have not been site corrected will not have SVEL.

The \emph{grid\_field} tags are followed by the \emph{grid\_data} tag, the gridded data, and the closing tags:
\begin{quote}

\begin{Verbatim}[commandchars=\\\{\}]
\PYGZlt{}grid\PYGZus{}data\PYGZgt{}
\PYGZhy{}119.7857 35.0463 4.3 4.21 5.26 5.76 5.76 1.09 0.5 1 800
\PYGZhy{}119.7774 35.0463 4.34 4.23 5.27 5.8 5.78 1.1 0.5 1 800
\PYGZhy{}119.7690 35.0463 4.37 4.25 5.27 5.84 5.81 1.1 0.5 1 800
…
\PYGZlt{}/grid\PYGZus{}data\PYGZgt{}
\PYGZlt{}/shakemap\PYGZus{}grid\PYGZgt{}
\end{Verbatim}
\end{quote}

The fast index for the coordinates is longitude, the slow index is latitude.
Dimensions are from upper left to lower right (i.e., from longitude
minimum/latitude maximum to longitude maximum/latitude minimum). The GMT program
\emph{xyz2grd} (coupled with \emph{gmtconvert}) is particularly useful for converting the
\emph{grid.xml} data into a usable grid file.

\textbf{Rock Grid XML}. The file \emph{rock\_grid.xml.zip} is a zipped XML file containing
the interpolated grid without site amplifications applied. The rock grid has the
same structure as \emph{grid.xml}, but Vs30 values and PGA uncertainty values are not
supplied. {\hyperref[tg_processing:amplify\string-ground\string-motions]{\crossref{\DUrole{std,std-ref}{Amplify Ground Motions}}}} in the {\hyperref[technical_guide:technical\string-guide]{\crossref{\DUrole{std,std-ref}{Technical Guide}}}}.

\textbf{Uncertainty Grid XML}. The file \emph{uncertainty.xml.zip} is a zipped XML file
containing the standard errors for each of the ground-motion parameters at each
point in the output grid. It has the same structure as \emph{grid.xml}, with the
additional \emph{grid\_field} names:
\begin{itemize}
\item {} 
\emph{STDPGA}:     Standard error of peak ground acceleration.

\item {} 
\emph{STDPGV}:     Standard error of peak ground velocity.

\item {} 
\emph{STDMMI}:     Standard error of seismic intensity.

\item {} 
\emph{STDPSA03}:   Standard error of 0.3 sec pseudo-spectral acceleration.

\item {} 
\emph{STDPSA10}:   Standard error of 1.0 sec pseudo-spectral acceleration.

\item {} 
\emph{STDPSA30}:   Standard error of 3.0 sec pseudo-spectral acceleration.

\end{itemize}

The standard errors are given in natural log units, except for intensity (linear
units). The PSA entries will be available only if the PSA ground motion
parameters were mapped (typically only for earthquakes of M \textgreater{}= 5.0). No
ground motion data or Vs30 values are available in
\emph{uncertainty.xml.zip}; for those, use \emph{grid.xml.zip}.

\textbf{Grid XYZ}. \emph{grid.xyz} is a plain-text comma-separated file of gridded ground motions.

\begin{notice}{note}{Note:}
The use of \emph{grid.xyz} is deprecated. It is difficult to maintain and have it remain backward-compatible. All users are urged to use the XML grids instead, and to switch to the XML grids if they are using \emph{grid.xyz}. \emph{grid.xyz} will disappear in a future ShakeMap release.
\end{notice}


\subsubsection{Station Lists}
\label{products:station-lists}
As discussed in the section {\hyperref[products:sec\string-input\string-files]{\crossref{\DUrole{std,std-ref}{Input Files}}}}, ShakeMap produces station lists of input data
in XML and text format. We also produce a version in GeoJSON format, which is available for
download, and is used by the website to plot the stations on the interactive maps. The station
data is available for viewing online by selecting the “Station List” tab on an event's ShakeMap
page. See \hyperref[products:napa-station-table]{Figure  \ref*{products:napa-station-table}} for an example.
\begin{figure}[htbp]\begin{flushleft}
\capstart

\includegraphics{{Napa_station_table}.png}
\caption{Station table view from ShakeMap event-specific webpages. Link is at right of tabs above the map (see \hyperref[products:napa-event-page]{Figure  \ref*{products:napa-event-page}}).}\label{products:napa-station-table}\label{products:id13}\end{flushleft}\end{figure}


\subsubsection{GIS Products}
\label{products:gis-products}
ShakeMap GIS Files (zipped) are a collection of shapefiles of polygons of the
ShakeMap model outputs for each shaking metric: MMI, PGA, PGV, and PSA at three
periods (0.3, 1.0, and 3.0 sec).  These file should be easily importable into a GIS
system. The ESRI Raster
Files (also zipped) are a collection of ESRI-formatted binary files.  It should
be relatively easy to convert these to (for example) ArcGIS grids using the
standard tools provided with the software. The contours are useful primarily for
overlaying with other data for visualization purposes.

The file base names in each archive are abbreviations of the
type of ground-motion parameter:
\begin{quote}

\begin{tabulary}{\linewidth}{|L|L|}
\hline

\emph{mi}
 & 
macroseismic intensity (usually, but not necessarily, mmi)
\\
\hline
\emph{pga}
 & 
peak ground acceleration
\\
\hline
\emph{pgv}
 & 
peak ground velocity
\\
\hline
\emph{psa03}
 & 
0.3 s pseudo-spectral acceleration
\\
\hline
\emph{psa10}
 & 
1.0 s pseudo-spectral acceleration
\\
\hline
\emph{psa30}
 & 
3.0 s pseudo-spectral acceleration
\\
\hline\end{tabulary}

\end{quote}

The sub-sections that follow describe available file and product types.


\paragraph{Shapefiles}
\label{products:shapefiles}
GIS shapefiles are comprised of four or five standard associated GIS files:
\begin{quote}

\begin{tabulary}{\linewidth}{|L|L|}
\hline

\emph{.dbf}
 & 
database file with layer attributes
\\
\hline
\emph{.shp}
 & 
the file with geographic coordinates
\\
\hline
\emph{.shx}
 & 
an index file
\\
\hline
\emph{.prj}
 & 
contains projection information
\\
\hline
\emph{.lyr}
 & 
contains presentation properties (only available for PGA, PGV, and MMI)
\\
\hline\end{tabulary}

\end{quote}

In this application, the shapefiles are contour polygons of the peak
ground-motion amplitudes in ArcView shapefiles. These contour polygons are
actually equal-valued donut-like polygons that sample the contour map at fine
enough intervals to accurately represent the surface function. We generate the
shapefiles independent of a GIS using a shareware package (\emph{shapelib.c}).
Contouring, as well as polygon formation and nesting, is performed by a program
written in the \emph{C} programming language by Bruce Worden, and is included in the ShakeMap
software distribution.

\textbf{GIS Shapefiles}. Contour polygons for the PGM parameters are
available as shapefiles intended for use with any GIS software that can
read ArcView shapefiles.  Note, however, that the peak ground velocity (PGV)
contours are in cm/s, and are therefore \textbf{not} suitable for HAZUS input.

The contour intervals are 0.04g for PGA and the three
PSA parameters, and 2cm/s for PGV. The file also includes MMI
contour polygons in intervals of 0.2 intensity units.  These shapefiles have
the same units as the online ShakeMaps. The archive of files is
compressed in zip format and called \emph{shape.zip}.  The \emph{shape.zip} file is
available for all events, but the spectral values are generally only included
for earthquakes of magnitude 4.0 and larger.
\phantomsection\label{products:hazus}
\textbf{HAZUS’99 Shapefiles and HAZUS-MH Geodatabases}. We generate shapefiles that
are designed with contour polygons intervals that are appropriate for use with
the Federal Emergency Management Agency’s (FEMA) \href{http://www.fema.gov/hazus/}{HAZUS-MH®} software, though they may be imported into any
GIS package that can read ArcView shapefiles. Because HAZUS software requires
PGV in inches/sec, this file is not suitable for all
applications. The contour intervals are 0.04g for PGA and the two PSA
parameters (HAZUS only uses 0.3 and 1.0 sec periods), and 4
inches/sec for PGV.

HAZUS’99 users can use the \emph{hazus.zip} shapefiles (see below) directly.  However,
the 2004 release of HAZUS-MH uses geodatabases, not shapefiles.  As of this
writing, FEMA has a temporary fix in the form of Visual Basic script that
imports ShakeMap shape files and exports geodatabases.  FEMA has plans to
incorporate such a tool directly into HAZUS-MH in the next official release (D.
Baush, FEMA, Region VIII, oral commun., 2015).

HAZUS traditionally used the epicenter and magnitude of an earthquake as
reported, and used empirical relationships to estimate ground motions over the
affected area.  These simplified ground-motion estimates would drive the computation of
losses to structures and infrastructure, estimates of casualties and displaced
households (for more details, see {\hyperref[references:nibs1997]{\crossref{\DUrole{std,std-ref}{NIBS, 1997}}}}).  With the
improvements to seismic systems nationally, particularly in digital
strong-motion data acquisition, and the advent of ShakeMap, HAZUS can now
directly import a much more accurate description of ground shaking.  The
improved accuracy of the input to loss-estimation routines can dramatically
reduce the uncertainty in loss estimation due to poorly constrained shaking
approximations.

The HAZUS GIS files are only generated for events that are larger than
(typically) magnitude 4.5.  The set of shapefiles for these parameters is an
archive of files compressed in
zip format (\emph{hazus.zip}) to facilitate file transfer.

\begin{notice}{note}{Note:}
An important note on the values of the parameters in the HAZUS shape files is that they are empirically corrected from the standard ShakeMap \textbf{peak ground motion values} to approximate the \textbf{geometric mean} values used for HAZUS loss estimation.  HAZUS was calibrated to work with mean ground motion values (FEMA, 1997). Peak amplitudes are corrected by scaling values down by 15 percent (Campbell, 1997; Joyner, oral commun., 2000). While more recent work by {\hyperref[references:beyer2006]{\crossref{\DUrole{std,std-ref}{Beyer and Bommer (2006)}}}} suggests different conversion factors, for the HAZUS shape files we continue to use 15 percent to maintain consistency in HAZUS results. As of this writing, FEMA is considering switching to peak ground motions as presented by ShakeMap rather than employing the geometric mean component.
\end{notice}


\paragraph{ESRI Raster Files (\emph{.fit} files)}
\label{products:esri-raster-files-fit-files}
ESRI raster grids of the ground-motion
parameters and their uncertainties are also available. The files are found in a
zipped archive called \emph{raster.zip}. Each archive contains four files per
parameter: \emph{\textless{}param\textgreater{}.fit} and \emph{\textless{}param\textgreater{}.hdr}, which contain the ground-motion
data, and \emph{\textless{}param\textgreater{}\_std.fit} and \emph{\textless{}param\textgreater{}\_std.hdr}, which contain the
uncertainties for the ground motions. See \emph{grid.xml} for information on units.
As with the other GIS files, PGA, PGV, and MMI are available for all events,
while the PSA parameters are usually only included for earthquakes
M4.5 and larger.
\setbox0\vbox{
\begin{minipage}{0.95\linewidth}
\textbf{Loading ESRI Raster Grid ShakeMaps into ArcGIS}

\medskip

\begin{enumerate}
\item {} 
Open the ArcToolbox in ArcMap

\item {} 
Select Multidimension Tools -\textgreater{} Make NetCDF Raster Layer

\item {} 
In the dialog that appears, select the input \emph{.grd} file you downloaded and unzipped, and name the layer    appropriately (``vs30'', etc.)

\item {} 
The new layer should appear in your list of layers.

\item {} 
Note: This layer is ephemeral---if you want to keep the raster version of the data, you'll have to save the layer to a file.

\end{enumerate}
\end{minipage}}
\begin{center}\setlength{\fboxsep}{5pt}\shadowbox{\box0}\end{center}


\paragraph{Google Earth Overlay}
\label{products:google-earth-overlay}
The file \emph{\textless{}event\_id\textgreater{}.kmz} enables the user to view the
ShakeMap in Google Earth (or other KML-compliant applications). A
color-scaled intensity overlay is provided along with a complete station list,
contours and polygons of intensity and peak ground motion, a fault representation (if
provided), epicenter indicator, intensity scale, and the USGS logo. The
transparency of the intensity overlay is adjustable by the user, as is the
appearance of seismic stations. The KMZ file embeds several
other files that may be found in the event’s download directory:
\begin{quote}

\begin{Verbatim}[commandchars=\\\{\}]
\PYG{n}{epicenter}\PYG{o}{.}\PYG{n}{kmz}
\PYG{n}{fault}\PYG{o}{.}\PYG{n}{kmz}
\PYG{n}{overlay}\PYG{o}{.}\PYG{n}{kmz} \PYG{p}{(}\PYG{n}{links} \PYG{n}{to} \PYG{n}{ii\PYGZus{}overlay}\PYG{o}{.}\PYG{n}{png}\PYG{p}{)}
\PYG{n}{stations}\PYG{o}{.}\PYG{n}{kmz}
\PYG{n}{contours}\PYG{o}{.}\PYG{n}{kmz}
\end{Verbatim}
\end{quote}

Note that the KMZ file is static and will not automatically update when we update the ShakeMap
for an event, so periodic checks for updated maps and reloading of the KMZ is
recommended.

In addition to the ShakeMap-produced KMZ file, the USGS produces a KML file
(linked near the top of the page in the event-centric pages with the title
“Google Earth KML”) which contains not only ShakeMap data, but also data from
PAGER, DYFI, and other sources. This file should be the preferred
source, as it will have the most-up-to-date links, though it does not have all of
the layers available in the ShakeMap KMZ file.


\paragraph{Contour Files}
\label{products:contour-files}
As mentioned above in the ShakeMap Output GIS format section,
contour files are available for general GIS, HAZUS, and KML formats. We also
provide GeoJSON format contours, all under the ShakeMap event-specific
``Downloads'' tab.


\subsection{Real-Time Product Distribution, Automatic Access, and Feeds}
\label{products:real-time-product-distribution-automatic-access-and-feeds}
ShakeMap products are distributed by a number of means immediately after they
are produced. The intent of these products is to help responders and
other responsible parties effectively manage their post-earthquake
activities, so we make it as easy as possible for users with a variety of
technological sophistication and infrastructure to access them. The general
distribution methods are
interactive Web downloads, RSS feeds, GeoJSON feeds, ShakeCast, the Product
Distribution Layer (PDL) client, and GIS web mapping services.


\subsubsection{Interactive Web Downloads}
\label{products:interactive-web-downloads}
The easiest way to obtain ShakeMap products immediately following an earthquake
is from the \href{http://earthquake.usgs.gov/earthquakes/shakemap/}{ShakeMap} or
\href{http://earthquake.usgs.gov/}{USGS Earthquake Program} webpages. The event
page for any given earthquake has a download link where all of the products for
that event may be found. The ShakeMap page for an event also has a download link
that lists just the ShakeMap products. The variety
of formats for ShakeMap are described in the previous section.


\subsubsection{RSS Feeds}
\label{products:rss-feeds}
USGS Earthquake Program earthquake information \href{http://earthquake.usgs.gov/earthquakes/feed/v1.0/}{Feeds} currently include Really
Simple Syndication (RSS) feeds. However, the RSS feeds are deprecated; they will be
decommissioned in 2016.


\subsubsection{GeoJSON Feeds}
\label{products:geojson-feeds}
\textbf{Automatically Retrieving Earthquake Data and ShakeMap Files}. The USGS
Earthquake Program GeoJSON feed provides USGS ShakeMap, along with most other USGS
real-time earthquake products. \href{http://geojson.org/}{GeoJSON} is an extension
of the JavaScript Object Notation (JSON) standard and allows for a
variety of geospatial data structures.  There are JSON parsers in most modern
languages, including Python, Perl, Matlab, and R.

In order to automatically ingest the above data, use the automated
\href{http://earthquake.usgs.gov/earthquakes/feed/v1.0/geojson.php}{GeoJSON feeds}.
Mike Hearne (USGS), provides \href{https://gist.github.com/mhearne-usgs/6b040c0b423b7d03f4b9}{an example python script} for querying the USGS
Magnitude 2.5+ thirty-day GeoJSON feed, and downloading the most recent version of
the event products desired by the user. In addition, the USGS Haz-Dev group provides
\href{https://github.com/usgs/devcorner}{other scripts} in various programming languages
that allow access to the GeoJSON feeds. Modifications to these scripts allow
access to any ShakeMap (or other) products automatically, GIS flavors included.

\textbf{Example}. \emph{How can I use your API to get ShakeMap files download for specific events (that shook Guatemala)?}

The following GeoJSON summary query includes events between 2015-01-01
and 2016-01-01 in the bounding box lat. 10 to 20, long. -95 to -85,
in case an event outside Guatemala results in shaking inside
Guatemala; and includes a ShakeMap product:
\begin{quote}

\begin{Verbatim}[commandchars=\\\{\}]
http://earthquake.usgs.gov/fdsnws/event/1/query?format=geojson\PYGZam{}
starttime=2015\PYGZhy{}01\PYGZhy{}01T00:00:00\PYGZam{}maxlatitude=20\PYGZam{}minlatitude=10\PYGZam{}maxlongitude=\PYGZhy{}85\PYGZam{}
minlongitude=\PYGZhy{}95\PYGZam{}endtime=2016\PYGZhy{}01\PYGZhy{}01T00:00:00\PYGZam{}producttype=shakemap
\end{Verbatim}
\end{quote}

The results include an array of features with summary information for
each event.  The \emph{detail} property is a URL for the GeoJSON detail
feed that includes URLs for ShakeMap files. For example, for the
\emph{us100044xh} event, the GeoJSON detailed feed URL is:
\begin{quote}

\begin{Verbatim}[commandchars=\\\{\}]
HTTP://earthquake.usgs.gov/fdsnws/event/1/query?eventid=us100044xh\PYGZam{}format=geojson
\end{Verbatim}
\end{quote}

The URLs for the ShakeMap files can be found inside the feed:
\begin{quote}

\begin{Verbatim}[commandchars=\\\{\}]
\PYG{n}{FEED}\PYG{o}{.}\PYG{n}{properties}\PYG{o}{.}\PYG{n}{products}\PYG{o}{.}\PYG{n}{shakemap}\PYG{p}{[}\PYG{l+m+mi}{0}\PYG{p}{]}\PYG{o}{.}\PYG{n}{contents}\PYG{p}{[}\PYG{l+s+s1}{\PYGZsq{}}\PYG{l+s+s1}{download/grid.xml.zip}\PYG{l+s+s1}{\PYGZsq{}}\PYG{p}{]}\PYG{o}{.}\PYG{n}{url}
\PYG{n}{FEED}\PYG{o}{.}\PYG{n}{properties}\PYG{o}{.}\PYG{n}{products}\PYG{o}{.}\PYG{n}{shakemap}\PYG{p}{[}\PYG{l+m+mi}{0}\PYG{p}{]}\PYG{o}{.}\PYG{n}{contents}\PYG{p}{[}\PYG{l+s+s1}{\PYGZsq{}}\PYG{l+s+s1}{download/shape.zip}\PYG{l+s+s1}{\PYGZsq{}}\PYG{p}{]}\PYG{o}{.}\PYG{n}{url}
\end{Verbatim}
\end{quote}

In this case, these are the specific URLs are for the \emph{grid.xml} file
and for the \emph{shape.zip} files, respectively:
\begin{quote}

\begin{Verbatim}[commandchars=\\\{\}]
\PYG{n}{http}\PYG{p}{:}\PYG{o}{/}\PYG{o}{/}\PYG{n}{earthquake}\PYG{o}{.}\PYG{n}{usgs}\PYG{o}{.}\PYG{n}{gov}\PYG{o}{/}\PYG{n}{archive}\PYG{o}{/}\PYG{n}{product}\PYG{o}{/}\PYG{n}{shakemap}\PYG{o}{/}\PYG{n}{us100044xh}\PYG{o}{/}\PYG{n}{us}\PYG{o}{/}\PYG{l+m+mi}{1450404175265}\PYG{o}{/}
\PYG{n}{download}\PYG{o}{/}\PYG{n}{grid}\PYG{o}{.}\PYG{n}{xml}\PYG{o}{.}\PYG{n}{zip}
\PYG{n}{http}\PYG{p}{:}\PYG{o}{/}\PYG{o}{/}\PYG{n}{earthquake}\PYG{o}{.}\PYG{n}{usgs}\PYG{o}{.}\PYG{n}{gov}\PYG{o}{/}\PYG{n}{archive}\PYG{o}{/}\PYG{n}{product}\PYG{o}{/}\PYG{n}{shakemap}\PYG{o}{/}\PYG{n}{us100044xh}\PYG{o}{/}\PYG{n}{us}\PYG{o}{/}\PYG{l+m+mi}{1450404175265}\PYG{o}{/}
 \PYG{n}{download}\PYG{o}{/}\PYG{n}{shape}\PYG{o}{.}\PYG{n}{zip}
\end{Verbatim}
\end{quote}


\subsubsection{Additional Feeds}
\label{products:additional-feeds}
More information about USGS earthquake data feeds is available at our \href{http://earthquake.usgs.gov/earthquakes/feed/v1.0/index.php}{Feeds \&
Notifications page}.


\subsubsection{ShakeCast System}
\label{products:shakecast-system}
ShakeCast delivers user-specified ShakeMap products to the user’s
local or virtual system(s), and runs fragility-based damage (or
inspection priority) calculations for specific portfolios. More advanced
features of ShakeCast include a complete suite of damage
estimation and mapping tools, coupled with sophisticated tools to notify
responsible parties within an organization on a per-facility basis. See
{\hyperref[related_systems:sec\string-related\string-systems]{\crossref{\DUrole{std,std-ref}{Related Systems}}}} for more details. Complete background on ShakeCast
can be found on the ShakeCast \href{http://earthquake.usgs.gov/research/software/shakecast/}{homepage} and \href{https://my.usgs.gov/confluence/display/ShakeCast/Home}{Wiki} and the documentation provided therein.


\subsubsection{Product Delivery Layer (PDL) Client}
\label{products:product-delivery-layer-pdl-client}
Finally, for academic and government users, ShakeMap products (and other
earthquake products) are communicated through the USGS’s \href{http://earthquake.usgs.gov/research/software/\#PDL}{Product Distribution
Layer (PDL)}.


\subsubsection{Web Mapping (GIS) Services}
\label{products:web-mapping-gis-services}\label{products:gis-services}
In addition to the downloadable GIS-formatted ShakeMaps (including shapefiles) that are
readily available for each ShakeMap event, USGS also hosts a real-time \href{http://earthquake.usgs.gov/arcgis/rest/services/eq/sm\_ShakeMap30DaySignificant/MapServer}{30-day
Signficant {}`Earthquake GIS ShakeMap Feed}.
\href{http://www.esri.com}{ESRI} provides a separate \href{http://www.esri.com/}{Disaster Response ArcGIS service}, providing \href{https://tmservices1.esri.com/arcgis/rest/services/LiveFeeds/USGS\_Seismic\_Data/MapServer}{live feeds}
to several USGS post-earthquake products.
\setbox0\vbox{
\begin{minipage}{0.95\linewidth}
\textbf{Related GIS Service Interactions}

\medskip


Users can access the ShakeMap data behind the GIS service in a variety of ways via the ArcGIS Server “REST API”. Some examples of commonly used data-access options are detailed below.
\end{minipage}}
\begin{center}\setlength{\fboxsep}{5pt}\shadowbox{\box0}\end{center}
\begin{itemize}
\item {} 
\href{http://resources.arcgis.com/en/help/rest/apiref/export.html}{Export Map Image}: Download a static image of the map to include in their work.

\item {} 
\href{http://resources.arcgis.com/en/help/rest/apiref/identify.html}{Identify}: Retrieve service data for given geographic location. (Point, Line, Polygon or Envelop)

\item {} 
\href{http://resources.arcgis.com/en/help/rest/apiref/find.html}{Find}: Query service data that contains certain attributes. (ex. ShakeMap data for distinct event id)

\item {} 
\href{http://resources.arcgis.com/en/help/rest/apiref/query.html}{Query}: Query a specific layer in a service and return a detailed featureset.

\end{itemize}

Along with the common GIS service interactions listed above, there are many
other calls that GIS developers can make through the \href{http://resources.arcgis.com/en/help/rest/apiref/}{REST API}.

\begin{notice}{note}{Note:}
\textbf{Earthquake Significance}. The NEIC associates a \href{https://github.com/usgs/earthquake-event-ws/blob/master/src/lib/sql/fdsnws/getEventSummary.sql\#L157}{*significance*} number with each earthquake event. Larger numbers indicate more significance. This value is determined by a number of factors, including magnitude, maximum MMI, felt reports, and estimated impact.  The significance number ranges from 0 to 1000.  The ``30 day significant earthquake feed'' that determines which events are included in the ShakeMap GIS feed uses events with a significance of 600 and greater.
\end{notice}

\textbf{Accessing ShakeMap GIS Files:} While this GIS service only provides access to
significant earthquakes that have occurred within the last thirty days, users can
download GIS files for \href{https://tmservices1.esri.com/arcgis/rest/services/LiveFeeds/USGS\_Seismic\_Data/MapServer}{significant events}
on our website after the thirty-day period.  The significant earthquake archive has
a list of large events with links to each event’s webpage.  From the event
page, users can click on the ShakeMap tab and navigate to the “Downloads”
section to get a zipped bundle of shapefiles.

\textbf{Acknowledgement}: USGS appreciates guidance from the Esri Aggregated Live Feed
team, specifically Derrick Burke and Paul Dodd.  Their willingness to share
best practices for robust real-time sharing of GIS data enabled this project to
be completed.


\section{ShakeMap Archives}
\label{shakemap_archives::doc}\label{shakemap_archives:sec-shakemap-archives}\label{shakemap_archives:shakemap-archives}
All ShakeMaps are available for viewing and download online. The ShakeMap
Archives consist of three primary repositories: \textbf{Recent ShakeMaps}, the
\textbf{ShakeMap Atlas} for historic earthquakes (primarily 1970-2012), and a
collection of hypothetical earthquake \textbf{ShakeMap Scenarios}. For example,
scenario earthquakes compiled for Northern and Southern California represent
over 200 different earthquake ruptures studied for California, as detailed
below. Formats for all ShakeMaps, whether near--real-time, historic, or
future scenarios, are uniform.

\begin{notice}{note}{Note:}
Some older archival ShakeMaps
online were generated with earlier versions of the ShakeMap
software; hence, they do not contain up-to-date formats and all
products. This will be remedied as older events are rerun and
updated. One can tell from the \emph{processed time} on the bottom of
any ShakeMap when it was run.
\end{notice}


\subsection{Real-time ShakeMaps}
\label{shakemap_archives:real-time-shakemaps}
\textbf{In the U.S.}, ShakeMaps are generated via independent systems running at ANSS
Regional Seismic Systems (RSNs) in Northern California, Southern California, the
Pacific Northwest, Utah, Nevada, and Alaska. For the rest of the U.S., the
ShakeMap group at the USGS National Earthquake Information Center (NEIC) in Golden, Colorado
produces maps for the regional networks operating in Hawaii, New England, and
the Central and Eastern U.S. on a system referred to as Global ShakeMap (GSM).
The input, metadata, and output files produced by all these instances are
aggregated by the USGS via the Earthquake Hazards Program web system. GSM also provides
backup capabilities for the RSNs, but with degraded capabilities; not all data
are flowing from the RSNs to GSM automatically.

Separate independent systems running in Puerto Rico and New England generate
ShakeMaps, but these instances do not deliver them through the USGS Earthquake Hazards
Program webpages (at the time of this writing). GSM covers these regions but
does not yet access the full set of data available to these regional
systems. For more information on the ANSS regional and the national
ShakeMap system implementations and operations, see the section on {\hyperref[regionals:sec\string-regionals]{\crossref{\DUrole{std,std-ref}{Regional Operations}}}}.

\textbf{Internationally}, USGS ShakeMap is installed and operational in Italy,
France, Portugal, Switzerland, Romania, Indonesia, Iran, Iceland,
Panama, and several other nations (see \hyperref[shakemap_archives:international-shakemaps]{Figure  \ref*{shakemap_archives:international-shakemaps}}).
Several instances of ShakeMap are in testing
or operational mode in the Middle East (including Oman, Morocco, and the U.A.E.; M.
Franke, written comm., 2015). In addition, other ShakeMap installations are in
testing in Latin America and the Caribbean (Chile, Costa Rica, Colombia, Mexico,
Costa Rica), and in Southeast Asia (Malaysia and Korea). Discussions have taken
place with several other interested countries.

\begin{notice}{note}{Note:}
Very impressive systems analogous to ShakeMap operate in
Japan (JMA), Taiwan, China, New Zealand, Turkey, and several other countries.
\end{notice}
\begin{figure}[htbp]\begin{flushleft}
\capstart

\includegraphics{{International_shakemaps}.png}
\caption{International ShakeMap Systems}\label{shakemap_archives:international-shakemaps}\label{shakemap_archives:id4}\end{flushleft}\end{figure}


\subsection{ShakeMap Atlas}
\label{shakemap_archives:shakemap-atlas}
ShakeMap was designed with near--real-time earthquake response purposes in mind.
However, many of the strategies for mapping the patterns of peak ground motions for
real-time applications described above prove useful for re-creating the shaking from
historic earthquakes.

The ShakeMap Atlas ({\hyperref[references:allen2008]{\crossref{\DUrole{std,std-ref}{Allen et al., 2008}}}}, {\hyperref[references:allen2009a]{\crossref{\DUrole{std,std-ref}{2009a}}}}) is a self-consistent, well-calibrated
collection of historic earthquakes for which ShakeMaps were systematically generated.
The Atlas constitutes an invaluable online resource for investigating near-source strong
ground motion, as well as for seismic hazard, scenario, risk, and loss-model
development.
\setbox0\vbox{
\begin{minipage}{0.95\linewidth}
\textbf{\textbf{Finding Atlas ShakeMaps Online}}

\medskip

\begin{itemize}
\item {} 
\textbf{Atlas Version 1.0} ({\hyperref[references:allen2008]{\crossref{\DUrole{std,std-ref}{Allen et al., 2008}}}}) ShakeMaps are available online on the
\href{http://earthquake.usgs.gov/earthquakes/shakemap/}{ShakeMap homepage}, which consists of all the standardized ShakeMap products and formats. Output grids for the entire dataset can also
be obtained at that site.

\item {} 
\textbf{Atlas Version 2.0} ({\hyperref[references:garcia2012a]{\crossref{\DUrole{std,std-ref}{Garcia et al. (2012a)}}}}
ShakeMaps are available by searching the USGS \href{http://earthquake.usgs.gov/earthquakes/search/}{Comprehensive Catalogue
(ComCat) Earthquake database}. Be sure to select “ShakeMap Atlas” as the “Contributor” from
the “Advanced Options” dropdown menu.

\end{itemize}
\end{minipage}}
\begin{center}\setlength{\fboxsep}{5pt}\shadowbox{\box0}\end{center}

The original (2009) Atlas is a compilation of nearly 5,000 ShakeMaps for the most
significant global earthquakes between 1973 and 2007 ({\hyperref[references:allen2008]{\crossref{\DUrole{std,std-ref}{Allen et al., 2008}}}}).
{\hyperref[references:garcia2012a]{\crossref{\DUrole{std,std-ref}{Garcia et al. (2012a)}}}} introduced an update of the Atlas, which extends the time period through 2012,
with a total of 6,100 events. The revised Atlas 2.0 includes: a new version of the
ShakeMap software (V3.5) which improves interpolation and uncertainty estimations;
an updated earthquake source catalogue that includes regional locations and finite fault
models; a refined strategy to select prediction and conversion equations based on a
new seismotectonic regionalization scheme ({\hyperref[references:garcia2012b]{\crossref{\DUrole{std,std-ref}{Garcia et al., 2012b}}}}); and vastly more
macroseismic-intensity and ground-motion data from international agencies.

In order to best replicate shaking that occurred during historic and recent earthquakes, we
employ many of the ShakeMap tools described in the previous sections. For many older
events, the important constraints (in addition to the usual site condition map) are the
fault rupture geometry, macroseismic intensity, and peak ground motion data. As
previously described, combining peak ground motions and macroseismic data was accomplished seamlessly
and rigorously with the new interpolation scheme developed by
{\hyperref[references:worden2010]{\crossref{\DUrole{std,std-ref}{Worden et al. (2010)}}}}. This strategy was in part aimed at most accurately representing
historic earthquake shaking maps, which are often constrained predominantly by key
macroseismic observations, and is essential for the Atlas.
\begin{figure}[htbp]\begin{flushleft}
\capstart

\includegraphics{{Figure_1_14}.png}
\caption{Example of the macroseismic intensity ShakeMaps for one ShakeMap Atlas event:
the 1999 M6.0 Athens, Greece earthquake. (A) overview map; and (B) zoomed map. The black
rectangle delineates the surface projection of the finite fault (a normal fault dipping southwest).
Circles represent native MMI data; triangles show PGM data converted to MMI values via the {\hyperref[references:worden2012]{\crossref{\DUrole{std,std-ref}{Worden et al.
(2012)}}}} GMICE, the choice of which automatically redefines the legend scale.
After {\hyperref[references:garcia2012a]{\crossref{\DUrole{std,std-ref}{Garcia et al. (2012a)}}}}.}\label{shakemap_archives:figure1-14}\label{shakemap_archives:id5}\end{flushleft}\end{figure}

The Atlas provides a hazard base layer for an number of systems that require estimates of the shaking values where losses occurred.
To this end, the Atlas is used for the Earthquake Consequences Database within the Global Earthquake
Model initiative (GEMECD; {\hyperref[references:so2014]{\crossref{\DUrole{std,std-ref}{So, 2014}}}}).
The ``GEMECD subset'' is a collection of approximately 100 events which constitute
the most important and damaging
events since about 1973. The purpose of the GEMECD subset is to provide the
Global
Earthquake Model (GEM) Foundation---and hence the wider earthquake hazard and
loss community---a common-denominator hazard layer
for calibrating and testing earthquake damage and loss models. The Atlas is also
the calibration hazard layer for the USGS
\href{http://earthquake.usgs.gov/research/pager/}{PAGER} system  (e.g., {\hyperref[references:wald2008]{\crossref{\DUrole{std,std-ref}{Wald et
al., 2008}}}}; {\hyperref[references:jaiswal2010]{\crossref{\DUrole{std,std-ref}{Jaiswal and Wald, 2010}}}}; {\hyperref[references:pomonis2011]{\crossref{\DUrole{std,std-ref}{Pomonis and So, 2011}}}}).

A subset of the Atlas was also employed by {\hyperref[references:zhu2014]{\crossref{\DUrole{std,std-ref}{Zhu et al. (2014)}}}}
for the calibration of near--real-time
liquefaction probability maps, and by {\hyperref[references:nowicki2014]{\crossref{\DUrole{std,std-ref}{Nowicki et al. (2014)}}}} for near--real-time
landslide mapping. As with earlier studies (including {\hyperref[references:godt2008]{\crossref{\DUrole{std,std-ref}{Godt et al., 2008}}}}; {\hyperref[references:jaiswal2010]{\crossref{\DUrole{std,std-ref}{Jaiswal et al.,
2010}}}}, {\hyperref[references:jaiswal2012]{\crossref{\DUrole{std,std-ref}{2012}}}}; {\hyperref[references:knudsen2011]{\crossref{\DUrole{std,std-ref}{Knudsen and Bott, 2011}}}}; {\hyperref[references:matsuoka2015]{\crossref{\DUrole{std,std-ref}{Matsuoka et al, 2015}}}}), these authors recognized the
importance of calibrating empirical ground failure and loss models against a
standardized
set of uniformly-produced shaking hazard maps so as to allow comparison of
models
based on consistent hazard inputs. \hyperref[shakemap_archives:figure1-15]{Figure  \ref*{shakemap_archives:figure1-15}} shows an example of
the possibility of
constraining shaking at landslide sites using ShakeMap layers for the 2008 M7.9
Wenchuan, China earthquake, employing shaking constraints provided by strong-motion
and intensity data as well as detailed fault geometry.
\begin{figure}[htbp]\begin{flushleft}
\capstart

\includegraphics{{Figure_1_15}.png}
\caption{Example of the ShakeMaps for the 2008 M 7.9 Wenchuan, China earthquake for (A)
Intensity and (B) PGA. Green polygons show areas of landsliding mapped out by {\hyperref[references:dai2010]{\crossref{\DUrole{std,std-ref}{Dai et al.
(2010)}}}}. Black rectangles delineate the surface projection of the different fault segments involved
in the rupture. Triangles indicate native strong motion stations; circles represent MMI data converted to GM
values via a GMICE (here {\hyperref[references:worden2012]{\crossref{\DUrole{std,std-ref}{Worden et al., (2012)}}}}, the choice of which automatically redefines the
legend scale.}\label{shakemap_archives:figure1-15}\label{shakemap_archives:id6}\end{flushleft}\end{figure}


\subsection{ShakeMap Scenarios}
\label{shakemap_archives:shakemap-scenarios}\label{shakemap_archives:sec-scenarios}
In addition to historical and near--real-time applications, ShakeMap has become widely
used for earthquake mitigation and planning exercises through earthquake scenarios.
A scenario represents one realization of a potential future earthquake by assuming a
particular magnitude, location, and fault-rupture geometry and estimating shaking using a
variety of strategies (including ShakeMap with GMPEs).
Some of the technical issues related to scenario generation are discussed in the {\hyperref[technical_guide:technical\string-guide]{\crossref{\DUrole{std,std-ref}{Technical Guide}}}}.
Here we cover the many uses for earthquake scenarios from the users' perspective.

In planning and coordinating emergency response, utilities, local government, and other
organizations are best served by conducting training exercises based on realistic
earthquake situations---ones similar to those they are most likely to face. ShakeMap
Scenario earthquakes can fill this role. They can also be used to examine
exposure of structures, lifelines, utilities, and transportation corridors to specified
potential earthquakes.

The September, 2015, \href{http://nehrp.gov/pdf/2015ACEHRReportFinal.pdf}{Report to NEHRP Agencies from the Advisory Committee on
Earthquake Hazards Reduction (ACEHR)},
notes:
\begin{quote}

\emph{USGS Recommendation 4 - ACEHR recommends the USGS expand earthquake scenario
development in conjunction with stakeholder engagement in order to examine
consequences of earthquakes in high-risk urban areas.}
\end{quote}

To this end, USGS ShakeMap webpages now display many earthquake scenarios, and
we are working to develop a comprehensive suite of scenarios for all at-risk
regions of the United States (see {\hyperref[references:thompson2016]{\crossref{\DUrole{std,std-ref}{Thompson et al., 2016}}}}).
\begin{quote}

\emph{USGS Recommendation 5 - ACEHR recommends the USGS work with operators of
critical infrastructure and lifeline systems to define and integrate
near real-time earthquake data and other seismic information into
system monitoring so that operators can quickly assess system
impacts from earthquake movements
and take appropriate actions.  This development should be linked
with the EEW program.}
\end{quote}

A ShakeMap earthquake scenario is simply a ShakeMap with an assumed magnitude and
location, and, optionally, specified fault geometry. For example, \hyperref[shakemap_archives:figure1-13u]{Figure  \ref*{shakemap_archives:figure1-13u}} shows
ShakeMap scenario intensity (top) and peak velocity (bottom) maps for a hypothetical
earthquake of M7.05 on the Hayward Fault in the eastern San Francisco Bay area. Due to
the proximity to populated regions of Oakland, Berkeley, and surrounding cities, this
scenario represents one of the most destructive earthquakes that could impact the region.
Different renditions of this particular scenario have been widely used for evaluating the
region's capacity to respond to such a disaster among federal, state, utility, business, and
local emergency response organizations.
\begin{figure}[htbp]\begin{flushleft}
\capstart

\includegraphics{{Figure_1_13}.png}
\caption{ShakeMap scenario intensity (top) and peak velocity (bottom) maps for a M7.05
Hayward Fault, CA, earthquake: A) intensity; no directivity, B) intensity; directivity added, C)
peak velocity; no directivity, and D) peak velocity; directivity added.}\label{shakemap_archives:figure1-13u}\label{shakemap_archives:id7}\end{flushleft}\end{figure}

The USGS and ANSS partners receive numerous requests for ShakeMap scenarios
annually. The NEIC Global ShakeMap (GSM) operators have also generated scores
of scenarios for colleagues, partners, other federal agencies, non-profit organizations,
and governments around the globe. These and other scenarios are available online
on the ShakeMap webpages. They are formatted the same as other ShakeMaps, so they
can be easily used in response planning and loss estimation as well as for educational
purposes.

ShakeMap earthquake scenarios can be an integral part of earthquake emergency
response planning.
Primary users include city, county, state and
federal government agencies (e.g., the California EMA, FEMA); and
emergency-response planners and managers for utilities, businesses, and other
large organizations.
ShakeMap scenarios are particularly useful in planning and
exercises when combined with loss-estimation systems such as PAGER, HAZUS, and
ShakeCast, which provide ShakeMap-based estimates of overall social and economic
impact, detailed loss estimates, and inspection priorities, respectively. Since
ShakeMap’s inception, operators have generated hundreds of earthquake
scenarios that have been used in formal earthquake response exercises around the
world.
\setbox0\vbox{
\begin{minipage}{0.95\linewidth}
\textbf{\textbf{Finding ShakeMap Scenarios Online}}

\medskip

\begin{itemize}
\item {} 
\textbf{Scenarios 1.0}. ShakeMaps are available online on the
\href{http://earthquake.usgs.gov/earthquakes/shakemap/}{ShakeMap homepage}, which
consists of all the standardized ShakeMap products and formats.
Output grids for the entire dataset can also be obtained at that site.

\item {} 
\textbf{Scenarios 2.0}. The Next Generation Scenarios (NGS) will be available by
searching the USGS \href{http://earthquake.usgs.gov/earthquakes/search/}{Comprehensive Catalogue
(ComCat) Earthquake database}. Be sure
to select “ShakeMap Scenarios”
as the “Contributor” in the “Advanced Options” dropdown menu. The
available catalogues of scenarios will change over time.

\end{itemize}
\end{minipage}}
\begin{center}\setlength{\fboxsep}{5pt}\shadowbox{\box0}\end{center}


\subsubsection{Generating Earthquake Scenarios}
\label{shakemap_archives:generating-earthquake-scenarios}
Given a selected event, we have developed tools to make it relatively easy to generate a
ShakeMap earthquake scenario. All that is required is to assume a particular fault or fault
segment will (or did) rupture over a certain length and with a chosen magnitude, and to
generate a file describing the fault geometry and another describing the magnitude and
hypocenter of the ostensible earthquake (see the {\hyperref[software_guide:software\string-guide]{\crossref{\DUrole{std,std-ref}{Software \& Implementation Guide}}}} for details). ShakeMap
can then estimate the ground shaking at all locations over a chosen area surrounding the
fault and produce a full suite of data products just as if the event were a real earthquake.
Ground motions are usually estimated using GMPEs to compute peak ground motions on
rock conditions; however, the operator may also supply ground-motion estimates from
external programs in the form of GMT grid files. As described in {\hyperref[tg_processing:amplify\string-ground\string-motions]{\crossref{\DUrole{std,std-ref}{Amplify Ground Motions}}}},
ShakeMap corrects the amplitudes based on the local site soil conditions unless
configured otherwise.

At present, ground motions are estimated using empirical attenuation
relationships (though we can use gridded ground-motion estimates from other
sources for those who wish to provide them). We then correct the amplitudes
based on the local site soil conditions (Vs30) as we do in the general ShakeMap
interpolation scheme.  Fault finiteness is included explicitly, basin depth can
be incorporated where appropriate, and source directivity is included via the
relationships developed by {\hyperref[references:rowshandel2010]{\crossref{\DUrole{std,std-ref}{Rowshandel (2010)}}}}.  Depending on the level of
complexity needed for the scenario, event-specific factors, such as variable
slip distribution, could also be incorporated in the amplitude estimates fed to
ShakeMap.

In most cases, we do not consider the direction of rupture, nor do we modify the peak
motions by a directivity term. Fault geometries are specified with a fault file that
represents the fault planar segments. With this approach, the location of
the earthquake hypocenter does not have any effect on the resulting ground-motions; only
the location and dimensions of the fault matter. If we were to add directivity to the
calculations, then different choices of hypocentral location could result in significantly
different motions for the same magnitude earthquake and fault segment.

Rather, our approach is to generally show the average effect because it is difficult to justify a
particular choice of hypocenter or to show the results for every possible hypocentral
location. Our empirical predictive approach also only gives median peak--ground-motion
values, so it does not account for all the expected variability in motions, only the
aforementioned site amplification variations. Actual ground motions show significant
variability for a given distance, magnitude, and site condition and, hence, the scenario
ground-motions are more uniform than would be expected for a real earthquake.
2D and 3D wave propagation, path effects (such as
basin edge amplification and focusing), differences in motions among earthquakes of the
same magnitude, and complex site effects are not accounted for with our methodology.
For scenarios in which we wish to explore directivity explicitly, ShakeMap includes a
tool based on {\hyperref[references:rowshandel2010]{\crossref{\DUrole{std,std-ref}{Rowshandel (2010)}}}} as shown in
\hyperref[shakemap_archives:figure1-13u]{Figure  \ref*{shakemap_archives:figure1-13u}} and described in {\hyperref[tg_processing:sec\string-directivity]{\crossref{\DUrole{std,std-ref}{Directivity}}}}. We
are also exploring delivery of scenarios with multiple realizations of
spatial variability (see {\hyperref[future_directions:future\string-directions]{\crossref{\DUrole{std,std-ref}{Future Directions}}}} and {\hyperref[references:verros2016]{\crossref{\DUrole{std,std-ref}{Verros et al. (2016)}}}}.

In terms of generating scenarios with the ShakeMap system, a number of specific
considerations and a number of configuration changes are made for scenario events as
opposed to actual events triggered by the network.  For example, after generating a
scenario for a major but hypothetical event, obviously one does not want to automatically
deliver the files to customers who are expecting real events.  To avoid these sorts of
errors, the \emph{Event ID*s for all scenarios are tagged with the suffix *\_se}. Such events are
recognized by the processing and delivery software, which is configured to handle the
scenarios as special cases. Scenarios are also given their own separate space on the
webpages. The scenario earthquake ground-motion maps are identical to those made for real earthquakes, with one exception: ShakeMap scenarios are labeled with the word “SCENARIO” prominently displayed to avoid potential confusion with real earthquake occurrences.

See the {\hyperref[software_guide:software\string-guide]{\crossref{\DUrole{std,std-ref}{Software \& Implementation Guide}}}} for additional information on generating earthquake scenarios.


\subsubsection{Standardizing Earthquake Scenarios}
\label{shakemap_archives:standardizing-earthquake-scenarios}
The USGS has evaluated the probabilistic hazard from active faults in
the U.S. for the \href{http://earthquake.usgs.gov/hazards/}{National Seismic Hazard Mapping Project}.
From these maps it is
possible to prioritize the best scenario earthquakes to be used in planning exercises by
considering the most likely candidate earthquake fault first, followed by the next likely,
and so on. Such an analysis is easily accomplished by hazard disaggregation, in which the
contributions of individual earthquakes to the total seismic hazard, their probability of
occurrence, and the severity of the ground-motions are ranked.  Using the individual
disaggregated components of these hazard maps, a user can select the appropriate
scenarios given their location, regional extent, and specific planning requirements.

ShakeMap operators are in the process (early 2016; see {\hyperref[references:thompson2016]{\crossref{\DUrole{std,std-ref}{Thompson et al., 2016}}}}) of developing a full suite of
scenario ShakeMaps from the disaggregated U.S. National Seismic Hazard Map event
catalog produced by {\hyperref[references:petersen2014]{\crossref{\DUrole{std,std-ref}{Petersen et al. (2014)}}}}. By disaggregating these hazard maps, we will
be able to produce scenarios for a substantial number of the potential significant earthquakes
in the United States. It is hoped that these scenarios will satisfy most of
the requests that ShakeMap operators typically receive, and the need for ad
hoc scenarios will be minimized. Each regional seismic network will be ultimately
responsible for producing the scenarios for their region using their local ShakeMap
configuration and the fault and magnitude information provided from the hazard maps.
For areas outside of the regional networks, we will use the Global ShakeMap system to
produce the scenarios. International ShakeMap operators may be able to follow a similar
disaggregation of their own seismic hazard maps to generate a suite of scenarios.

After a suite of standardized ShakeMap scenarios is developed for a region or
a state, the ShakeMaps can be processed through HAZUS-MH, FEMA's loss and risk estimation software, to
develop associated damage estimates and other loss information products.
Both Utah and Washington State officials have worked with USGS, FEMA,
and other collaborators to produce online collections for scenario exercises and mitigation efforts,
shown in \hyperref[shakemap_archives:shakemap-hazus-utah]{Figure  \ref*{shakemap_archives:shakemap-hazus-utah}} and \hyperref[shakemap_archives:shakemap-hazus-washington]{Figure  \ref*{shakemap_archives:shakemap-hazus-washington}}, respectively.
\begin{figure}[htbp]\begin{flushleft}
\capstart

\includegraphics{{ShakeMap-HAZUS_Utah}.png}
\caption{State of Utah using ShakeMap-based earthquake scenario collection. More details can be found online
at the \href{https://www.fema.gov/media-library/assets/documents/16125}{FEMA}
and \href{http://www.shakeout.org/utah/scenarios/}{ShakeOut.org} Web sites.}\label{shakemap_archives:shakemap-hazus-utah}\label{shakemap_archives:id8}\end{flushleft}\end{figure}
\begin{figure}[htbp]\begin{flushleft}
\capstart

\includegraphics{{ShakeMap-HAZUS_Washington}.png}
\caption{Washington State ShakeMap-based earthquake scenario collection. More details can be found online
at the \href{https://fortress.wa.gov/dnr/seismicscenarios/}{Washington State (DNR)} Web site.}\label{shakemap_archives:shakemap-hazus-washington}\label{shakemap_archives:id9}\end{flushleft}\end{figure}
\begin{figure}[htbp]\begin{flushleft}
\capstart

\includegraphics{{ShakeMap-Washington_railways}.png}
\caption{Washington State ShakeMap-based earthquake scenario collection.
The selected layer (left) shows railways.}\label{shakemap_archives:shakemap-hazus-railways}\label{shakemap_archives:id10}\end{flushleft}\end{figure}

\hyperref[shakemap_archives:shakemap-hazus-railways]{Figure  \ref*{shakemap_archives:shakemap-hazus-railways}} provides an example Washington
State ShakeMap-based M9.0 Cascadia earthquake scenario.
More details can be found online at the
\href{https://fortress.wa.gov/dnr/seismicscenarios/}{Washington State (DNR)} Web site.


\section{Applications of ShakeMap}
\label{shakemap_applications::doc}\label{shakemap_applications:sec-shakemap-applications}\label{shakemap_applications:applications-of-shakemap}
The distribution of shaking from a significant earthquake, whether expressed as
PGA, PGV, or intensity, provides responding organizations a significant
increment of information beyond magnitude and epicenter.
Real-time ground-shaking maps provide an immediate opportunity to assess the
scope and impact of an event.  Thus, they can allow emergency managers and
responders to determine what areas were likely subjected to the highest
intensities and what the probable impacts were in those areas.  Importantly,
ShakeMap also affords decision-makers a rapid portrayal of those areas that
received only weak motions and are likely to be undamaged. The latter areas can
potentially be used for mutual aid.

Though initially developed primarily for emergency management, ShakeMaps have been
shown to be highly beneficial for other user sectors. These other uses include
improved loss estimation, public information and education through the media and
webpages, financial decision making, and engineering and seismological
research. Some specific examples are provided below for these use cases.

As a side benefit, an intensity-based depiction of shaking hazards through
ShakeMap (and DYFI) facilitates the adoption of the intensity scale more
generally. Inculcating the populace to earthquake shaking using intensity scales
(as opposed to magnitude alone) has become crucial not just for communicating
post-earthquake shaking hazards; the color coding and utilization of intensity
has more generally helped depict imminent and future shaking hazards. For
example, the ShakeMap intensity color palette has been adopted for Earthquake
Early Warning (EEW; see for example \href{http://www.shakealert.org/faq/}{QuakeAlert}) as well as for communicating future hazards
through deterministic scenarios and with Probabilistic Seismic Hazard Maps
(see {\hyperref[shakemap_archives:sec\string-scenarios]{\crossref{\DUrole{std,std-ref}{ShakeMap Scenarios}}}}).


\subsection{Emergency Management and Response}
\label{shakemap_applications:emergency-management-and-response}
The value of seismic monitoring and ShakeMap was addressed by a report by the National Research Council's
(NRC) ad-hoc Committee on the Economic Benefits of Improved Seismic
Monitoring ({\hyperref[references:nrc2006]{\crossref{\DUrole{std,std-ref}{National Research Council, 20006}}}}). In Chapter 7, ``Benefits for Emergency Response and Recovery”, the
committee refers to the Oct. 16, 1999 M7.1 Hector Mine, California earthquake (ShakeMap shown
in \hyperref[shakemap_applications:hector-mine-shakemap]{Figure  \ref*{shakemap_applications:hector-mine-shakemap}}).
\begin{figure}[htbp]\begin{flushright}
\capstart

\scalebox{0.500000}{\includegraphics{{NRC_11327-450}.jpg}}
\caption{NRC Seismic Monitoring Report}{\small \begin{quote}

\emph{The very rapid availability of earthquake source data---including
magnitude, location, depth, and fault geometry---provides basic
orienting information for emergency responders, essential
information for the news media and the public,
and input data for other applications and response-relevant
products. Maps of ground shaking intensity (ShakeMap) have many
important applications in emergency management. Because ShakeMap is available via the
Internet, all emergency responders at all levels of government and
the private sector have access to the same rapidly available information. With this
information, responders can quickly assess the scope of the emergency and
mobilize resources accordingly. Early reconnaissance efforts can target areas known to
have been shaken most severely, and key emergency services
including search and rescue, emergency medical response, safety
assessment of critical facilities, and shelter and mass care can be
expedited based on a more rapid identification of
incident location. Monitored information is also useful for rapidly
assessing situations in which a large, widely felt earthquake
occurs but causes little damage (such as the Hector Mine earthquake
of October 16, 1999). Clearly, there are significant economic
benefits in scaling a response to the consequences of
an event, including no response for an earthquake that requires none.}
\end{quote}
}\label{shakemap_applications:nrc-committee-report}\label{shakemap_applications:id1}\end{flushright}\end{figure}
\begin{figure}[htbp]\begin{flushleft}
\capstart

\includegraphics{{Hector_Mine_ShakeMap}.png}
\caption{Instrumental Intensity ShakeMap for the 1999
M7.1 Hector Mine, CA earthquake. (Map regenerated in 2006)}\label{shakemap_applications:hector-mine-shakemap}\label{shakemap_applications:id2}\end{flushleft}\end{figure}

Specifically in the context of disaster management and response in the
U.S., ShakeMap has been recognized as a top priority:
\begin{quote}

\emph{ShakeMap has
become a valuable tool to assist emergency responders in identifying
the likely extent of earthquake damage. Strong-motion data (now
increasingly available in real-time) can be correlated with
documentation and evaluation of the performance of the built
environment, leading to understanding the causes of earthquake damage
and the occurrence of good structural and non-structural performance''
({}`Western States Seismic Policy Council Policy Recommendation 14-3 \textless{}www.wsspc.org/wp-content/.../PR\_14-3\_SeismicMonitoring\_WebPub.pdf\textgreater{}{}`\_).}
\end{quote}

Similarly, a report by the National Science and Technology Council Subcommittee on Disaster Reduction* (Grand Challenges for Disaster Reduction: Priority Interagency Earthquake Implementation Actions) describes “Grand Challenge 1”:
\begin{quote}

\emph{Provide hazard and disaster information where and when it is needed. {[}...{]}
Expand the Advanced National Seismic System to improve seismic monitoring and
deliver rapid, robust earthquake information products; For all urban areas with
moderate to high seismic risk, produce ShakeMaps that show the variation of
shaking intensity within minutes after an earthquake based on near real time data
transmission from densely spaced seismic networks.}
\end{quote}

One of the earliest examples of the use of ShakeMap for emergency management and response was the
the M7.1 Hector Mine earthquake of October 16, 1999 (see \hyperref[shakemap_applications:hector-mine-shakemap]{Figure  \ref*{shakemap_applications:hector-mine-shakemap}}).
This event provides an important lesson in the use of
ShakeMap to assess the scope of an event and to determine the level of mobilization necessary.
The Hector Mine earthquake produced ground motion that was widely felt in
the Los Angeles basin and, at least in the immediate aftermath, required an assessment of potential impacts.
It was rapidly apparent, based on ShakeMap, that the Hector Mine earthquake was not a disaster, and despite an
extensive area of strong ground shaking, only a few small desert settlements were affected. Thus,
mobilization of a response effort was limited to a small number of companies
with infrastructure in the region and brief activations of emergency operations centers in
San Bernardino and Riverside Counties and the California Office of Emergency Services
(now the California Emergency Management Agency, or CalEMA).

While prioritizing earthquake response and management is considered
the primary goal of systems like ShakeMap, unnecessary response to an
earthquake---although not as costly as inadequate or misguided response in a real
disaster---can be avoided with proper well-constrained shaking information. Had the magnitude-7 earthquake occurred in
urban Los Angeles or another urban area in California, ShakeMap could be
employed to quickly identify the communities and jurisdictions requiring
immediate response.  To help facilitate the use of ShakeMap in emergency
response, ShakeMap is now provided to organizations with critical emergency
response functions automatically through USGS webpages, ShakeCast, and similar
tools.


\subsection{Loss Estimation}
\label{shakemap_applications:loss-estimation}
The \href{http://www.fema.gov/}{Federal Emergency Management Agency} (FEMA) employs ShakeMap for
post-earthquake damage assessments.
USGS generates customized, formatted ESRI shapefiles for direct input into
FEMA’s Hazards U.S. (\href{http://www.fema.gov/hazus}{HAZUS-MH} ; {\hyperref[references:fema2006]{\crossref{\DUrole{std,std-ref}{FEMA
(2006)}}}} loss estimation software. The customization includes specific contour intervals for
all events, geometric-mean ground motions (as opposed to ShakeMap standard
maximum component), and peak ground velocity in units of inches/sec rather than
cm/s. The HAZUS-formatted ShakeMap shapefiles are made available to FEMA for scenarios and
all significant domestic (U.S.) earthquakes
via webpages and ArcGIS services (see {\hyperref[products:gis\string-services]{\crossref{\DUrole{std,std-ref}{Web Mapping (GIS) Services}}}}).

The use of ShakeMaps as the shaking hazard input into HAZUS is a major improvement in loss-estimation accuracy
because actual ground-motion observations are used directly to assess damage, rather than relying on simpler
estimates based on epicenter and magnitude alone, or from predefined earthquake scenarios built into the HAZUS software.

FEMA's HAZUS loss estimates can be important for coordinating state and federal
response efforts, including Disaster Declarations. HAZUS's detailed impact
reports can provide focus to the mobilization of resources and
expedite the local, state, and federal disaster declaration process, thus
initiating the government’s response and
recovery machinery. ShakeMap, when overlaid with inventories of
critical lifelines and
facilities (e.g., hospitals, utilities, and substations), highways and
bridges, and vulnerable structures,
provides an important means of prioritizing response. Such response activities
can include shelter and mass care,
mutual aid assignments, emergency management, damage and safety assessment,
utility and lifeline restoration,
and emergency public information.
\begin{figure}[htbp]\begin{flushleft}
\capstart
\includegraphics{{parkfield_hazus}.png}\caption{2004 M6.0 Parkfield, CA earthquake ShakeMap shapefiles (green polygons) and HAZUS estimated impact to selected
infrastructure (circles) examined. Figure courtesy of D. Bausch, FEMA.}\label{shakemap_applications:id3}\end{flushleft}\end{figure}

As of 2015, the HAZUS-MH software is run interactively, not automatically, so
qualified FEMA personnel must be on hand to initiate HAZUS calculations and post
the results. In addition, for heavily populated areas (such as major cities in
California), HAZUS software can take a few hours to
compute losses. Thus, initial HAZUS-based losses are well behind initial
ShakeMap and PAGER results, and of course they are limited to
earthquakes in the U.S. However, the HAZUS results provide much greater detail and
information about infrastructure than PAGER-based aggregated losses.

As described in the section on Scenarios ({\hyperref[shakemap_archives:sec\string-shakemap\string-archives]{\crossref{\DUrole{std,std-ref}{ShakeMap Archives}}}}), HAZUS-MH is the standard approach for
delivering loss estimates for ShakeMap scenarios domestically. For real events,
the ShakeMap-to-HAZUS handoff has been formalized with a
liaison agreement (a Memo of Understanding, MOU) involving Doug Bausch, formerly
of FEMA Region VIII, and David Wald at the USGS NEIC. Because ShakeMap shaking
estimates evolve with time, and HAZUS loss estimates take time to compute, it is
essential that direct communications between the two agencies takes place
immediately after a serious earthquake to optimize loss
estimates.

The USGS-FEMA partnership has been activated for several domestic earthquakes
since this system was put into place, including 2004 M6.0 Parkfield, California;
2006 M6.0 Kiholo Bay, Hawaii; 2010 M7.2 Baja California, Mexico; 2011 M5.6
Prague, Oklahoma; 2011 M5.8 Mineral, Virginia; the M6.0 2015 American Canyon,
California; and several other events. The same approach has been
tested and applied retrospectively against the 1994 M6.7 Northridge, California and 1989 M6.9
Loma Prieta, California earthquakes, among others.


\subsection{Financial Sector Decision-Making}
\label{shakemap_applications:financial-sector-decision-making}
Post-earthquake financial decision making has evolved considerably over the past
decade. Insurers and reinsurers, private companies, governments, and aid
organizations have shown increasing creativity in the utilization of
near--real-time earthquake information for their own loss estimation, financial
adjudication, and situational awareness. Such financial analyses can be of
significant benefit to stakeholders, facilitating risk-transfer operations,
fostering sensible management of risk portfolios, and assisting disaster
responders. Ultimately, these improvements translate to benefits for the public
and those at risk ({\hyperref[references:franco2015]{\crossref{\DUrole{std,std-ref}{Franco, 2015}}}}).

In general, there are three categories of post-earthquake financial services and
decision making: 1) analysis of expected losses arising from an actual event
against a portfolio of exposures; 2) the triggering of payments for parametric
insurance products; and 3) the use of quantitative loss estimates to manage
disaster response and aid. Business and public-sector portfolio managers can
employ tools like ShakeCast or in-house applications to automatically retrieve
and compute losses based on pre-assigned fragility curves. Within the
(re)insurance sector, catastrophe (CAT) bonds and contingency loans based on
earthquake risk models are often triggered via parametric analyses, which are
dependent on earthquake parameters or intensity-measure (IM) estimates as well
as their uncertainties.

Anticipating potential losses and acting rapidly and
accordingly is also of utmost importance to emergency management and disaster
aid communities. Estimated losses constitute vital input for rapid situational
awareness, facilitating decision-making on whether or not to commit and deploy
resources, and to what level.
\setbox0\vbox{
\begin{minipage}{0.95\linewidth}
\textbf{USE CASE \#1}

\medskip


The Inter-American Development Bank \url{http://www.iadb.org} (IADB) employs ShakeMap for objective post-earthquake assessments for within 72 hours of any significant earthquake in Latin America and the Caribbean (LACR). IADB's Contingent Credit Facility Loans has set up disaster contingency loans for up to several hundred million USD, conditional on predefined levels of population exposed to ShakeMap intensity VI and higher. Typically, loans can be distributed when the population experiencing intensity VI or higher reaches at least 2\% of the population within the coverage area. Loans are available in six LACR countries during the period of availability (J. Martinez, IADB, written comm., 2014).
\end{minipage}}
\begin{center}\setlength{\fboxsep}{5pt}\shadowbox{\box0}\end{center}

To a large extent, the advancement of post-earthquake financial instruments has
been facilitated by the availability of rapid and accurate earthquake parameters
and more quantitative geospatial hazard information. Commensurately, USGS
products like ShakeMap and PAGER have evolved to further accommodate specific
requirements of the financial sector. For instance, improved approaches for
quantifying uncertainty can better inform loss estimates, and historical ShakeMap
Atlas data can assist in loss-model calibration. In addition, USGS now provides
PAGER loss estimates broken down by country to fulfill the need required in the
CAT bond and contingency loan arena, while still remaining within the confines
of reasonable spatial accuracy. Similarly, requests have been made by U.S. State
governments to further compute losses at the state level, although such
resolution is not yet warranted, particularly in areas of sparse real-time
strong-motion instrumentation. Lastly, for many uses, the automatic retrieval
and processing of ShakeMaps has been facilitated via GeoJSON feeds, webmapping
servers, and the ShakeCast systems.

Several types of data and information products available or under development
that may be of benefit to the financial sector. The generation of suites of
standardized earthquake scenarios---both domestic and internationally---is underway,
and an update of the global Atlas of ShakeMaps has been completed (see
{\hyperref[shakemap_archives:sec\string-shakemap\string-archives]{\crossref{\DUrole{std,std-ref}{ShakeMap Archives}}}}).

There are several continuing challenges under consideration
and scrutiny: implementing directivity; computing and depicting spatial ground
motion correlations; improving approaches for quantifying and conveying
uncertainties; and creating more explicit ShakeMap policy and version-control
documentation. {\hyperref[references:wald2016]{\crossref{\DUrole{std,std-ref}{Wald and Franco (2016)}}}} describe how these
advances may in turn facilitate the appearance of new and more refined financial
instruments and insurance products.


\subsection{Public Information and Education}
\label{shakemap_applications:public-information-and-education}
The rapid availability of ShakeMap on the Internet, combined with the urgent
desire for information following a significant earthquake, makes this mapping
tool a huge potential source of public information and education. In instances
in which an earthquake receives significant news coverage, the ShakeMap site
and “Did You Feel It?” (DYFI) receive an enormous influx of
visitors ({\hyperref[references:wald2011c]{\crossref{\DUrole{std,std-ref}{Wald et al., 2011}}}}). Such opportunities are
amplified by widespread adoption of ShakeMap into media and educational
materials by other institutions.

ShakeMap's intensity scale is key for introducing and impressing upon the public and
the media the importance of macroseismic intensity, rather than the continuing
sole dependence on magnitude as the scale of reference for earthquakes. Although
Japanese Meterological Agency (JMA) Intensity (.e.g., {\hyperref[references:jma1996]{\crossref{\DUrole{std,std-ref}{JMA, 1996}}}}) differs slightly from its U.S. counterpart, Modified Mercalli Intensity (MMI)---JMA's is strictly
instrumentally-derived---it is widely used and understood in Japan (e.g.,
{\hyperref[references:celsi2005]{\crossref{\DUrole{std,std-ref}{Celsi et al., 2005}}}}). JMA has successfully made intensity the
norm for communicating to the Japanese population about real-time and future
earthquake hazards via television, smartphone, web content, annual earthquake
drills, and the educational system. Because JMA intensity is widely understood,
the public is be more attuned to earthquake risks than populations familiar only
with magnitude descriptions of earthquakes (e.g., {\hyperref[references:celsi2005]{\crossref{\DUrole{std,std-ref}{Celsi et al., 2005}}}}).
\begin{quote}

\emph{Earthquake education also occurs through the media. The anchoring effect we
report may be lessened significantly if the press consistently used the Mercalli
scale and helped to educate the public about the scale. Research should be
conducted to better understand if and how news organizations can successfully
utilize the Mercalli scale in communicating earthquake information. Alternative
formats, for example, using letters rather than Roman numerals for the
categories, may ameliorate the confusion between magnitude and Mercalli scales.
The experience in Japan provides support for the idea that laypeople can learn
to use the two scales side by side. The Japanese media report both intensities
and magnitude, with viewers maintaining a clearer understanding of the
relationship between magnitude and intensity. In Japan, the overall magnitude
and the intensity are both instrument numbers, with the latter being
location-specific.}
\end{quote}

The acclimatization of the public to intensity is inline with the findings of
{\hyperref[references:gomberg2013]{\crossref{\DUrole{std,std-ref}{Gomberg and Jokobitz (2013)}}}}:
\begin{quote}

\emph{Simpler messaging and
explanations are needed by some users,
and this may be achieved by developing two styles of some products, one designed
for nontechnical users and the other tailored for engineers and scientists. The
tangible impacts of an earthquake must be conveyed more simply and succinctly,
employing a scale useful for decision-making at the regional and local levels.}
\end{quote}

Acknowledging the importance of ShakeMap as a tool for public information and
education, considerable effort was taken to provide a range of formats
suitable for broadcast and webpages. Initially “Media Maps”, simplified versions
of the Instrumental Intensity maps, were packaged in a way that makes them more
suitable for broadcast
to low-resolution devices, such as TV monitors---roads and borders are thicker,
fonts are
larger, and the title and intensity scale are simplified---and a ``TV guide''
information sheet was provided to supplement the Media Maps, to allow
easier delivery of basic earthquake information. These formats have naturally
evolved to GIS, KML, and now interactive (zoomable) maps that allow
customization of the basemap layers and other content. Such
interactive maps are in favor in newsrooms and educational contexts.

However, some of the static maps have made for the most widespread distribution.
A widely used graphic (\hyperref[shakemap_applications:northridge-nisqually]{Figure  \ref*{shakemap_applications:northridge-nisqually}}), for example,
compares ShakeMap-generated intensities for the 1994 Northridge earthquake, a
shallow crustal earthquake near Los Angeles, with the 2001 deep, intraslab
Nisqually, WA earthquake. This figure was reprinted in numerous reports,
textbooks, classes, reports, and briefings, including \href{http://www.earthquakecountry.org/roots/shaking.html}{Putting Down Roots} and the {\hyperref[references:nrc2006]{\crossref{\DUrole{std,std-ref}{National
Research Council}}}}.
\begin{figure}[htbp]\begin{flushleft}
\capstart

\includegraphics{{Northridge_Nisqually}.png}
\caption{Widely adopted graphic of comparing ShakeMaps for the 2001 M6.8 Nisqually, WA and 1994 M6.7 Northridge, CA earthquakes, showing how distance from an earthquake affects the level of shaking experienced. Even though the magnitude of the Nisqually earthquake was slightly greater than that of the Northridge earthquake, the shaking was lower on average, primarily because the fault that ruptured during the Northridge earthquake was shallower (5-20km deep) than that of the Nisqually earthquake (about 45-50km deep).}\label{shakemap_applications:northridge-nisqually}\label{shakemap_applications:id4}\end{flushleft}\end{figure}

The continued longterm education of the public to intensity continues through many TV channels and other means,
for instance, in academic courses (e.g., \href{http://web.ics.purdue.edu/~braile/edumod/eqhazard/eqhazard2.htm}{Larry Braile's undergraduate courses}), textbooks (e.g., {\hyperref[references:yeats2004]{\crossref{\DUrole{std,std-ref}{Yeats, 2004}}}}
``Living with Earthquakes in the Pacific Northwest''),
and even \href{https://en.wikipedia.org/wiki/2001\_Nisqually\_earthquake}{Wikipedia}.


\subsection{Emergency Preparedness}
\label{shakemap_applications:emergency-preparedness}
One of the leading tools for earthquake emergency preparedness has
been the widespread adoption of \href{http://www.shakeout.org/home.html}{``ShakeOut''} and other earthquake drills and
planning scenarios. In many of these cases, ShakeMap is employed both for
developing the framework for portraying each earthquake in its
hazard context and for computing loss estimates to examine and
communicate its potential societal impact. The initial success of the
Great Southern California ShakeOut ({\hyperref[references:jones2011]{\crossref{\DUrole{std,std-ref}{Jones et al. (2011}}}}) has been built by SCEC, USGS, and others into a
worldwide \href{http://www.shakeout.org/home.html}{annual exercise}
(on Oct 15th of each year) involving millions of participants.

On a statewide basis, exercises take place in several of the more
tectonically active areas of the country, such as \href{http://www.shakeout.org/utah/scenarios/}{ShakeOuts in
Utah} and the 2012
\href{http://earthweb.ess.washington.edu/gomberg/ShakeMap/ShakeMaps.html}{Evergreen Earthquake Exercise ShakeMaps}
in Washington State.

Nationwide, FEMA's \href{http://www.fema.gov/national-exercise-program}{National Level Exercises (NLEs)} program is
another source for planning for complex, whole-community, large-scale disasters and emergencies.
Here, too, NLEs often employ ShakeMap as the basis for their
exercises. The ShakeMap-HAZUS combination was to support the \href{http://www.cusec.org/plans-a-programs/multi-state-planning/156-cusec-new-madrid-seismic-zone-catastrophic-planning-project.html}{New Madrid 2011 NLE},
involving an M7.7 New Madrid region mainshock and several significant aftershocks. In 2014,
the ``Capstone Exercise'' NLE was a complex emergency
preparedness exercise including the Alaska Shield 2014 exercise, sponsored by the State of Alaska to
commemorate the 50th anniversary of the 1964 Great Alaskan Earthquake. The exercise
also involved significant damage from earthquake shaking as well as
tsunami, triggering impacts in the Pacific Northwest. The Department of Defense (DOD) aligned key components
of the Capstone exercise with a connected ``Ardent Sentry'' table-top exercise with the
same ShakeMap input in order for DOD to focus on defense support to civilian authorities.
\begin{figure}[htbp]
\centering
\capstart

\includegraphics{{caribewave_onepager}.pdf}
\caption{Annual ``Caribe Wave'' earthquake and tsunami exercise for the Caribbean region.}\label{shakemap_applications:caribewave-nle}\label{shakemap_applications:id5}\end{figure}

Internationally, USGS participates (through the National Oceanic and Atmospheric Association, NOAA) in an annual ``Caribe
Wave'' earthquake and tsunami exercise for the Caribbean region
({\hyperref[references:ioc2012]{\crossref{\DUrole{std,std-ref}{IOC, 2012}}}}; see \hyperref[shakemap_applications:caribewave-nle]{Figure  \ref*{shakemap_applications:caribewave-nle}}). The USGS ShakeMap and PAGER group also work directly with the U.S.
Agency for International Development (USAID) Office of Foreign
Disaster Assistance (OFDA), the World Bank, and Geohazards International (GHI) (among many other agencies, countries, and NGOs) to
develop global planning exercises and scenarios.


\subsection{Earthquake Engineering and Seismological Research}
\label{shakemap_applications:earthquake-engineering-and-seismological-research}
For potentially damaging earthquakes, ShakeMap produces response spectral
acceleration grid values for three periods (0.3, 1.0, and 3.0 sec). The
spectral acceleration values are used for loss estimation, as
mentioned above, yet these measures also serve many earthquake engineering analysis purposes. In a
post-earthquake environment, information from engineering analyses of structures
(including via ShakeCast, see below) provides a framework for post-earthquake
occupancy, tagging, and damage inspection by civil engineers.

ShakeMap products and metadata aggregate earthquake source
information, shaking intensity measures (IMs) including both seismic and macroseismic observations, and fault geometries and station-sources distances.
In addition to providing these data systematically for recent events,
the same constraints are made available for numerous earthquakes, for recent events
as well as historical events ({\hyperref[shakemap_archives:sec\string-shakemap\string-archives]{\crossref{\DUrole{std,std-ref}{ShakeMap Archives}}}}).

The aggregation of earthquake information and fault geometries---in
conjunction with reported shaking and macroseismic data---provide the
basis for analyses of best-estimate ground motion IMs at specific
sites for comparison with human behavior and response by the
natural and built environments. Here is a
sampling of the range of studies these products motivate and facilitate:

\textbf{Example Engineering Research and Analyses:}
\begin{itemize}
\item {} 
Analyses of potential damage to column/beam welds in steel
buildings following the 1994 Northridge earthquake. {\hyperref[references:atc2002]{\crossref{\DUrole{std,std-ref}{ATC 2002}}}}.

\item {} 
ATC-54: Guidelines for using strong-motion data and ShakeMaps in
Post-Earthquake Response.  {\hyperref[references:atc2002]{\crossref{\DUrole{std,std-ref}{ATC 2002}}}}.

\item {} 
An Empirical Model for Global Earthquake Fatality
Estimation. {\hyperref[references:jaiswal2010]{\crossref{\DUrole{std,std-ref}{Jaiswal and Wald (2010)}}}}.

\item {} 
Guidelines for the Collection of Consequence Data, Global Earthquake Consequences Database Global
Component Project. {\hyperref[references:pomonis2011]{\crossref{\DUrole{std,std-ref}{Pomonis and So (2011)}}}}.

\item {} 
ShakeCast Case Study on Nevada Bridges. {\hyperref[references:biasi2016]{\crossref{\DUrole{std,std-ref}{Biasi et al (2016)}}}}.

\end{itemize}

\textbf{Example Seismological Research and Analyses:}
\begin{itemize}
\item {} 
Intensity attenuation for active crustal regions. {\hyperref[references:allen2012]{\crossref{\DUrole{std,std-ref}{Allen et al, 2012}}}}.

\item {} 
Ground Motion to Intensity Conversion Equations (GMICEs): A Global
Relationship and Evaluation of Regional Dependency. {\hyperref[references:caprio2015]{\crossref{\DUrole{std,std-ref}{Caprio
et al. (2015)}}}}.

\item {} 
Fault extent estimation for near-real time ground shaking map
computation purposes. {\hyperref[references:convertito2011]{\crossref{\DUrole{std,std-ref}{Convertito et al. (2011)}}}}.

\item {} 
Bayesian Estimations of Peak Ground Acceleration and 5\% Damped
Spectral Acceleration from Modified Mercalli Intensity Data*.
{\hyperref[references:ebel2003]{\crossref{\DUrole{std,std-ref}{Ebel and Wald (2003)}}}}.

\item {} 
Regression analysis of MCS intensity and ground motion parameters
in Italy and its application in ShakeMap. {\hyperref[references:faenza2010]{\crossref{\DUrole{std,std-ref}{Faenza and Michilini (2010)}}}}

\item {} 
A Global Earthquake Discrimination Scheme to Optimize Ground-Motion
Prediction Equation Selection. {\hyperref[references:garcia2012b]{\crossref{\DUrole{std,std-ref}{Garcia et al. (2012b)}}}}.

\end{itemize}


\section{Related Systems}
\label{related_systems::doc}\label{related_systems:related-systems}\label{related_systems:sec-related-systems}
Here is a brief listing of USGS Earthquake Hazards Program rapid earthquake information products:
\begin{itemize}
\item {} 
\href{https://sslearthquake.usgs.gov/ens}{Earthquake Notification System} sends automated, customizable notifications of earthquakes through email, pager, or cell phone.

\item {} 
\href{http://earthquake.usgs.gov/earthquakes/map/}{Realtime Earthquake Map}: Automatic maps and event information
displayed online within minutes after earthquakes worldwidema

\item {} 
\href{http://earthquake.usgs.gov/earthquakes/shakemap/}{ShakeMap} automatically generates maps displaying
instrumentally measured shaking intensities.

\item {} 
\href{http://earthquake.usgs.gov/earthquakes/dyfi/}{Did You Feel It?}: Map of earthquake effects derived from citizen
input via online Web forms

\item {} 
{\hyperref[related_systems:pager]{\crossref{PAGER}}} (Prompt Assessment of Global Earthquakes for Response) rapidly
compares the population exposed to various shaking intensities to estimate likely fatalities and economic losses

\item {} 
\href{http://www.cisn.org/software/cisndisplay.html}{CISN Display}: Stand-alone application that graphically alerts
users, in near real time, of earthquakes and related hazards information.

\item {} 
\href{http://earthquake.usgs.gov/research/software/shakecast/}{ShakeCast}: An application for automated delivery of
ShakeMaps and potential damage or inspection priority for specific user-selected facilities.

\item {} 
\href{http://www.shakealert.org/faq/}{ShakeAlert}: Prototype Earthquake Early Warning (EEW) System.

\end{itemize}

While ShakeMap has met with success as a standalone product for communicating
earthquake effects to the public and the emergency response and recovery
community, it is increasingly being incorporated into value-added products that
help in the assessment of earthquake impacts for response management and
government officials.
\begin{figure}[htbp]
\centering
\capstart

\includegraphics{{SMap_SCast_DYFI_PAGER}.png}
\caption{Interplay between ShakeMap, DYFI, ShakeCast, and PAGER.}\label{related_systems:figure-related-systems}\label{related_systems:id2}\end{figure}

As discussed in detail the {\hyperref[technical_guide:technical\string-guide]{\crossref{\DUrole{std,std-ref}{Technical Guide}}}}, ShakeMap is augmented by
DYFI input for constraining intensities, and from those, estimates of peak
ground motions (in some cases, and for some regions), as shown
in \hyperref[related_systems:figure-related-systems]{Figure  \ref*{related_systems:figure-related-systems}}.  DYFI and ShakeMap in conjunction then represent
the shaking hazard
input for two other primary systems that estimated losses: ShakeCast and PAGER.
ShakeCast is intended for specific users to prioritize response for specific
user-centric portfolios of facilities; PAGER is for more general societal-impact
assessments, providing estimated loss of life and economic impacts for the
region affected.


\subsection{ShakeCast}
\label{related_systems:id1}\label{related_systems:sec-shakecast}
\href{http://earthquake.usgs.gov/research/software/shakecast/}{ShakeCast} is a freely available
post-earthquake situational awareness application that automatically retrieves
earthquake shaking data from ShakeMap, compares intensity measures against
users’ facilities, and generates potential damage assessment notifications,
facility damage maps, and other web-based products for emergency managers and
responders.
\setbox0\vbox{
\begin{minipage}{0.95\linewidth}
\textbf{USE CASE \#2}

\medskip


The \href{http://www.earthquakeauthority.com/}{California Department of Transportation} (Caltrans) employs ShakeMap
for post-earthquake overpass and bridge assessments for significant
California earthquakes:

\emph{The Caltrans ShakeCast system performed
reliably and as expected during the Napa earthquake. The system delivered key
information on the potential impacts to the state bridge inventory
within 11 minutes of the event.  Of a total of 2720 state bridges
within the ShakeMap region, 87 state bridges were identified by
ShakeCast as having undergone significant ground shaking and were
assigned an inspection priority state. Of the 87 state bridges
identified, 9 were later confirmed to have sustained minor damage.
These 9 state bridges were ranked in the top 40\% of the ShakeCast
list.}
--({\hyperref[references:turner2014]{\crossref{\DUrole{std,std-ref}{Turner, 2014}}}})
\end{minipage}}
\begin{center}\setlength{\fboxsep}{5pt}\shadowbox{\box0}\end{center}

ShakeCast, short for ShakeMap Broadcast, is a fully automated system for
delivering specific ShakeMap products to critical users and for triggering
established post-earthquake response protocols. ShakeMap was developed
and is used primarily for emergency response, loss estimation, and public
information; for an informed response to a serious earthquake, critical users
must go beyond just looking at ShakeMap, and understand the likely extent and
severity of impact on the facilities for which they are responsible. To this
end, the USGS has developed ShakeCast.
\begin{figure}[htbp]
\centering
\capstart
\includegraphics{{Caltrans_Napa_Report}.png}\caption{Caltrans ShakeCast report for the 2011 M6.0 Napa, CA earthquake.}\label{related_systems:id3}\end{figure}

ShakeCast allows utilities, transportation agencies, businesses, and other
large organizations to control and optimize the earthquake information they
receive. With ShakeCast, they can automatically determine the shaking value at
their facilities, set thresholds for notification of damage states for each
facility, and then automatically notify (by pager, cell phone, or email)
specified operators and inspectors within their organizations who are
responsible for those particular facilities so they can set priorities for
response.
\setbox0\vbox{
\begin{minipage}{0.95\linewidth}
\textbf{USE CASE \#3}

\medskip


\emph{``Thought you might like to see the {[}Division of Safety of Dams{]}
ShakeCast message for the recent Napa {[}Aug, 2014{]} Earthquake.  We have since
divided the 1250 dams into three fragility classes (called levels of
concern). The message provides explicit instructions on what action
to take for each dam and transmits owner contact information. The
message was received in my inbox 16 minutes after the earthquake,
which was about 10 minutes after the ShakeMap version 1 was
released. The technology has become very well accepted by the field
inspectors. Thanks for such a great product.''}
--W. A. Fraser, C.E.G.,
Chief, Geology Branch, CA Division of Safety of Dams, Sacramento, CA.
\end{minipage}}
\begin{center}\setlength{\fboxsep}{5pt}\shadowbox{\box0}\end{center}


\subsection{PAGER}
\label{related_systems:pager}
Another important USGS product that uses ShakeMap output as its primary data
source is {\hyperref[related_systems:pager]{\crossref{PAGER}}} (Prompt Assessment of Global Earthquakes for Response), an
automated system that produces content concerning the impact of significant
earthquakes around the world, informing emergency responders, government and aid
agencies, and the media of the potential scope of the disaster. PAGER rapidly
assesses earthquake impacts by comparing the population exposed to each level of
shaking intensity with models of economic and fatality losses based on past
earthquakes in each country or region of the world. Earthquake alerts---which
were formerly sent based only on event magnitude and location, or population
exposure to shaking---will now be generated based also on the estimated range of
fatalities and economic losses.

PAGER alerts are based on the “Earthquake Impact Scale” developed by {\hyperref[references:wald2011b]{\crossref{\DUrole{std,std-ref}{Wald et al. (2011)}}}}.
\begin{figure}[htbp]\begin{flushright}
\capstart
\includegraphics{{Nepal_M7_8_onepager_V5}.pdf}\caption{Nepal OnePAGER Alert Example}\label{related_systems:id4}\end{flushright}\end{figure}


\subsection{Public and Private Sector Tools}
\label{related_systems:public-and-private-sector-tools}
Alternatives, modifications, and enhancements to the ShakeMap methodology are
used widely around the world. Likewise, downstream derivative products and systems for loss estimation are
widely employed, both in the public and private sector. What follows is
a brief (and incomplete) description of some of these systems. Many
proprietary hazard and loss modeling systems exist in the private
sector, and typically they are openly described or referenced.

On the shaking hazard front, domestically, some public/private sector
utilities run in-house shaking aggregation and estimation systems,
including the East Bay Metropolitan Utility District (EBMUD's Marconi
system) and Pacific Gas and Electric (PG\&E).

Impressive systems also exist in Japan, Taiwan, New Zealand, Turkey,
among other countries.
\begin{itemize}
\item {} 
JMA

\item {} 
GNS

\item {} 
INGV

\end{itemize}

On the rapid loss estimation front, several systems are in place in the U.S.

Internationally, {\hyperref[references:erdik2011]{\crossref{\DUrole{std,std-ref}{Erdik et al. (2011)}}}}
and {\hyperref[references:erdik2014]{\crossref{\DUrole{std,std-ref}{Erdik et al. (2014)}}}} provide examples of
operative rapid earthquake loss estimation systems.
\begin{itemize}
\item {} 
Taiwan Earthquake Rapid Reporting System,

\item {} 
Realtime Earthquake Assessment Disaster System in Yokohama

\item {} 
Real Time Earthquake Disaster Mitigation System of the Tokyo Gas
Co.

\item {} 
IGDAS Earthquake Protection System

\item {} 
Istanbul Earthquake Rapid Response System

\item {} 
ELER

\item {} 
SELENA

\item {} 
OpenQuake (OQ, GEM Foundation)

\item {} 
GDACS

\item {} 
QuakeLoss (WAPMERR)

\item {} 
PAGER (USGS)

\end{itemize}

\begin{notice}{note}{Note:}
Links and pointers to non-USGS sites are provided for information only and do not constitute endorsement by the USGS (see \href{http://www.usgs.gov/laws/info\_policies.html}{USGS policy and disclaimers}).
\end{notice}

Lastly, many systems are available and in operation in the U.S. for
aggregating hazard and impact information for emergency response and
awareness. Many are multihazard oriented, and only those with focus on
earthquake information are mentioned here. Some examples include:
\begin{itemize}
\item {} 
InLet (ImageCat,Inc.)

\item {} 
HAZUS-MH,

\item {} 
ArcGIS online

\end{itemize}

As summarized by {\hyperref[references:gomberg2013]{\crossref{\DUrole{std,std-ref}{Gomberg and Jokobitz (2013)}}}}:
“others have built in-house systems to organize, share and display observations
using commercial applications like Microsoft’s Streets and Trips and SharePoint,
Google’s GoogleEarth, or ESRI’s ArcGIS. WebEOC, a real-time Web-enabled crisis
information management system developed commercially by Esri, is meant to be an
official link among public sector emergency managers in Washington State (see
\url{http://www.esi911.com/esi}). While used by many agencies, it always was just one
of multiple communication tools. A commonly expressed desire was for a
centralized hub for all types of disaster information (like the
Department of Homeland Security’s \href{https://www.dropbox.com/home/Correlation/figures/SanDiego?preview=economic+losses0.png}{Virtual USA}).''

Further information on private sector tools can
be found in the Department of Homeland Security
(DHS) summary for the \href{http://www.cusec.org/capstone14/documents/2014.03.06\_PSW/2014.03.06\_CAPSTONE\_Private\_Sector\_GIS.pdf}{Capstone 2014}
National Level (scenario) Exercise.


\section{Disclaimers}
\label{disclaimers::doc}\label{disclaimers:sec-disclaimers}\label{disclaimers:disclaimers}

\subsection{General Disclaimer}
\label{disclaimers:general-disclaimer}
\begin{notice}{warning}{Warning:}
Some USGS information accessed through this page may be preliminary in nature and presented prior to final review and approval by the Director of the USGS. This information is provided with the understanding that it is not guaranteed to be correct or complete, and conclusions drawn from such information are the sole responsibility of the user. In addition, ShakeMaps are automatic, computer-generated maps and have not necessarily been checked by human oversight, so they may contain errors. Further, the input data is raw and unchecked, and may contain errors.
\end{notice}
\begin{itemize}
\item {} 
Contours can be misleading, since data gaps may exist. Caution should be used
in deciding which features in the contour patterns are required by the data.
Ground motions and intensities can vary greatly over small distances, so these
maps are only approximate.

\item {} 
Locations within the same intensity area will not necessarily experience the
same level of damage, since damage depends heavily on the type of structure,
the nature of the construction, and the details of the ground motion at that
site. For these reasons, more or less damage than described in the intensity
scale may occur. The ground motion levels and descriptions associated with
each intensity value are based on recent damaging earthquakes. There may be
revisions in these parameters as more data become available or due to further
improvements in methodologies.

\item {} 
Large earthquakes can generate very long-period ground motions that can cause
damage at great distances from the epicenter; although the intensity estimated
from the ground motions may be small, significant effects to large structures
(e.g., bridges, tall buildings, storage tanks) may be notable.

\item {} 
The utilization of DYFI data on ShakeMap, in addition to using recorded peak
ground motions, is standard-operating-procedure for some ShakeMap
operations, including the Global ShakeMap (GSM) system operated out of the
USGS NEIC. The algorithmic strategy for including these data in ShakeMap is
documented in the Technical Guide. As described by Wald et al.
(2011), the ultimate discretion to use, filter, or overrule specific
internet-based or historic intensities (or specific suspect strong motion
data, for that matter) remains with the ShakeMap operators. A number of
filtering and quality control strategies are in place (e.g., Wald et al.,
2011), but erroneous or suspect data can not always be culled immediately.
While we make efforts to provide consistent quality control of the data, the
DYFI system depends upon open, citizen-science based input from the public.
A number of studies have shown these data to be generally reliable, but the
data reliability may vary from event to event. Moreover, macroseismic
intensities are fundamentally non-unique: Differing polygonal aggregations
for computing Community Decimal Intensities (CDI, using geocoded boxes or
ZIP codes, for example, Wald et al., 2011) may yield varying intensity
values at specific locations. {[}Historic or modern MMI or EMS-98 intensity
assignments are also non-unique; the assignment can vary from expert to
expert, the selection of areas may vary, and occasionally different
structure types may indicated alternative intensities{]}. Changes to the
size of the areas used to aggregate CDIs often trade off a greater number
of responses per polygon (hence greater confidence in the derived
intensity) against a more precise spatial location. DYFI data are
routinely used on the GSM systems and other regional ShakeMap systems of the
ANSS. DYFI data are not currently (as of 2016) used in the Northern or
Southern California ShakeMap systems, in part due to the adequacy of
strong-motion station coverage there.

\end{itemize}


\subsection{ShakeMap Update Policy}
\label{disclaimers:shakemap-update-policy}
\begin{notice}{warning}{Warning:}
ShakeMaps are preliminary in nature and will be updated as data arrives from a variety of distributed sources. Our practice is to improve the maps as soon as possible, but there are factors beyond our control that can result in delayed updates.
\end{notice}
\begin{itemize}
\item {} 
For regions around the world, where there are insufficient near--real-time
strong-motion seismic stations to generate an adequate strong-ground-motion
data-controlled ShakeMap, we can still provide a very useful estimate of the
shaking distribution using the ShakeMap software. Site amplification is
approximated from a relationship developed between topographic gradient and
shear-wave velocity. Additional constraints for these predictive maps come
primarily from  additional earthquake source information, particularly fault
rupture dimensions, observed macroseismic intensities (including via the USGS
``Did You Feel It?'' system), and observed strong ground motions, when and where
available.

\item {} 
There is no formal “final” version of any ShakeMap. Version Control is up to
users. ShakeMap version numbers and timestamps are provided on the maps,
webpages, grid files, and metadata.

\item {} 
Our strategy is to update ShakeMaps as warranted from a scientific
perspective. We reserve the option to update ShakeMaps as needed to add data
or to improve scientific merit and/or presentation of the maps in any way
beneficial. This most typical update is after new data arrive, finite-fault
models get established or revised, magnitude gets revised, or as improved site
amplification maps, ground motion prediction equations, or even interpolation
or other procedures become available.

\end{itemize}
\setbox0\vbox{
\begin{minipage}{0.95\linewidth}
\textbf{Recent ShakeMap update examples:}

\medskip

\begin{itemize}
\item {} 
For the very deadly 2008 Wenchuan, China, earthquake, the Chinese strong-motion data were not made available for several months.

\item {} 
For the 2011 Tohoku, Japan earthquake, the magnitude was updated from 7.9 to 8.9 over the course of the first hour after origin time. The Japanese strong-motion data processing center was impacted by the earthquake, yet they provided data for nearly a thousand seismic stations within several days after the earthquake. These vital data were added to the ShakeMap as soon as they became available.

\item {} 
Due to telemetry limitations, some important seismic station data for the 2014 American Canyon, California, earthquake came in minutes, hours, and as late as four days after the event. The data were added to the ShakeMap soon after they were received and processed. The magnitude also changed from an initial M5.7 to M6.0, and this, too, affected the ShakeMap. Lastly, the causative fault location was added by the Northern California ShakeMap operators several days after the earthquake.

\end{itemize}
\end{minipage}}
\begin{center}\setlength{\fboxsep}{5pt}\shadowbox{\box0}\end{center}

\textbf{Updates to Online Maps}
\begin{itemize}
\item {} 
\textbf{Real-time ShakeMap Updates}. Changes can be tracked with the ShakeMap
version numbers and timestamps found in the metadata, the \emph{info.xml} and
\emph{grid.xml} files, and on the maps themselves (time generated). The \emph{info.xml}
file contains timestamps, number of stations used, GMPE information, and
many other attributes that could have changed from version to version.
Often a text statement is provided that notes significant changes for a
particular version.

\item {} 
\textbf{ShakeMap Atlas Updates}. The ShakeMap Atlas uses version numbers
for each Atlas event; the overall Atlas collections is also Versioned.
Currently ShakeMap Atlas Version 2.0 is online in the ComCat database, and
the older Atlas (Version 1.0) can be found online on the \href{http://earthquake.usgs.gov/earthquakes/shakemap/}{legacy ShakeMap
Archives pages}.

\item {} 
\textbf{Scenario Revisions}. ShakeMap Scenario collections uses version
numbers for each event; the overall scenario collections is named
according to their source. Currently
ShakeMap Atlas Version 2.0 is online in the USGS \href{http://earthquake.usgs.gov/earthquakes/search/}{Comprehensive Catalogue
(ComCat) Earthquake database}. Some older
scenarios are online on the \href{http://earthquake.usgs.gov/earthquakes/shakemap/}{legacy ShakeMap Archives pages}. Scenario ShakeMaps
will be revised when the underlying probabilistic seismic map fault
segmentation or other particulars (like GMPE selection) change. Older
versions will be archived in \href{http://earthquake.usgs.gov/earthquakes/search/}{ComCat}.

\end{itemize}


\chapter{Software \& Implementation Guide}
\label{software_guide::doc}\label{software_guide:software-guide}\label{software_guide:software-implementation-guide}
Installing and operating a ShakeMap system is a non-trivial endeavor. While the software
can be easily obtained, and installed and configured with a few hours work, there are a
great many issues that need to be addressed when an organization considers deploying a
ShakeMap system. In this section, we will discuss a number of these issues.


\section{Implementation Considerations}
\label{software_guide:implementation-considerations}

\subsection{Seismic Network}
\label{software_guide:seismic-network}
The single most important prerequisite for operating a ShakeMap system is the existence
of a seismic network capable of determining the location and magnitude of significant
earthquakes within minutes of their occurrence. This network must also produce
parametric data for ShakeMap within minutes of the earthquake. The parametric data
should come from free-field strong motion sensors with real-time or near--real-time
telemetry, and minimally consist of the horizontal components of Peak Ground
Acceleration (PGA) and Peak Ground Velocity (PGV), as well as (ideally) 5\% damped pseudo-
spectral acceleration (PSA) at 0.3, 1.0, and 3.0 sec periods.

In the absence of such a network, there is little reason to operate ShakeMap locally. The
USGS operates a ShakeMap system that produces maps for significant global
earthquakes within a few minutes of their occurrence, and incorporates any available data
from the Global Seismic Network (GSN), “Did You Feel It?” (DYFI), and any other parametric data to which
we have access. These maps are immediately delivered to the PAGER system and
ShakeCast. It would be far simpler in many cases to obtain the
ShakeMap products from the USGS (through the website or ShakeCast) and use them for
local or regional response and recovery efforts, perhaps even hosting them on local
servers, than to attempt to duplicate our existing system.

Even where a seismic network is operational, it may be preferable to work with the
USGS to produce ShakeMaps. As the following sections will illustrate, running a
ShakeMap system is a considerable operational undertaking, and in many instances it will
strain the resources of local or regional networks. The USGS operates a fully-staffed,
24/7 operations center for global earthquakes and produces ShakeMaps on a daily basis.
If a regional network's parametric data can be made available to the USGS, the USGS's
global ShakeMap system will include that data in the processing, and finished ShakeMap
products can be delivered via ShakeCast for local consumption. Please contact us if you
are interested in this kind of arrangement.


\subsection{Seismic Stations and Parametric Data}
\label{software_guide:seismic-stations-and-parametric-data}
As mentioned above, seismic waveform or parametric data must be communicated to the
processing center in real-time or near---real--time. The sensors must be ``free field'' (i.e., not
located in, or adjacent to, large structures) and installed in accordance with modern standard
practices. Strong-motion instruments are preferred for ShakeMap, as broadband
instruments typically clip when ground motions reach the level of interest to ShakeMap.
Co-located strong-motion and broadband instruments can, however, increase the dynamic
range of a station and its overall usefulness, but the network operator is responsible for
specifying the crossover from broadband to strong motion, and ShakeMap should be
presented only with the amplitudes from the favored instrument. Off-scale, clipped,
below-noise, or otherwise unreliable data must be flagged or omitted from the ShakeMap
input files. Horizontal components are mandatory---ShakeMap does not use vertical
components.

At a minimum, the network must produce PGA and PGV for the ShakeMap stations, but
PSA (at 0.3, 1.0, and 3.0 sec) is also desirable. The algorithms
used to compute the parametric data should be verified against known standards.
ShakeMap can be quite helpful in
highlighting grossly mis-calibrated stations, but it is best to find these (and more subtle)
errors before a large earthquake strikes.

Intensity data from the USGS's DYFI system (or an equivalent
regional internet intensity system) is acceptable as input to ShakeMap; however, care
should be exercised with international DYFI data, as it is often aggregated by city and
therefore may be too coarse-grained for large-scale maps.


\subsection{Triggering ShakeMap}
\label{software_guide:triggering-shakemap}
The operator must give careful consideration to the way ShakeMap will be triggered.
Unless an operator is constantly standing by to run ShakeMap, the system must be
automatic to be useful. The operator must select the boundaries of the region ShakeMap
will cover and the minimum earthquake magnitude that will trigger a run. These choices
will be influenced by the reliability of the network's earthquake location and magnitude
data over the region in question. The operator must also consider that a large earthquake may
occur outside their region of responsibility, but have effects inside their region.

The ShakeMap software is distributed with a program for responding to event triggers
and queuing events to be processed, but it is only applicable to the AQMS network
software. While this software can serve as a guide, users of other systems (e.g., Antelope,
Earthworm) will need to develop their own triggering systems.

Once triggered, ShakeMap will expect to find event and parametric data files in the
event's input directory. Whether the amplitudes are retrieved from a database or stored in
files, it is the operator's responsibility to provide ShakeMap with properly formatted
files. Again, example programs are distributed with ShakeMap, but operators should
anticipate that some coding may be necessary.


\subsection{System Configuration}
\label{software_guide:system-configuration}
While it is relatively easy to install and run the ShakeMap software, a great deal of
consideration is needed to properly configure the system for a particular region.
ShakeMap presents the operator with a large number of configuration options, and these
options will directly affect the accuracy and reliability of the products. The
operator must select the proper GMPE or GMPEs for the region, and specify under which
conditions they will be used. An appropriate Ground-Motion--Intensity Conversion
Equation (GMICE) must likewise be chosen. The operator may also elect to use an Intensity
Prediction Equation (IPE) or to use the default virtual IPE. For best results, the operator
will need to provide a Vs30 grid, and while the \href{http://earthquake.usgs.gov/hazards/apps/vs30/}{USGS's Vs30 server} can supply such a
grid based on topographic slope, a grid based on regional geology is preferable \footnote[1]{\sphinxAtStartFootnote%
The VS30 server currently provides GMT \emph{grd} files in pixel node registration and
ShakeMap works in gridline node registration. You can fix your Vs30 file by:

grdsample your\_vs30\_grid.grd -Gnew\_file\_name.grd –T

You then configure \emph{grind.conf} to look at \emph{new\_file\_name.grd}.
See \emph{grind.conf} for details.
}. The
user must also decide whether to use GMPE-native or Borcherdt-style site correction
factors. This is just a partial list of configuration choices. There are numerous other
configurable parameters, all of which must be carefully chosen.

While the default ShakeMap webpages are configurable, they are fairly rudimentary and
outdated. In addition, the webpages and ShakeMap products are all in English. Operators
wishing more sophisticated webpages or non-English support can anticipate a substantial
investment in bringing the system online. Some modifications of the ShakeMap software
may be required for languages other than English, but such modifications may make it
more difficult to update the software.

For further information on ShakeMap configuration, see the {\hyperref[software_guide:sm35\string-software\string-guide]{\crossref{\DUrole{std,std-ref}{Software Guide}}}} and the
ShakeMap configuration files themselves.


\subsection{Operations}
\label{software_guide:operations}
Once an organization begins producing and distributing ShakeMaps, end users will begin
to depend upon them and develop systems that incorporate ShakeMap products into their
response and recovery operations. This means that it essentially becomes mandatory to
produce ShakeMaps following significant earthquakes, and failure to do so is
extremely conspicuous. A robust ShakeMap operation requires the 24/7 availability of
operational personnel. If the facility is not continuously manned, on-call staff must be
designated, and those staff must have the ability to access and operate the ShakeMap
system remotely. Significant earthquakes almost always require some manual
intervention (changing map scale, re-centering, addition of finite fault,
inclusion/exclusion of data, etc.), and experienced personnel are required to evaluate the
situation and perform the necessary tasks.

There are additional, more routine, operational considerations. An experienced
seismologist should routinely review all of the ShakeMaps produced by the system and
take action to correct any deficiencies. A network seismologist should also review the
inputs and outputs of ShakeMap to insure that all stations are producing appropriate data.
A ShakeMap operator should routinely review all ShakeMap processes, logs, databases,
and outputs to insure the system is operating as expected.

The ShakeMap software is usually updated a few times a year. These updates contain
important bug fixes, new functionality, new products, and general improvements. An
operator must review the change logs, decide when to apply the updates, and test the
updated software before it is put into production mode. Occasionally it may be desirable
to rerun earlier events or scenarios to take advantage of the capabilities of the new code.

Hardware and software systems will need to be monitored and maintained for around-the-
clock availability. This includes not just the seismic network and ShakeMap systems, but
also web servers and other network hardware and software required for delivering
products to end users. The personnel responsible for these systems must be on-call and
able to access the necessary systems remotely. Automatic monitoring of mission-critical
hardware and software is strongly encouraged. These systems should also have several
hours of backup power in case of an outage. Periodic outage tests should be
conducted to ensure that all necessary systems remain operational.

As mentioned above, users can be expected to make use of ShakeMaps in a variety of
ways. However, many organizations that could make use of ShakeMap products are
unaware of ShakeMap and the ways it could serve their earthquake response and
recovery needs. We have found that a sustained outreach effort is necessary to maximize
the adoption of ShakeMap and, thus, its value to society. Potential end users include
public utilities, government and private transportation companies, police and fire
departments, regional and national emergency response organizations, private companies
with distributed facilities (e.g., banks, chain stores, telecoms), insurance companies,
investment houses, and many others. Not only can ShakeMap-improved response efforts
benefit post-earthquake recovery, these organizations can provide much-needed support
for network and ShakeMap operations. It is highly recommended that regional networks
considering the implementation of ShakeMap develop a detailed outreach plan.


\subsection{Scenarios}
\label{software_guide:scenarios}
One important use of ShakeMap is the generation of earthquake scenarios. Scenarios are
predictive maps of the potential shaking resulting from hypothetical future (or past) earthquakes.
Scenarios can be used for planning exercises, public information, or research. Some
users may request specific scenarios, but it is generally worthwhile to develop a suite of
scenarios to cover the likely earthquake hazards of a region. At the USGS, we have begun
using disaggregated hazard maps as the basis for our nationwide scenario project. In other
words, we separate out the individual earthquakes (and causative faults) that together
comprise the hazard in a probabilistic hazard map. The disaggregated maps represent the
best scientific consensus of the probable earthquakes in a region, and should be sufficient
for most uses. Requests for custom scenarios should be carefully evaluated. The
earthquakes represented should be credible in terms of both the causative fault and the
magnitude. In most cases, one of the disaggregated hazard scenarios should suffice.


\subsection{Backup}
\label{software_guide:backup}
Because of the importance of ShakeMap, it is advisable to run redundant systems. Most
ShakeMap operations have a primary and backup machine. The backup machine runs
events as if it were the primary, except it does not transfer its products to the web or other
destinations. If the primary server fails, the backup can be switched over to primary
merely by changing the transfer configuration. This arrangement is also useful when
software updates are available. The update can be applied and tested on the backup
system. Once it is deemed to be operating correctly, it can be made primary, and the
primary server can be updated.

Since most seismic networks are operated from earthquake-prone regions, there is also
the potential that the entire facility will be taken offline. For this reason, it is desirable to
have a backup system operating in a remote location, preferably many kilometers away.

As we have mentioned elsewhere, the USGS makes ShakeMaps for global earthquakes
and provides backup to U.S. regional networks. If you would like to discuss remote
backup for your ShakeMap system, please contact us.


\section{ShakeMap Implementation Checklist}
\label{software_guide:shakemap-implementation-checklist}
The checklist below is based on the one we use when discussing ShakeMap operations with active or
potential producers within the USGS's Advanced National Seismic System (ANSS).
While some of the issues are ANSS-specific, there may be analogous considerations for
other regional or national networks.
\begin{enumerate}
\item {} 
\textbf{Triggering}
\begin{enumerate}
\item {} 
Automatic Triggering System.  How is ShakeMap triggered and how does it
access or receive parametric data?  How is robustness of this approach
achieved?

\item {} 
Location \& Magnitude Reliability.  Are there limitations to location and
magnitude determination by the regional network that would adversely affect
automatic ShakeMap products?

\item {} 
Regional Coverage.  What are the boundaries of the area within which the
local network will generate ShakeMaps?

\item {} 
Alarm Region.  For events outside ShakeMap boundaries, is a ShakeMap run
initiated?  Under what conditions?

\item {} 
ShakeMap ID.  Does the naming of ShakeMap ID follow the ANSS
convention?  If not, can they be easily associated with the authoritative ID?

\end{enumerate}

\item {} 
\textbf{Station Coverage and Parametric Data}
\begin{enumerate}
\item {} 
Real-time or near--real-time data flow.  What are the types and distribution of
stations contributing to ShakeMap? Are all stations ``ShakeMap-quality”?

\item {} 
Parametric Data.  How are the parametric data computed? (Five parameters:
PGA, PGV, and three periods of PSA.)

\item {} 
Are parametric data imported from other sources (NSMP-triggered stations,
state or commercial agencies, neighboring networks, etc.)? How are these
integrated with the ShakeMap input?

\item {} 
Are ``Did You Feel It?'' data used as input?

\item {} 
Co-location of different sensor types, priorities, and preventing redundant
input data. How are co-located instruments resolved by the network to
produce only a single (best) set of amplitudes for ShakeMap?

\end{enumerate}

\item {} 
\textbf{System Specifications}
\begin{enumerate}
\item {} 
Grind parameters. Review the parameters in \emph{grind.conf}. How were they
determined?
\begin{enumerate}
\item {} 
GMPEs. Which Ground-Motion Prediction Equations are used, and
under what conditions?

\item {} 
IPEs. Which Intensity Prediction Equations are used, and under what
circumstances?

\item {} 
GMICEs. Which Ground-Motion--Intensity Conversion Equations are
used?

\item {} 
Site Amplification.  How are site conditions established and what
amplifications are used (GMPE-native, Borcherdt-style)?

\item {} 
Other parameters. Grid spacing, map area, outlier levels, bias
parameters. Have all parameters been evaluated for optimal
performance?

\item {} 
\emph{Shake.conf}. When is map size increased, PSA and HAZUS output
produced, etc.?

\end{enumerate}

\item {} 
Spatial Correlation Function. Which spatial correlation function is used?

\item {} 
Basin response. Is a basin response applied in any areas? If so, how was the
basin depth file produced, and are predicted ground motions consistent with
reality?

\end{enumerate}

\item {} 
\textbf{Operations}
\begin{enumerate}
\item {} 
Which version of ShakeMap is operational? Who is responsible for updating
the software when updates are released? When and how are the updates performed?

\item {} 
Who is responsible for routine scientific review of ShakeMaps produced by
the network? Do these people receive alarms when ShakeMaps are produced?

\item {} 
Who is responsible for routine operational review of the ShakeMap system
(checking logs, process and database monitoring, etc.)? When are reviews
performed?

\item {} 
Reprocessing. Under what circumstances are events reprocessed (new data,
change in source parameters, etc.)? What about in the longer term (ShakeMap
software updates, changes in operational parameters)?

\item {} 
Finite faults. For larger earthquakes, who is responsible for producing a finite
fault model for inclusion in ShakeMap? What procedures are in place for
assuring this is done?

\item {} 
Aftershock exclusion. How will you change the triggering threshold
immediately after a major earthquake in your region?

\item {} 
Version history. Under what circumstances are maps (and their input data)
preserved using ShakeMap versioning?

\item {} 
Have there been any local changes to the ShakeMap software that will hinder
upgrades? Can these customizations be incorporated into the ShakeMap
distribution for easier upgrades? If not, how can they be structured to
accommodate easy upgrades of ShakeMap?

\item {} 
What is the hardware for ShakeMap processing and for local web service?

\item {} 
How is hardware redundancy achieved?

\item {} 
Are the hardware and software systems automatically monitored? Do they
generate alerts when problems are detected?

\end{enumerate}

\item {} 
\textbf{Product Distribution and Uniformity}
\begin{enumerate}
\item {} 
Are products delivered to Earthquake Program Web Servers via PDL?

\item {} 
Are local webpages produced? Where do they reside? How is ShakeMap
transferred? Are redundant web servers and 24/7 support available?

\item {} 
Are regional ShakeMap webpages customized to reflect regional
configurations and implementation specifics?

\end{enumerate}

\item {} 
\textbf{ANSS Coordination}
\begin{enumerate}
\item {} 
Provide Software/Feedback to ANSS.  To benefit current operators and to
ensure compatibility and ease of installing new ShakeMap software releases,
changes to ShakeMap software (above and beyond configuration changes)
should be provided to Bruce Worden for review, standardization, and
inclusion in new releases.

\item {} 
Provide contacts, their background, and roles in implementation, coordination,
and operations.

\item {} 
Are all responsible parties subscribed to the \emph{shake-dev} mailing list?

\end{enumerate}

\item {} 
User Coordination:
List significant users and outline any outreach efforts or plans. It is very useful to
have a feeling for which users will rely on ShakeMap in each region, as well as to
coordinate efforts for users of ShakeMaps for multiple regions (e.g., FEMA,
DHS, Military).

\item {} 
\textbf{Scenarios and Archives}
\begin{enumerate}
\item {} 
Scenario earthquakes should be made to be consistent with USGS National
Hazard Maps, both with attenuation relations and in source parameterization.
Coordination with the National Earthquake Information Center (NEIC) is essential.

\item {} 
Is a copy of scenarios also available on the USGS web site?

\item {} 
How and when will scenarios be reprocessed?

\item {} 
Archive “final'' ShakeMaps for significant events.  Many users want
ShakeMaps for significant events ``frozen in time''. Once a ShakeMap gets
used as a reference for damage-loss modelers, insurance investigators, and
researchers, there needs to be an archival version of these events. Once all the
available ground-motion data have been collected and included in ShakeMap,
that Version of the map needs to be kept available even if additional updates
are made. (This process has not yet been fully vetted.)

\end{enumerate}

\item {} 
\textbf{Backup Strategy}
\begin{enumerate}
\item {} 
If the primary system fails, what provisions exist for a backup system or
another network to take over ShakeMap operations? Is this backup automatic
or manual?

\item {} 
If the entire facility goes offline, is there an off-site backup?

\item {} 
Are waveform or parametric data transmitted to NEIC for national-level
backup?

\end{enumerate}

\item {} 
\textbf{Feedback}:
Do you have any recommendations for further support, software, features, etc.?

\end{enumerate}


\section{Software Availability \& Software Guide}
\label{software_guide:software-availability-software-guide}\label{software_guide:sm35-software-guide}
ShakeMap requires the freely available PERL, MySQL, and GMT (Generic Mapping Tools),
as well as a few other packages. PERL and GMT are used quite extensively, so any background
with them is advantageous. You will need to assemble the basic GMT-formatted basemaps,
road, city data files, etc., but such data may already be available for your area.

The ShakeMap software is freely available, open-source, and distributed under a Public
Domain License. It runs on Solaris, FreeBSD, Mac OS X, (U)nix, and numerous versions
of Linux (including Red Hat and Debian). It
does not run on Windows. See the Software Guide for more information. The software
is available as a \href{http://subversion.apache.org/}{SubVersion} checkout from:

\url{https://vault.gps.caltech.edu/repos/products/shakemap/tags/release-3.5/}

\begin{notice}{note}{Note:}
Do not attempt to install ShakeMap on Ubuntu Linux. It has been nothing
but a problem for everyone who has tried it, and we will no longer provide
support for this operating system.
\end{notice}

The Software Guide included in the \emph{doc} directory of the distribution will always be the
most up-to-date and should be consulted when installing and configuring ShakeMap. The
Software Guide may also be obtained by \href{http://usgs.github.io/shakemap/\_static/SoftwareGuideV3\_5.pdf}{download}.
This version of Guide is not guaranteed to be the most up-to-date, however. It should be
used only to familiarize oneself with the general requirements of installing and operating
ShakeMap. When installing the software, the Guide in the \emph{doc} directory of the software
distribution should be followed.

We strongly recommend that ShakeMap operators and users sign up for the \emph{shake-dev} mailing list:

\url{https://geohazards.usgs.gov/mailman/listinfo/shake-dev}

We use this mailing list to communicate software updates, as well as provide support
when users have problems, suggestions, etc.


\chapter{Regional Operations}
\label{regionals::doc}\label{regionals:regional-operations}\label{regionals:sec-regionals}
\begin{notice}{note}{Note:}
This section will be updated with Regional ShakeMap specifications after input from the
ANSS Regional Seismic Networks operators.
\end{notice}

As described in the section on real-time ShakeMap archives, ShakeMaps are generated via independent systems running at ANSS
Regional Seismic Systems (RSNs) in Northern California, Southern California, the
Pacific Northwest, Utah, Nevada, and Alaska. For the rest of the U.S., the
ShakeMap group at the USGS National Earthquake Information Center (NEIC)
produces maps for the regional networks operating in Hawaii, New England, and
the Central and Eastern U.S. on a system referred to as Global ShakeMap (GSM).
The input, metadata, and output files produced by all these instances are
aggregated by the USGS via the Earthquake Hazards Program web system. GSM also provides
backup capabilities for the RSNs, but with degraded capabilities; not all data
are flowing from the RSNs to GSM automatically.

Separate independent systems running in Puerto Rico and New England generate
ShakeMaps, but these instances do not deliver them through the USGS Earthquake
Hazards Program webpages (at the time of this writing). GSM covers these regions, but
does not yet access the full set of data available to these regional
systems.

In this section, we describe customizations employed by ShakeMap systems running
throughout the Advanced National Seismic System (ANSS) regions
nationwide as well as the Global ShakeMap (GSM) system running at the NEIC in
Golden, Colorado.

\begin{notice}{note}{Note:}
Specifications of input parameters, data, and other configurations
used for any ShakeMap can be found in the event-specific summary
files (\emph{info.xml}).
\end{notice}

Details for about regional configurations for ShakeMap operators can
be found in the {\hyperref[software_guide:software\string-guide]{\crossref{\DUrole{std,std-ref}{Software \& Implementation Guide}}}}. For contact information for
ANSS regional operators see the {\hyperref[acknowledgments:acknowledgments]{\crossref{\DUrole{std,std-ref}{Acknowledgments}}}}.
\begin{itemize}
\item {} 
\textbf{Northern California}

\item {} 
\textbf{Southern California}

\item {} 
\textbf{Pacific Northwest}

\item {} 
\textbf{Intermountain West}

\item {} 
\textbf{Mid-America}

\item {} 
\textbf{Northeast}

\item {} 
\textbf{Alaska}

\item {} 
\textbf{Puerto Rico}

\item {} 
\textbf{Hawaii}
\begin{itemize}
\item {} 
\emph{Coverage Area}. State of Hawaii (Bounds:)

\item {} 
\emph{Operations}. ShakeMap operated at NEIC in conjunction with HVO RSN operators

\item {} 
\emph{Triggering and Data Flow}.

\item {} 
\emph{Site Condition Map}. Vs30 from topographic slope-based ({\hyperref[references:allen2009b]{\crossref{\DUrole{std,std-ref}{Allen and Wald, 2009b)}}}}

\item {} 
\emph{Ground Motion Prediction and Conversion Equations (GMPE/IPE/GMICE)}.

\item {} 
\emph{Other Local Characteristics}.

\end{itemize}

\end{itemize}


\chapter{Future Directions}
\label{future_directions::doc}\label{future_directions:id1}\label{future_directions:future-directions}
ShakeMap is a continual work-in-progress. We note several ongoing developments and ``To-Do'' lists. Please make suggestions if you would like to weigh in.


\section{Research \& Development}
\label{future_directions:research-development}
\textbf{Feature Requests:}
\begin{itemize}
\item {} 
Add NGA-West2, NGA-East, and NGA-Subduction GMPEs, including basin terms for NGA-West 2 GMPEs.

\item {} 
Improved and additional site amplification approaches and tables, in addition to native GMPE (Vs30) site corrections (e.g., {\hyperref[references:seyhan2014]{\crossref{\DUrole{std,std-ref}{Seyhan and Stewart, 2014}}}}).

\item {} 
R\&D to improve PGV-to-MMI conversion for large-magnitude and high-velocity ranges. May require switch to converting long-period spectral acceleration
to MMI. Simulated ground motion time histories may be useful to augment sparse data at high PGV/MMI.

\item {} 
Consideration of vector-component IMs, static displacements, and duration-based IMs (Arias Intensity; Cumulative Average Velocity, or CAV) \footnote[1]{\sphinxAtStartFootnote%
We are continuously considering the use of additional ground-motion parameters (IMs)
for ShakeMap. However, any such additions cannot be made lightly. In part, this is
due to the fact that this requires upgrading process seismic network processing streams
that produce parametric and these processes vary significantly among ANSS data sources.
}

\end{itemize}

\textbf{In Progress:}
\begin{itemize}
\item {} 
Spatial variability. Implement optimization methods to compute the spatial correlation field for ShakeMap
using successive conditional simulations (Verros et al., 2016).
Deliver ShakeMap scenarios with multiple realizations of variability.

\item {} 
Directivity. Update Rowshandel (2010) model and implement selected NGA-West2 models.

\item {} 
Landslide and liquefaction likelihood grid (\emph{sechaz.grid.xml}). Computing probability and distribution of landsliding and liquefaction per
ShakeMap grid cell. Delivery via Product Distribution Layer (PDL) for ShakeCast, PAGER, and open access.

\item {} 
Scenario update for all U.S. regions. Delivery to ComCat/Web and allow users a variety of search capabilities (site- or fault-specific).

\item {} 
Interactive (dynamic) webpages plots (regression, bias, outliers, station amplitudes).

\item {} 
Improved content and rendering of ShakeMap metadata (\emph{info.xml}; see {\hyperref[references:thompson2016]{\crossref{\DUrole{std,std-ref}{Thompson et al., 2016}}}}).

\end{itemize}


\section{Software: ShakeMap 4.0 (Python)}
\label{future_directions:software-shakemap-4-0-python}
The release of ShakeMap version 4.0 will represent a significant departure from
previous versions. All of the important computational modules have been
refactored into the Python programming language and make use of the tools in
the widely available Python “scientific distribution”. The core ShakeMap code,
approaching fifteen years old, was due for a major overhaul---to more
organically incorporate the many extensions that had been added over its
lifetime, and to facilitate several new demands from ShakeMap software and
ShakeMap’s expanded role as a global provider of post-earthquake information,
earthquake scenarios, and inputs for loss-modeling software.

One of the advantages of the rewrite of ShakeMap into the Python language was
the availability of the \href{http://www.globalquakemodel.org/openquake/about/}{GEM OpenQuake}
(OQ) library of Ground Motion Prediction
Equations (GMPEs). The OQ hazard library provided us with a broad range of
well-tested, high-performance, open-source GMPEs. Due to constraints imposed by
the software architecture of earlier implementations of ShakeMap, the
development of GMPE modules was time-consuming and difficult, which restricted
the number and timeliness of the available modules. The OQ library provides a
broad array of current GMPE modules, as well as a framework for easily adding
new modules (whether by GEM or ShakeMap staff), jumpstarting our efforts to
re-implement ShakeMap.

The OQ hazard library also provides supporting functions for using the GMPE
modules, including a set of software classes for computing the various distance
measures required by the GMPEs. The ShakeMap fault model, however, was somewhat
more general than allowed for by the OQ planar surface modules, so we
sub-classed the OQ “surface” class and implemented our own fault
module. The open-source, cooperative nature of the OQ project allowed us to
contribute our new module back to the OQ repository, and thus make it available
to other users.

In addition to making use of the GEM OQ library, there are a number of other
advantages to using Python for an application like ShakeMap.  The dynamic
nature of the language means that development time is much reduced, allowing a
small team to generate useful code in a short amount of time.  Also, there is
an active scientific computing Python community that has created many tools
that solve common problems, including an array object for vectorized
operations, input/output routines for common data formats, and plotting/mapping
libraries.  These tools further help to reduce development time and effort.
{[}\textbf{Delivery Date: 2016}{]}


\chapter{Acknowledgments}
\label{acknowledgments::doc}\label{acknowledgments:acknowledgments}\label{acknowledgments:id1}
Many contributions in a variety of forms have greatly helped in the development,
implementation, and use of ShakeMap. ShakeMap is just one end
product of a very sophisticated seismic network. Credit is given to all involved
with the regional and national networks in the United States working under the
auspices of the USGS Advanced National Seismic System (ANSS).

Much of the early conceptual development of ShakeMap benefitted greatly from
discussions with Professors Kanamori and Heaton at Caltech. Both the TriNet
Steering and Advisory Committees also provided ongoing oversight and feedback in
the early years of TriNet. Discussions with many colleagues, including Woody
Savage, Ken Campbell, Robert Nigbor, and Mark Petersen, provided additional guidance.
Early trips to the Japanese Meteorological Agency (JMA), and in particular
discussions with Keiji Doi, were also  helpful.

Early ShakeMap webpages survived substantial traffic spikes due to the
ingenuity and vigilance of Stan Schwarz (USGS, Pasadena). Many aesthetic improvements
and integration of the ShakeMap webpages into the USGS Earthquake Hazards Team
webpage standard templates were guided by Lisa Wald (USGS, Golden). Later
implementations were greatly improved by the Haz-Dev Web Team, particularly by
Jeremy Fee and Eric Martinez. They also provide the GeoJSON and other feeds
needed for automatic ShakeMap downloads.

Craig Scrivner, then at the CA Department of Mines \& Geology (CDMG),
contributed greatly to the initial ShakeMap software development. Pete
Lombard (U.C. Berkeley) contributed to several important aspects
of ShakeMap code development and Q\&A, including developing \emph{plot\_regr}.
\setbox0\vbox{
\begin{minipage}{0.95\linewidth}
\textbf{ANSS National ShakeMap Coordinators:}

\medskip

\begin{itemize}
\item {} 
Bruce Worden: Synergetics Inc., Golden, Colorado, \href{mailto:cbworden@usgs.gov}{cbworden@usgs.gov}

\item {} 
David Wald: USGS, Golden, Colorado, \href{mailto:wald@usgs.gov}{wald@usgs.gov}

\end{itemize}
\end{minipage}}
\begin{center}\setlength{\fboxsep}{5pt}\shadowbox{\box0}\end{center}

Global ShakeMap (GSM) is run out of NEIC, Golden, CO, and is operated
primarily by Kuo-Wan Lin, Kristin Marano, Vince Quitoriano, Eric Thompson,
David Wald, and Bruce Worden.

At regional network centers, Kris Pankow (University of Utah), Steve Malone
(University of Washington), Kuo-Wan Lin (formerly CGS, now USGS, Golden),
Doug Dreger, Pete Lombard (U.C. Berkeley), and Lind Gee (USGS, Menlo Park),
Egill Hauksson (Caltech), Glen Biasi (Univ Nevada, Reno), and
Howard Bundock (ret.), Tim MacDonald, David Oppenheimer, and John Boatwright
(USGS, Menlo Park) all played a critical role in testing, providing
feedback, and improving the ShakeMap system. In addition, a number of other
people assisted the above colleagues in the regional ShakeMap implementation and
operation.
\begin{quote}

\begin{tabulary}{\linewidth}{|L|L|}
\hline
 \multicolumn{2}{|l|}{
\textbf{ShakeMap ANSS Regional Coordinators}
}\\
\hline \multirow{2}{*}{
Southern California:
} & 
Robert Graves, \href{mailto:rgraves@caltech.edu}{rgraves@caltech.edu}
\\
\cline{2-2} & 
Gary Gann, \href{mailto:agann@usgs.gov}{agann@usgs.gov}
\\
\hline \multirow{4}{*}{
Northern California:
} & 
John Boatwright, \href{mailto:boat@usgs.gov}{boat@usgs.gov}
\\
\cline{2-2} & 
Peggy Hellweg, \href{mailto:peggy@seismo.berkeley.edu}{peggy@seismo.berkeley.edu}
\\
\cline{2-2} & 
Tim MacDonald, \href{mailto:tmacdonald@usgs.gov}{tmacdonald@usgs.gov}
\\
\cline{2-2} & 
Pete Lombard, \href{mailto:lombard@seismo.berkeley.edu}{lombard@seismo.berkeley.edu}
\\
\hline
Utah:
 & 
Kris Pankow, \href{mailto:pankow@seis.utah.edu}{pankow@seis.utah.edu}
\\
\hline \multirow{2}{*}{
Alaska:
} & 
Natalia Ruppert, \href{mailto:natasha@gi.alaska.edu}{natasha@gi.alaska.edu}
\\
\cline{2-2} & 
Matt Gardine, \href{mailto:mgardin2@alaska.edu}{mgardin2@alaska.edu}
\\
\hline \multirow{2}{*}{
Pacific Northwest:
} & 
Paul Bodin, \href{mailto:bodin@uw.edu}{bodin@uw.edu}
\\
\cline{2-2} & 
Renate Hartog, \href{mailto:jrhartog@uw.edu}{jrhartog@uw.edu}
\\
\hline \multirow{2}{*}{
Nevada:
} & 
Ken Smith, \href{mailto:ken@seismo.unr.edu}{ken@seismo.unr.edu}
\\
\cline{2-2} & 
Gabe Plank, \href{mailto:gabe@seismo.unr.edu}{gabe@seismo.unr.edu}
\\
\hline
Puerto Rico:
 & 
Victor Huerfano, \href{mailto:victor@prsn.uprm.edu}{victor@prsn.uprm.edu}
\\
\hline\end{tabulary}

\end{quote}

We received extremely important feedback regarding the user interface from
participants through a number of meetings and workshops in CA for
scientific and engineering perspectives, as well as for a very wide variety of
users’ perspectives. These workshops were organized by Jim Goltz and Margaret
Vinci. In addition, ongoing feedback has always been abundant and provides
critical advice and ideas that seeds ongoing, iterative improvements to the
ShakeMap system.

Over the years, numerous student researchers contributed greatly to the operation,
development, user support and many other aspects of ShakeMap, particularly
developing the ShakeMap Atlas. The students include undergraduates Chloe Gustafson,
Paul Geimer, Alicia Hotovec, Kendra Johnson, Rebecca Johnson, Russell Mah, Tanya Slota;
graduate students Lily Seidman, Sarah Verros, Leslie McWhirter;
and post-doctoral fellows Trevor Allen and Daniel Garcia.

International collaborators include Alberto Michelini, and Licia Faenza (INGV,
Rome) and Carlo Cauzzi (ETH Zurich). ShakeMap Workshops help at U.C. Berkeley
and at the Erice, Sicily Summer School were very helpful in improving ShakeMap
implementations around the country and around the world.

Doug Bausch (formerly FEMA) played a vital role in standardizing HAZUS loss
estimates using ShakeMap input, responding to numerous national earthquake
responses, and promoting the use of ShakeMap among the earthquake
response and planning communities. In interfacing with HAZUS software
and loss estimates, contributions from Jahar Bouabid at
Durham Technology, and Charlie Kircher of Charlie Kircher Assoc. were key.

USGS colleagues Vince Quitoriano and Kuo-Wan Lin have long operated,
supported, and greatly added to the development of ShakeMap. Mike Hearne and
Kristin Marano of the PAGER team have also provided support for ShakeMap. Eric Thompson
has contributed greatly to the development of scenarios and advancing ShakeMap
science. Kishor Jaiswal, Ned Field, Nico Luco, Mark Petersen have all been very helpful
in software calibration and validation and overall advice.

Eric Thompson (USGS, Golden) and Emmette Wald provided helpful rewivews of this Manual.

Lastly, we are also extremely grateful for the recognition of the
importance of ShakeMap and the ongoing internal and external support for its
development at all levels within the USGS. Along the way, the support of
John Filson, David Applegate, Bill Leith, Harley Benz, Cecily Wolfe, Woody Savage, Jill
McCarthy, and many others has been critical.

ShakeMap relies extensively on the Generic Mapping Tools
({\hyperref[references:gmt\string-ref]{\crossref{\DUrole{std,std-ref}{Wessel and Smith, 1995}}}}).

Thank you all.

Bruce Worden \& David Wald, December 2015


\chapter{References \& Bibliography}
\label{references::doc}\label{references:references}\label{references:references-bibliography}\phantomsection\label{references:abrahamson2008}
Abrahamson, N.A. and W.J. Silva (2008). Summary of the Abrahamson \& Silva NGA ground motion
relations, \emph{Earthquake Spectra} 24, 67-97.

Abrahamson, N., G. Atkinson, D. Boore, Y. Bozorgnia, K. Campbell, B. Chiou, B., I.M. Idriss, W. Silva,
and R. Youngs (2008).  Comparisons of the NGA ground-motion relations, \emph{Earthquake Spectra} 24(1), 45-66.

\phantomsection\label{references:abrahamson2014}
Abrahamson, N.A., W.J. Silva, and R. Kamai (2014). Summary of the ASK14 Ground Motion Relation
for Active Crustal Regions, \emph{Earthquake Spectra} 30(3), 1025-1055.

Akkar, S. and J.J. Bommer (2010). Empirical equations for the prediction of PGA, PGV and spectral
accelerations in Europe, the Mediterranean region and the Middle East, \emph{Seis. Res. Lett.} 81(2), 195-
206.

\phantomsection\label{references:akkar2014}
Akkar, S., M.A. Sandikkaya, and J.J. Bommer (2014). Empirical ground-motion models for point
and extended-source crustal earathquake scenarios in Europe and the Middle East, \emph{Bull.
Earthquake Eng.} 12, 359-387.

\phantomsection\label{references:allen2006}
Allen, R.M. (2006). Probabilistic Warning Times for Earthquake Ground Shaking in the San Francisco
Bay Area, \emph{Seis. Res. Lett.} 77(3), 371-376.

\phantomsection\label{references:allen2008}
Allen, T.I., D.J. Wald, A.J. Hotovec, K.W. Lin, P.S. Earle, and K.D. Marano (2008). An
Atlas of ShakeMaps for selected global earthquakes, \emph{U.S. Geological Survey Open-File Report 2008-1236}, 35pp.

\phantomsection\label{references:allen2009a}
Allen, T.I., D.J. Wald, P.S. Earle, K.D. Marano, A.J. Hotovec, K.W. Lin, and M.J. Hearne (2009a). An
Atlas of ShakeMaps and population exposure catalog for earthquake loss modeling, \emph{Bull.
Earthquake Eng.} 7, 701-718.

\phantomsection\label{references:allen2009b}
Allen, T. and D.J. Wald (2009b). On the use of high-resolution topographic data as a proxy for seismic
site conditions (VS30), \emph{Bull. Seism. Soc. Am.} 99(2A), 935-943.

Allen, T.I., K.D. Marano, P.S. Earle, and D.J. Wald (2009c). PAGER-CAT: a composite earthquake
catalogue for calibrating global fatality models, \emph{Seis. Res. Lett.} 80(1), 57-62.
DOI: 10.1785/gssrl.80.1.57.

\phantomsection\label{references:allen2012}
Allen, T.I., D.J. Wald, and C.B. Worden (2012). Intensity attenuation for active crustal regions, \emph{J.
Seismol.} 16, 409-433.

Anderson, J.G. (2013). Surface Motions on Near-Distance Rock Sites in the 2011 Tohoku-Oki Earthquake,
\emph{Earthquake Spectra} 29(S1), S23-S35

Atkinson, G.M. (2008). Ground-motion prediction equations for eastern North America from a referenced
empirical approach: Implications for epistemic uncertainty, \emph{Bull. Seism. Soc. Am.} 98(3), 1304-1318.

Atkinson, G.M. (2010). Ground-motion prediction equations for Hawaii from a referenced empirical
approach, \emph{Bull. Seism. Soc. Am.} 100(2), 751-761.

Atkinson, G.M. and D.M. Boore (2003). Empirical ground-motion relations for subduction zone
earthquakes and their application to Cascadia and other regions, \emph{Bull. Seism. Soc. Am.} 93(4), 1703-
1729.

Atkinson, G.M. and D.M. Boore (2006). Earthquake ground-motion prediction equations for Eastern North
America, \emph{Bull. Seism. Soc. Am.} 96(6), 2181-2205.

Atkinson, G.M. and D.M. Boore (2008). Erratum: Empirical ground-motion relations for subduction zone
earthquakes and their application to Cascadia and other regions, \emph{Bull. Seism. Soc. Am.} 98(5), 2567-2569.

Atkinson, G.M. and D.M. Boore (2011). Modifications to existing ground-motion prediction equations in
light of new data, \emph{Bull. Seism. Soc. Am.} 101(3), 1121-1135.

\phantomsection\label{references:atkinson2007}
Atkinson, G.M. and S.I. Kaka (2007). Relationships between Felt Intensity and Instrumental Ground
Motion in the Central United States and California, \emph{Bull. Seism. Soc. Am.} 97, 497-510.

Atkinson, G.M. and M. Macias (2009). Predicted ground motions for great interface earthquakes in the
Cascadia subduction zone, \emph{Bull. Seism. Soc. Am.} 99(3), 1552-1578.

\phantomsection\label{references:atkinson-wald2007}
Atkinson, G.M. and D.J. Wald (2007). ``Did You Feel It?'' Intensity Data: A surprisingly good measure
of earthquake ground motion, \emph{Seis. Res. Lett.} 78, 362-368.

Atkinson, G.M. and D.M. Boore (2003). Empirical ground-motion relations for subduction regions and
their application to Cascadia and other regions, \emph{Bull. Seism. Soc. Am} 93, 1703-1729.

Atkinson, G.M. and D. M. Boore (1997). Some comparisons between Recent ground-motion relations,
\emph{Seis. Res. Lett.} 68, 24-40.

Atkinson, G.M. and D.M. Boore (1995). Ground motion relations for eastern North America, \emph{Bull. Seism. Soc. Am.} 85, 17-30.

\phantomsection\label{references:atc2002}
Applied Technology Council (2002). ATC-54: Guidelines for using strong-motion data and ShakeMaps in
Post-Earthquake Response, Redwood City, California, 224pp.

Boatwright, J., H. Bundock, J. Luetgert, L. Seekins, L. Gee, and P. Lombard (2003). The dependence of
sPGA and PGV on distance and magnitude inferred from Northern California ShakeMap data, \emph{Bull.
Seism. Soc. Am.} 93(5), 2043-2055.

Boatwright, J., K. Thywissen, and L. Seekins (2001). Correlation of ground-motion and intensity for the
January 17, 1994, Northridge, California earthquake, \emph{Bull. Seism. Soc. Am.} 91, 739-752.

\phantomsection\label{references:beyer2006}
Beyer, K. and J. Bommer (2006). Relationships between Median Values and between Aleatory
Variabilities for Different Definitions of the Horizontal Component of Motion, \emph{Bull. Seism. Soc. Am.} 96(4A), 1512-1522.

\phantomsection\label{references:biasi2016}
Biasi, G., M.S. Mohammed, and D.H. Sanders (2016). Earthquake Damage
Estimations: ShakeCast Case Study on Nevada Bridges, \emph{Earthquake
Spectra}, submitted.

\phantomsection\label{references:bommer2012}
Bommer, J.J. and S. Akkar (2012). Consistent source-to-site distance metrics in ground-motion prediction
equations and seismic source models for PSHA, \emph{Earthquake Spectra} 28, 1-15.

\phantomsection\label{references:ba2008}
Boore, D.M. and G.M. Atkinson (2008). Ground-Motion Prediction Equations for the Average
Horizontal Component of PGA, PGV, and 5\%-Damped PSA at Spectral Periods between 0.01 s and
10.0 s, \emph{Earthquake Spectra} 24(1), 99-138.

Boore, D.M., J.F. Gibbs, W.B. Joyner, J.C. Tinsley, and D.J. Ponti (2003). Estimated Ground Motion
From the 1994 Northridge, California, Earthquake at the Site of the Interstate 10 and La Cienega
Boulevard Bridge Collapse, West Los Angeles, California, \emph{Bull. Seism. Soc. Am.} 93, 2737-2751.

Boore, D.M., W.B. Joyner, and T.E. Fumal (1997). Equations for Estimating Horizontal Response Spectra
and Peak Accelerations from Western North American Earthquakes: A Summary of Recent Work,
\emph{Seis. Res. Lett.} 68, 128-153.

Boore, D.M., W.B. Joyner, and T.E. Fumal (1994). Estimation of response spectra and peak accelerations
from Western North America Earthquakes: An Interim Report, Part 2, \emph{U. S. Geological Survey Open-
File Report 94-127}, 40pp.

Boore, D.M., W.B. Joyner, and T.E. Fumal (1997). Equations for estimating horizontal response spectral
and peak acceleration from western North American earthquakes: A summary of recent work, \emph{Seis.
Res. Lett.} 68, 128-153.

Boore, D.M. and W.B. Joyner (1991). Estimation of ground motion at deep-soil sites in eastern North
America, \emph{Bull. Seism. Soc. Am.} 81(6), 2167-2185.

\phantomsection\label{references:boore2010}
Boore, D.M. (2010). Orientation-Independent, Nongeometric-Mean Measures of Seismic Intensity
from Two Horizontal Components of Motion, \emph{Bull. Seism. Soc. Am.} 100(4), 1830–1835.

\phantomsection\label{references:borcherdt1994}
Borcherdt, R.D. (1994). Estimates of site-dependent response spectra for design (methodology and
justification), \emph{Earthquake Spectra} 10, 617-654.

Brackman, T. (2005). ShakeMap Implementation for the Upper Mississippi Embayment, Thesis, University
of Memphis, Department of Earth Sciences.

Campbell, K.W. (2002). Prediction of strong ground motion using the hybrid empirical method: example
application to eastern North America, \emph{Bull. Seism. Soc. Am.}, submitted.

Campbell, K.W. (1997). Empirical near-source attenuation relationships for horizontal and vertical
components of peak ground acceleration, peak ground velocity, and pseudoabsolute acceleration
response spectra, \emph{Seis. Res. Lett.} 68, 154-179.

Campbell, K.W. (2003). Prediction of strong ground motion using the hybrid empirical method and its use
in the development of ground-motion (attenuation) relations in eastern North America, \emph{Bull. Seism. Soc.
Am.} 93(3), 1012-1033.

Campbell, K.W. and Y. Bozorgnia (2007). Campbell-Bozorgnia NGA ground motion relations for the
geometric mean horizontal component of peak and spectral ground motion parameters, \emph{PEER Report
No. 2007/02}, Pacific Earthquake Engineering Research Center, University of California, Berkeley.

Campbell, K.W., and Y. Bozorgnia (2008). NGA ground motion model for the geometric mean horizontal
component of PGA, PGV, PGD and 5\% damped linear elastic response spectra for periods ranging from
0:01 to 10 s., \emph{Earthquake Spectra} 24(1), 139-171.

\phantomsection\label{references:caprio2015}
Caprio, M., B. Tarigan, C.B. Worden, D.J. Wald, and S. Wiemer (2015). Ground Motion to Intensity
Conversion Equations (GMICEs): A Global Relationship and Evaluation of Regional Dependency,
\emph{Bull. Seism. Soc. Am.} 105(3).

\phantomsection\label{references:celsi2005}
Celsi, R., M. Wolfinbarger, and D.J. Wald (2005). The Effects of Magnitude Anchoring, Earthquake Attenuation Estimation, Measure Complexity, Hubris, and Experience Inflation on Individuals’ Perceptions of Felt Earthquake Experience and Perceptions of Earthquake Risk, \emph{Earthquake Spectra} 21(4), 987-1008.

Chiou, B.S.J. and R.R. Youngs (2008a). An NGA model for the average horizontal component of peak
ground motion and response spectra, \emph{Earthquake Spectra} 24(1), 173-215.

Chiou, B.S.J. and R.R. Youngs (2008b). Chiou and Youngs PEER-NGA empirical ground motion model
for the average horizontal component of peak acceleration, peak velocity, and pseudo-spectral
acceleration for spectral periods of 0.01 to 10 seconds, Final Report submitted to PEER.

Choi, Y. and J.P. Stewart (2005). Nonlinear Site Amplification as Function of 30 m Shear
Wave Velocity, \emph{Earthquake Spectra} 21(1), 1-30.

\phantomsection\label{references:converse1992}
Converse, A. and A.G. Brady (1992). BAP basic strong-motion accelerogram processing software
version 1.0, \emph{U.S. Geological Survey Open-File Report 92-296}.

\phantomsection\label{references:convertito2011}
Convertito, V., M. Caccavale, R. De Matteis, A. Emolo, D.J. Wald, and A. Zollo (2011). Fault extent
estimation for near-real time ground shaking map computation purposes, \emph{Bull. Seism. Soc. Am.} 102(2), 661-679.

Cua, G. and D.J. Wald (2008). Calibrating PAGER (``Prompt Assessment of Global Earthquakes for
Response'') ground shaking and human impact estimation using worldwide earthquake datasets:
collaborative research with USGS and the Swiss Seismological Service, NEHRP Final Report (Award
number: 06HQGR0062).

Cua, G., D.J. Wald, T.I. Allen, D. Garcia, C.B. Worden, M. Gerstenberger, K. Lin, and K. Marano
(2010).  ``Best Practices'' for Using Macroseismic Intensity and Ground Motion to Intensity
Conversion Equations for Hazard and Loss Models, \emph{GEM Technical Report 2010-4}, Report Series,
69 pp., \url{http://www.globalquakemodel.org/node/747}.

\phantomsection\label{references:dai2010}
Dai, F.C., C. Xu, X. Yao, L. Xu, X.B. Tu, and Q.M. Gong (2010). Spatial distribution of
landslides triggered by the 2008 MS 8.0 Wenchuan earthquake, China, \emph{J. Asian Earth Sci.} 40,
883-895.

\phantomsection\label{references:dengler1998}
Dengler, L.A. and J.W. Dewey (1998). An Intensity Survey of Households Affected by the
Northridge, California, Earthquake of 17 January 1994, \emph{Bull. Seism. Soc. Am.} 88(2), 441-462.

\phantomsection\label{references:dewey1995}
Dewey, J.W., B.G. Reagor, L. Dengler, and K. Moley (1995). Intensity distribution and
isoseismal maps for the Northridge, California, earthquake of January 17, 1994, \emph{U.S.
Geological Survey Open-File Report 95-92}, 35pp.

\phantomsection\label{references:dewey2000}
Dewey, J., D.J. Wald, and L. Dengler (2000). Relating conventional USGS Modified Mercalli
Intensities to intensities assigned with data collected via the Internet \emph{Seis. Res. Lett.} 71, 264.

\phantomsection\label{references:ebel2003}
Ebel, J. and D.J. Wald (2003). Bayesian Estimations of Peak Ground Acceleration and 5\% Damped
Spectral Acceleration from Modified Mercalli Intensity Data, \emph{Earthquake Spectra} 19(3), 511-529.

Eguchi, R.T., J.D. Goltz, H.A. Seligson, P.J. Flores, N.C. Blais, T.H. Heaton, and
E. Bortugno (1997).  The Early Post-Earthquake Damage Assessment Tool (EPEDAT), \emph{Earthquake
Spectra} 13(4), 815-832.

\phantomsection\label{references:epri1991}
EPRI (1991). Standardization of cumulative absolute velocity, \emph{EPRI TR100082 (Tier 1)}, Palo Alto,
California, Electric Power Research Institute, prepared by Yankee Atomic Electric Company.

\phantomsection\label{references:epri2003}
EPRI (2003). CEUS Ground Motion Project: Model Development and Results, \emph{EPRI Report 1008910}, Palo Alto, CA, 105pp.

\phantomsection\label{references:erdik2014}
Erdik, M., K. Sesetyan, M.B. Demircioglu, C. Zulfikar, U. Hancılar, C. Tuzun, and E. Harmandar
(2014). Rapid earthquake loss assessment after damaging earthquakes,
in A. Ansal (ed.), Perspectives on European Earthquake Engineering and Seismology,
Geotechnical, \emph{Geological and Earthquake Engineering} 34. DOI: 10.1007/978-3-319-07118-3\_2.

\phantomsection\label{references:erdik2011}
Erdik, M., K. Sesetyan, M.B. Demircioglu, U. Hancılar, and C. Zulfikar
(2011). Rapid earthquake loss assessment after damaging earthquakes Soil Dynamics and Earthquake Engineering 31, 247–266.

\phantomsection\label{references:faenza2010}
Faenza, L. and A. Michilini (2010). Regression analysis of MCS intensity and ground motion
parameters in Italy and its application in ShakeMap, \emph{Geophys. J. Int.} 180, 1138–1152.

\phantomsection\label{references:fema2006}
Federal Emergency Management Agency (2006). HAZUS-MH MR2 Technical Manual: Washington, D.C.,
Federal Emergency Management Agency. \url{http://www.fema.gov/plan/prevent/hazus/hz\_manuals.shtm}.

\phantomsection\label{references:field2000}
Field, E.H. (2000). A modified ground-motion attenuation relationship for southern California that
accounts for detailed site classification and a basin-depth effect, \emph{Bull. Seism. Soc. Am.} 90, S209-S221.

\phantomsection\label{references:franco2015}
Franco, G. (2015). Earthquake Mitigation Strategies Through Insurance,
\emph{Encyclopedia of Earthquake Engineering}. DOI: 10.1007/978-3-642-36197-5\_401-1.

Frankel, A.D., M.D. Petersen, C.S. Mueller, K.M. Haller, R.L. Wheeler, E.V. Leyendecker,
R.L.  Wesson, S.C. Harmsen, C.H. Cramer, D.M. Perkins, and K.S. Rukstales (2002).
Documentation for the 2002 Update of the National Seismic Hazard Maps U.S.,
\emph{U.S. Geological Survey Open-File Report: 02-420}. \url{http://pubs.usgs.gov/of/2002/ofr-02-420/OFR-02-420.pdf}.

Garcia, D., S.K. Singh, M. Herraiz, M. Ordaz, and J.F. Pacheco (2005). Inslab earthquakes of central
Mexico: Peak ground-motion parameters and response spectra, \emph{Bull. Seism. Soc. Am} 95(6), 2272-2282.

\phantomsection\label{references:garcia2012a}
Garcia, D., R.T. Mah, K.L. Johnson, M.G. Hearne, K.D. Marano, K.W. Lin, D.J. Wald, C.B. Worden, and E.
So (2012a). ShakeMap Atlas 2.0: An Improved Suite of Recent Historical Earthquake
ShakeMaps for Global Hazard Analyses and Loss Models, \emph{Proc. 15th World Conf. on Eq. Eng.},
Lisbon, 10pp.

\phantomsection\label{references:garcia2012b}
Garcia, D., D.J. Wald, and M.G. Hearne (2012b). A Global Earthquake Discrimination Scheme to
Optimize Ground-Motion Prediction Equation Selection, \emph{Bull. Seism. Soc. Am.} 102, 185-203.

\phantomsection\label{references:godt2008}
Godt, J., B. Wener, K. Verdin, D.J. Wald, P. Earle, E. Harp, and R. Jibson (2008). Rapid assessment of
earthquake-induced landsliding, \emph{Proc. of the 1st World Landslide Forum}, Tokyo, Japan, Parallel
Sessions Volume, International Program on Landslides.

\phantomsection\label{references:gomberg2013}
Gomberg, J. and A. Jakobitz (2013). A collaborative user-producer
assessment of earthquake-response products, \emph{U.S. Geological Survey
Open-File Report 2013–1103}, 13pp. \url{http://pubs.usgs.gov/of/2013/1103/}.

\phantomsection\label{references:grunthal1998}
Grünthal, G., ed. (1998). European Macroseismic Scale 1998 (EMS-98), \emph{Cahiers du Centre Européen
de Géodynamique et de Séismologie} 15, 101pp.

Hauksson, E., L.M. Jones, and K. Hutton (2002). The 1999 Mw 7.1 Hector Mine, California,
Earthquake Sequence: Complex Conjugate Strike-Slip Faulting, \emph{Bull. Seism. Soc. Am.}
92(4), 1154–1170.

\phantomsection\label{references:ioc2012}
Intergovernmental Oceanographic Commission (IOC) (2012). Exercise
Caribe Wave/Lantex 13. A Caribbean Tsunami Warning Exercise, 20
March 2013. Volume 1: Participant Handbook. IOC Technical Series No. 101. Paris, UNESCO, 2012.

\phantomsection\label{references:jaiswal2010}
Jaiswal, K.S. and D.J. Wald (2010). An Empirical Model for Global Earthquake Fatality Estimation,
\emph{Earthquake Spectra} 26(4), 1017-1037.

\phantomsection\label{references:jaiswal2012}
Jaiswal, K.S. and D.J. Wald (2012). Estimating Economic Loss from Earthquakes Using an Empirical
Approach, \emph{Earthquake Spectra} 29(1), 309-324.

\phantomsection\label{references:jma1996}
Japan Meteorological Agency (1996). Note on the JMA seismic intensity, \emph{JMA report} 1996, Gyosei (in
Japanese).

\phantomsection\label{references:jones2011}
Jones, L. and M. Benthien (2011). Preparing for a “Big One”—The great
southern California ShakeOut, \emph{Earthquake Spectra} 27, 575–595.

Joyner, W.B. and D.M. Boore (1988). Measurement, characterization, and prediction of
strong ground-motions, in \emph{Proc. Conf. on Earthq. Eng. \& Soil Dyn. II}, Am. Soc. Civil Eng., Park City, Utah, 43-102.

Joyner, W.B. and D.M. Boore (1981). Peak horizontal accelerations and velocity from
strong-motion records including records from the 1979 Imperial Valley, California,
earthquake, \emph{Bull. Seism. Soc. Am.} 71, 2011-2038.

Kaka, S.I. and G.M. Atkinson (2004). Relationships between instrumental intensity and
ground motion parameters in eastern North America, \emph{Bull. Seism. Soc. Am.} 94, 1728-1736.

Kaka, S.I. and G.M. Atkinson (2005). Empirical ground-motion relations for ShakeMap
applications in southeastern Canada \& the northeastern United States,
\emph{Seis. Res. Lett.} 76(2), 274-282.

\phantomsection\label{references:kanamori1999}
Kanamori, H., P. Maechling, and E. Hauksson (1999). Continuous Monitoring of Ground-Motion
Parameters, \emph{Bull. Seism. Soc. Am.} 89(1), 311-316.

Kanno, T., A. Narita, N. Morikawa, H. Fujiwara, and Y. Fukushima (2006). A new attenuation relation for
strong ground motion in Japan based on recorded data, \emph{Bull. Seism. Soc. Am} 96(3), 879-897.

\phantomsection\label{references:knudsen2011}
Knudsen, K.L., and J.D.J. Bott (2011). Geologic and geomorphic evaluation of liquefaction
case histories- toward rapid hazard mapping, \emph{Seis. Res. Lett.} 82(2), 334-335.

Lin, K.W. and D.J. Wald (2008). ShakeCast Manual, \emph{U.S. Geological Survey Open File Report}
2008-1158, 90 pp.

Lin, K.W., D.J. Wald,  C.B. Worden, and A.F. Shakal (2005). Quantifying CISN ShakeMap Uncertainty,
\emph{Proc. of the California Strong Motion Instrumentation Program User's Workshop}, Los Angeles, 37-
49.

Lin, K.W. and D.J. Wald (2012). Developing Statistical Fragility Analysis Framework for the USGS
ShakeCast System for Rapid Post-Earthquake Assessment, \emph{Proc. 15th World Conf. on Eq. Eng.},
Lisbon, 10pp.

Marano, K.D., D.J. Wald, and T.I. Allen (2009). Global earthquake casualties due to
secondary effects: a quantitative analysis for improving rapid loss analyses. \emph{Natural
Hazards} 52, 319-328.

Mori, J., H. Kanamori, J. Davis, E. Hauksson, R. Clayton, T. Heaton, L. Jones, and A. Shakal (1998).
Major improvements in progress for southern California earthquake monitoring, \emph{Bull. Seism. Soc. Am.} 79, 217-221.

\phantomsection\label{references:matsuoka2015}
Matsuoka, M., K. Wakamatsu, M. Hashimoto, S. Senna, and S. Midorikawa (2015). Evaluation of
Liquefaction Potential for Large Areas Based on Geomorphologic Classification, \emph{Earthquake Spectra},
in press.

\phantomsection\label{references:musson2010}
Musson, R.M.W., G. Grunthal, and M. Stucchi (2010). The comparison of macroseismic intensity scales,
\emph{Journal of Seismology} 14, 413-428.

\phantomsection\label{references:nibs1997}
National Institute of Building Sciences (NIBS) (1997). Earthquake Loss Estimation Methodology:
HAZUS97 Technical Manual, \emph{Report prepared for the Federal Emergency Management Agency},
Washington, D.C.

NIBS (1999), HAZUS Technical Manual, SR2 edition, Vols. I, II, and III, prepared by the National
Institute of Building Sciences for the Federal Emergency Management Agency, Washington, D.C.

\phantomsection\label{references:nrc2006}
National Research Council (NRC) (2006). Improved Seismic Monitoring -
Improved Decision-Making: Assessing the Value of Reduced Uncertainty,
Couverture Committee on Seismology and Geodynamics, Committee on the
Economic Benefits of Improved Seismic Monitoring, Board on Earth
Sciences and Resources, Division on Earth and Life Studies, National Research Council
\emph{National Academies Press} 2006, 196pp. DOI: 10.17226/11327.

\phantomsection\label{references:newmark1982}
Newmark, N.M. and W.J. Hall (1982). Earthquake spectra and design, \emph{Geotechnique} 25, no. 2, 139-160.

Newmark, N.M. and W.J. Hall (1982). Earthquake Spectra and Design, \emph{Engineering Monographs on
Earthquake Criteria, Structural Design, and Strong Motion Records}, Vol. 3, Earthquake Engineering
Research Institute, University of California, Berkeley, CA.

\phantomsection\label{references:nowicki2014}
Nowicki, M.A., D.J. Wald, M.W. Hamburger, M. Hearne, and E.M. Thompson (2014). Development of
a Globally Applicable Model for Near Real-Time Prediction of Seismically Induced Landslides,
\emph{Engineering Geology}, submitted.

Pankow, K.L and J.C. Pechmann (2003). Addedum to SEA99: A new PGV and revised PGA and
pseudovelocity  predictive relationship for extensional tectonic regimes, \emph{Bull. Seism. Soc. Am.}, 364.

\phantomsection\label{references:petersen2014}
Petersen, M.D., M.P. Moschetti, P.M. Powers, C.S. Mueller, K.M. Haller, A.D. Frankel, Y.
Zeng, S. Rezaeian, S.C. Harmsen, O.S. Boyd, N. Field, R. Chen, K.S. Rukstales, N.
Luco, R.L. Wheeler, R.A. Williams, and A.H. Olsen (2014). Documentation for the 2014
update of the United States national seismic hazard maps, \emph{U.S. Geological Survey Open-File
Report} 2014–1091, 243pp. \url{http://dx.doi.org/10.3133/ofr20141091}.

\phantomsection\label{references:pomonis2011}
Pomonis, A. and E. So (2011). Guidelines for the Collection of Consequence Data, \emph{Global Earthquake
Consequences Database Global Component Project}, 71pp.
\url{http://www.nexus.globalquakemodel.org/gemecd/}.

\phantomsection\label{references:powers2008}
Powers, M., B. Chiou, N. Abrahamson, Y. Bozorgnia, T. Shantz, and C. Roblee (2008). An Overview of
the NGA Project, \emph{Earthquake Spectra} 24(1), 3-21.

\phantomsection\label{references:rowshandel2010}
Rowshandel, B. (2010). Directivity Correction for the Next Generation Attenuation (NGA)
Relations, \emph{Earthquake Spectra} 26(2), 525–559.

Scrivner, C.W., C.B. Worden, and D.J. Wald (2000). Use of TriNet ShakeMap to Manage Earthquake
Risk, \emph{Proc. of the Sixth International Conference on Seismic Zonation}, Palm Springs.

\phantomsection\label{references:seyhan2014}
Seyhan, E. and J.P. Stewart (2014). Semi-Empirical Nonlinear Site Amplification from NGA-West2 Data and Simulations, \emph{Earthquake
Spectra} 30(3), 1241-1256.

\phantomsection\label{references:shakal1998}
Shakal, A., C. Peterson, and V. Grazier (1998). Near-real-time strong motion data recovery and automated
processing for post-earthquake utilization, \emph{Proc. 6th Nat'l Conf. on Eq. Eng.}, Seattle.

Shimuzu, Y. and F. Yamasaki (1998). Real-time City Gas Network Damage Estimation System-SIGNAL,
\emph{Proc. 11th European Conf. on Eq. Eng.}, A.A. Balkema, Rotterdam.

Smith, W.H.F. and P. Wessel (1990). Gridding with continuous curvature splines in tension, \emph{Geophysics}
55, 293-305.

\phantomsection\label{references:so2014}
So, E. (2014). Introduction to the GEM Earthquake Consequences Database (GEMECD), \emph{GEM
Technical Report} 1.0.0, 158 pp., GEM Foundation, Pavia, Italy.
DOI: 10.13117/GEM.VULN-MOD.TR2014.14. \href{http://www.globalquakemodel.org/resources/publications/technical-reports/introduction-gem-earthquake-consequences-database-/}{Available online.}

Sokolov, V.Y. and Y.K. Chernov (1998). On the correlation of Seismic Intensity with Fourier Amplitude
Spectra, \emph{Earthquake Spectra} (14), 679-694.

Spudich, P., W.B. Joyner, A.G. Lindh, D.M. Boore, B.M. Margaris, and J.B. Fletcher (1999). SEA99 - A
revised ground-motion prediction relation for use in extensional tectonic regimes, \emph{Bull. Seism. Soc. Am.}
89, 1156-1170.

\phantomsection\label{references:thompson2012}
Thompson, E.M. and D.J. Wald (2012). Developing Vs30 Site-Condition Maps By Combining Observations
With Geologic And Topographic Constraints, \emph{Proc. 15th World Conf. on Eq. Eng.}, Lisbon, 9 pp.

\phantomsection\label{references:thompson2014}
Thompson, E.M., D.J. Wald, and C.B. Worden (2014).  A VS30 map for California with geologic and
topographic constraints, \emph{Bull. Seism. Soc. Am.} 104(5), 2313-2321.

\phantomsection\label{references:thompson2016}
Thompson, E.M., D.J. Wald, C.B. Worden, N. Field, N. Luco, M. D. Peterson, P. M. Powers,
and B. Rowshandel (2016).  ShakeMap Scenario Strategy, \emph{U.S. Geological Survey Open File Report},
in progress.

\phantomsection\label{references:turner2014}
Turner, L. (2014). Performance of the Caltrans ShakeCast System in the
2014 Napa M6.0 Earthquake”, \emph{Caltrans Report}, Division of Research,
Innovation, and System Information, September 2014, 14pp.

\phantomsection\label{references:turner2010}
Turner, L., D.J. Wald, and K.W. Lin (2010). ShakeCast - Developing a Tool for Rapid
Post-Earthquake Response, \emph{Final Report} CA09-0734, 325pp.

\phantomsection\label{references:usgs1999}
USGS (1999). An assessment of Seismic Monitoring in the United States: Requirements for an Advance
National Seismic System, \emph{U.S. Geological Survey Circular} 1188.

\phantomsection\label{references:verros2016}
Verros, S., M. Ganesh, M. Hearne, C.B. Worden, and D.J. Wald (2016).
Computing Spatial Correlation of Ground Motion Intensities for ShakeMap, manuscript in prep.

\phantomsection\label{references:wald1996}
Wald, D.J., T.H. Heaton, and K.W. Hudnut (1996). The Slip History of the 1994 Northridge,
California, Earthquake Determined from Strong-Motion, Teleseismic, GPS, and Leveling
Data, \emph{Bull. Seism. Soc. Am.} 86(1B), S49-S70.

Wald, D.J., T.H. Heaton, H. Kanamori, P. Maechling, and V. Quitoriano (1997). Research and
Development of TriNet ``Shake'' Maps, \emph{EOS} 78(46), F45.

Wald, D.J. (1999).  Gathering of Earthquake Shaking and Damage Information in California,
\emph{Proc. 3rd US-JAPAN High Level Policy Forum}, Yokohama, Japan.

\phantomsection\label{references:wald1999a}
Wald, D.J., V. Quitoriano, T.H. Heaton, H. Kanamori, C.W. Scrivner, and C.B. Worden (1999a).
TriNet ``ShakeMaps'': Rapid Generation of Peak Ground-motion and Intensity Maps for Earthquakes in
Southern California, \emph{Earthquake Spectra} 15(3), 537-556.

\phantomsection\label{references:wald1999b}
Wald, D.J., V. Quitoriano, T.H. Heaton, and H. Kanamori (1999b). Relationships between peak ground
acceleration, peak ground velocity, and modified Mercalli intensity in California, \emph{Earthquake
Spectra} 15, 557-564.

Wald, D.J., V. Quitoriano, L. Dengler, and J.W. Dewey (1999c). Utilization of the Internet
for Rapid Community Intensity Maps, \emph{Seis. Res. Letters} 70, 680-697.

Wald, D.J., L. Wald, J. Goltz, C.B. Worden, and C.W. Scrivner (2000). ``ShakeMaps'': Instant Maps of
Earthquake Shaking, \emph{U.S. Geological Survey Fact Sheet} 103-00.

Wald, D.J. and J. Goltz (2001). ShakeMap: A new Tool for Emergency Management and Public
Information, \emph{Proc. Los Angeles/Yokohama Disaster Prevention Workshop}, Yokohama,
Japan, November, 2001.

Wald, D.J., L. Wald, J. Dewey, V. Quitoriano, and E. Adams (2001). Did You Feel It? Community-Made
Earthquake Shaking Maps, \emph{U.S. Geological Survey Fact Sheet} 030-01.

Wald, D.J., L. Wald, C.B. Worden, and J. Goltz (2003). ShakeMap: A Tool for Earthquake Response, \emph{U.S.
Geological Survey Fact Sheet} 087-03.

Wald, D.J., P.A. Naecker, C. Roblee, and L. Turner (2003). Development of a ShakeMap-based,
earthquake response system within Caltrans, in \emph{Advancing Mitigation Technologies and Disaster
Response for Lifeline Systems}, J. Beavers, ed., Technical Council on Lifeline Earthquake Engineering,
Monograph No. 25, August 2003, ASCE.

\phantomsection\label{references:wald2005}
Wald, D.J., C.B. Worden, K.W. Lin, and K. Pankow (2005). ShakeMap
manual: technical manual, user's guide, and software guide,
U. S. Geological Survey, \emph{Techniques and Methods 12-A1}, 132 pp.
\url{http://pubs.usgs.gov/tm/2005/12A01/}

Wald, D.J., P.S. Earle, K.W. Lin, V. Quitoriano, and C.B. Worden (2006a). Challenges in Rapid Ground
Motion Estimation for the Prompt Assessment of Global Urban Earthquakes, \emph{Bull. Earthq. Res. Inst.},
Tokyo, 81, 273-282.

\phantomsection\label{references:wald2007}
Wald, D.J. and T.I. Allen (2007). Topographic slope as a proxy for seismic site conditions and
amplification, \emph{Bull. Seism. Soc. Am.} 97(5), 1379-1395.

Wald, D.J., K.W. Lin, and V. Quitoriano (2008). Quantifying and Qualifying USGS ShakeMap
Uncertainty, \emph{U.S. Geological Survey Open File Report} 2008-1238, 26pp.

\phantomsection\label{references:wald2008}
Wald, D.J., P.S. Earle, T.I. Allen, K.S. Jaiswal, K.A. Porter, and M.J. Hearne (2008). Development of
the U.S. Geological Survey's PAGER system (Prompt Assessment of Global Earthquakes for
Response), in World Conference on Earthquake Engineering, 14th, Beijing, China, October 2008,
\emph{Proc. World Conf. on Eq. Eng.} Beijing, China, Paper No. 10-0008.

\phantomsection\label{references:wald2011a}
Wald, D.J., L. McWhirter, E. Thompson, and A. Hering (2011a). A New Strategy for Developing Vs30
Maps, \emph{Proc. of the 4th International Effects of Surface Geology on Seismic Motion Symp.}, Santa
Barbara, 12pp.

\phantomsection\label{references:wald2011b}
Wald, D.J., K.S. Jaiswal, K.D. Marano, and D. Bausch (2011b). An Earthquake Impact Scale: Natural
Hazards Review, posted ahead of print. \url{http://dx.doi.org/10.1061/(ASCE)NH.1527-6996.0000040}.

\phantomsection\label{references:wald2011c}
Wald, D.J., V. Quitoriano, C.B. Worden, M. Hopper, and J.W. Dewey (2011c). USGS ``Did You
Feel It?'' internet-based macroseismic intensity maps. \emph{Annals of Geophysics} 54(6), 688-709.

\phantomsection\label{references:wald2016}
Wald, D.J. and G. Franco (2016). Applications of Near-Real time,
Post-earthquake Financial Decision-Making, \emph{Proc. 16th World Conf. on Eq. Eng.}, Santiago, Chile.

\phantomsection\label{references:gmt-ref}
Wessel, P., and W.H.F. Smith (1995). New Version of the Generic Mapping Tools Released,
\emph{EOS Trans.}, AGU, 76, 329.

\phantomsection\label{references:wgcep2003}
Working Group on California Earthquake Probabilities (WGCEP) (2003). Earthquake Probabilities
in the San Francisco Bay Region: 2003 to 2031, \emph{U.S. Geological Survey Open-File Report} 03-214.

\phantomsection\label{references:wells1994}
Wells, D.L. and K.J. Coppersmith (1994). New Empirical Relationships among Magnitude,
Rupture Length, Rupture Width, Rupture Area, and Surface Displacement, \emph{Bull. Seism. Soc.
Am.} 84(4), 974-1002.

\phantomsection\label{references:wills2000}
Wills, C.J., M.D. Petersen, W.A. Bryant, M.S. Reichle, G.J. Saucedo, S.S. Tan,
G.C. Taylor, and J.A. Treiman (2000). A site-conditions map for California based on
geology and shear wave velocity, \emph{Bull. Seism. Soc. Am.} 90, S187-S208.

Wills, C.J. and K.B. Clahan (2006). Developing a map of geologically defined site- condition
categories for California, \emph{Bull. Seism. Soc. Am.} 96, 1483-1501.

Wills, C.J. and C. Gutierrez (2008). Investigation of geographic rules for im- proving
site-conditions mapping, \emph{Calif. Geo. Sur. Final Tech. Rept.}, 20 pp. (Award No. 07HQGR0061).

Wood, H.O. and F. Neumann (1931). Modified Mercalli intensity scale of 1931, \emph{Bull. Seism.
Soc. Am.} 21, 277-283.

\phantomsection\label{references:worden2010}
Worden, C.B., D.J. Wald, T.I. Allen, K.W. Lin, D. Garcia, and G. Cua (2010). A revised
ground-motion and intensity interpolation scheme for ShakeMap, \emph{Bull. Seism. Soc. Am.}
100(6), 3083-3096.

\phantomsection\label{references:worden2011}
(See Worden et al, 2012, the actual publication date. Software was
written prior to publication.)

\phantomsection\label{references:worden2012}
Worden, C.B., M.C. Gerstenberger, D.A. Rhoades, D.J. and Wald (2012). Probabilistic
relationships between ground-motion parameters and Modified Mercalli intensity in
California \emph{Bull. Seism. Soc. Am.} 102(1), 204-221. DOI: 10.1785/0120110156.

\phantomsection\label{references:worden2015}
Worden, C.B., D.J. Wald, and E.M. Thompson (2015). Development of an Open-Source Hybrid
Global Vs30 Model, SSA Annual Meeting, Pasadena, CA. \emph{Seis. Res. Lett.} 86(2B), 713. \url{https://github.com/cbworden/earthquake-global\_vs30}.

\phantomsection\label{references:worden2016}
C.B. Worden, M. Hearne, D.J. Wald, and M. Pagani (2016). Complimentary Components of OpenQuake and ShakeMap, \emph{Proc. 16th World Conf. on Eq. Eng.}, Santiago.

\phantomsection\label{references:worden2016b}
Worden, C.B. and D.J. Wald (2016). ShakeMap Manual Online: technical manual, user's guide, and software guide,
U. S. Geological Survey. usgs.github.io/shakemap. DOI: 10.1234/012345678.

\phantomsection\label{references:yamakawa1998}
Yamakawa, K. (1998). The Prime Minister and the earthquake: Emergency Management Leadership of
Prime Minister Marayama on the occasion of the Great Hanshin-Awaji earthquake disaster, \emph{Kansai
Univ. Rev. Law and Politics} 19, 13-55.

Wu, Y.M., W.H.K. Lee, C.C. Chen, T.C. Shin, T.L. Teng, and Y.B. Tsai (2000). Performance of the
Taiwain Rapid Earthquake Information Release System (RTD) during the 1999 Chi-Chi (Taiwan)
earthquake, \emph{Seis. Res. Lett.} 71, 338-343.

Wu, Y.M., T.C. Shin, and C.H. Chang (2001). Near real-time mapping of peak ground acceleration and
peak ground velocity following a strong earthquake, \emph{Bull. Seism. Soc. Am.} 91, 1218-1228.

\phantomsection\label{references:wu2003}
Wu, Y.M., T.L. Teng, T.C. Shin, and N.C. Hsiao (2003). Relationship between peak ground
acceleration, peak ground velocity and Intensity in Taiwan, \emph{Bull. Seism. Soc. Am.} 93, 386-396.

\phantomsection\label{references:yeats2004}
Yeats, R. (2004). Living with Earthquakes in the Pacific Northwest A
Survivor's Guide, Second Edition, 400 pp. ISBN 978-0-87071-024-7.

Yong, A., S.E. Hough, J. Iwahashi, and A. Braverman (2012). A Terrain-Based Site-Conditions Map of
California with Implications for the Contiguous United States, \emph{Bull. Seism. Soc. Am.} 102, 114-128.

Yong, A., A. Martin, K. Stokoe, and J. Diehl (2013). ARRA-funded VS30 measurements using multi-
technique approach at California and central-eastern United States strong motion stations, \emph{U.S. Geological Survey Open-
File Report} 2013-1102.

\phantomsection\label{references:yong2015}
Yong, A., E.M. Thompson, D.J. Wald, K.L. Knudsen, J.K. Odum, W.J. Stephenson, and S. Haefner
(2015). A Compilation of VS30 in the United States, SSA Annual Meeting, Pasadena, CA, \emph{Seis. Res. Lett.} 86(2B), 713.

Youngs, R.R., S.J. Chiou, W.J. Silva, and J.R. Humphrey (1997). Strong ground-motion
relationships for subduction zones, \emph{Seis. Res. Lett.} 68(1), 58-73.

Zhao, J.X. (2010). Geometric spreading functions and modeling of volcanic zones for strong-motion
attenuation models derived from records in Japan, \emph{Bull. Seism. Soc. Am.} 100(2), 712-732.

Zhao, J.X., J. Zhang, A. Asano, Y. Ohno, T. Oouchi, T. Takahashi, H. Ogawa, K. Irikura, H.K. Thio, P.G.
Somerville, Y. Fukushima, and Y. Fukushima (2006). Attenuation relations of strong ground motion in
Japan using site classification based on predominant period, \emph{Bull. Seism. Soc. Am.} 96(3), 898-913.

\phantomsection\label{references:zhu2014}
Zhu, J., L.G. Baise, E.M. Thompson, D.J. Wald, and K.L. Knudsen (2014). A Geospatial Liquefaction
Model for Rapid Response and Loss Estimation, \emph{Earthquake Spectra}, in press.


\chapter{Glossary}
\label{glossary::doc}\label{glossary:glossary}
Abbreviations, acronyms, and initialisms used in this report
\begin{description}
\item[{ANSS}] \leavevmode
Advanced National Seismic System: A USGS initiative to provide seismic monitoring for at-risk
regions of the United States

\item[{California EMA}] \leavevmode
The California Emergency Management Agency (formerly the California Office of Emergency Services)

\item[{DYFI}] \leavevmode
Did You Feel It?

\item[{EHP}] \leavevmode
USGS Earthquake Hazard Program

\item[{ENS}] \leavevmode
Earthquake Notification System: earthquake notifications via text/email from USGS’s Earthquake Hazard Program

\item[{FEMA}] \leavevmode
Federal Emergency Management Agency: part of the U.S. Department of Homeland Security

\item[{GIS}] \leavevmode
Geographic Information System

\item[{GMICE}] \leavevmode
Ground-Motion--Intensity Conversion Equation

\item[{GMPE}] \leavevmode
Ground-Motion Prediction Equation

\item[{GSM}] \leavevmode
Global ShakeMap

\item[{HAZUS}] \leavevmode
HAZards US: a loss-estimation program developed by FEMA

\item[{IM}] \leavevmode
Intensity Measure: often used as a generic reference to Peak Ground Motions

\item[{IPE}] \leavevmode
Intensity Prediction Equation

\item[{ISO}] \leavevmode
International Standards Organization

\item[{MMI}] \leavevmode
Modified Mercalli Intensity

\item[{NEIC}] \leavevmode
the National Earthquake Information Center; part of the USGS, located in Golden, Colorado

\item[{PAGER}] \leavevmode
Prompt Assessment of Global Earthquakes for Response

\item[{PGA}] \leavevmode
Peak Ground Acceleration

\item[{PGM}] \leavevmode
Peak Ground Motion: a generic term for PGA and PGV

\item[{PGV}] \leavevmode
Peak Ground Velocity

\item[{USGS}] \leavevmode
the United States Geological Survey, a bureau of the U.S. Department of the Interior.

\end{description}



\renewcommand{\indexname}{Index}
\printindex
\end{document}
